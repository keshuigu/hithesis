% !Mode:: "TeX:UTF-8"
\documentclass[aspectratio=169,12pt]{beamer}

% 使用主题
\usetheme{Madrid}
\usecolortheme{default}

% 中文支持
\usepackage{xeCJK}
\setCJKmainfont{Noto Serif CJK SC}[
  BoldFont=Noto Serif CJK SC Bold,
  ItalicFont=Noto Serif CJK SC,
]
\setCJKsansfont{Noto Sans CJK SC}[
  BoldFont=Noto Sans CJK SC Bold,
]
\setCJKmonofont{Noto Sans Mono CJK SC}

% 其他宏包
\usepackage{graphicx}
\usepackage{booktabs}
\usepackage{amsmath}
\usepackage{algorithm}
\usepackage{algorithmic}
\usepackage{tikz}
\usetikzlibrary{positioning,shapes,arrows}

% 设置
\setbeamertemplate{navigation symbols}{}  % 隐藏导航符号
\setbeamertemplate{caption}[numbered]     % 图表编号
\setbeamertemplate{footline}[frame number] % 页码

% 标题信息
\title[人脸识别逆向重建方法研究]{面向人脸识别模型的逆向重建方法研究}
\subtitle{博士学位论文答辩}
\author[俞磊]{
  答辩人:俞磊\\
  导师:王莘 教授
}
\institute[哈工大]{
  哈尔滨工业大学\\
  计算学部\\
  网络空间安全
}
\date{\today}

\begin{document}

%% 标题页
\begin{frame}
\titlepage
\end{frame}

%% 目录页
\begin{frame}
\frametitle{目录}
\tableofcontents
\end{frame}

%% ============================================================
%% 第一部分:研究背景与动机
%% ============================================================
\section{研究背景与动机}

\begin{frame}
\frametitle{研究背景}
\begin{columns}[T]
  \column{0.5\textwidth}
  \textbf{人脸识别技术的广泛应用}
  \begin{itemize}
    \item 身份验证
    \item 访问控制
    \item 移动支付
    \item 安防监控
  \end{itemize}

  \vspace{0.5cm}
  \textbf{深度学习的突破}
  \begin{itemize}
    \item 特征提取能力强
    \item 识别准确率高
    \item 泛化性能优异
  \end{itemize}

  \column{0.5\textwidth}
  \begin{figure}
    \centering
    \includegraphics[width=\textwidth]{figures/face_recognition_apps.png}
    \caption{人脸识别应用场景}
  \end{figure}
\end{columns}
\end{frame}

\begin{frame}
\frametitle{隐私安全问题}
\begin{alertblock}{核心问题}
人脸识别系统在提供便捷服务的同时,面临着严重的\textbf{隐私泄露风险}
\end{alertblock}

\vspace{0.5cm}

\begin{columns}[T]
  \column{0.5\textwidth}
  \textbf{模板泄露风险}
  \begin{itemize}
    \item 特征模板存储不安全
    \item 数据库可能被攻击
    \item 模板包含敏感信息
  \end{itemize}

  \column{0.5\textwidth}
  \textbf{模型隐私风险}
  \begin{itemize}
    \item 模型记忆训练数据
    \item API接口暴露信息
    \item 输出可被恶意利用
  \end{itemize}
\end{columns}

\vspace{0.5cm}
\begin{block}{研究必要性}
需要评估系统的隐私泄露程度,为安全部署提供指导
\end{block}
\end{frame}

\begin{frame}
\frametitle{两类隐私威胁}
\begin{columns}[T]
  \column{0.5\textwidth}
  \begin{block}{模板逆向攻击 (TIA)}
  \textbf{攻击目标:}已泄露的特征模板\\
  \textbf{攻击目的:}重建可识别的人脸图像\\
  \textbf{威胁场景:}数据库泄露
  \end{block}

  \begin{figure}
    \centering
    \includegraphics[width=0.9\textwidth]{figures/tia_pipeline.png}
    \caption{TIA攻击流程}
  \end{figure}

  \column{0.5\textwidth}
  \begin{block}{模型反演攻击 (MIA)}
  \textbf{攻击目标:}训练好的分类模型\\
  \textbf{攻击目的:}重建训练数据特征\\
  \textbf{威胁场景:}API查询
  \end{block}

  \begin{figure}
    \centering
    \includegraphics[width=0.9\textwidth]{figures/mia_pipeline.png}
    \caption{MIA攻击流程}
  \end{figure}
\end{columns}
\end{frame}

\begin{frame}
\frametitle{现有方法的局限性}
\begin{table}
\centering
\small
\caption{现有攻击方法的主要问题}
\begin{tabular}{lll}
\toprule
\textbf{类别} & \textbf{代表方法} & \textbf{主要局限} \\
\midrule
TIA方法 & GAN Inversion & 视觉质量差、训练不稳定 \\
        & NBNet & 身份一致性有限 \\
\midrule
MIA方法 & GMI & 攻击成功率低 \\
        & BREP-MI & 生成多样性不足 \\
\bottomrule
\end{tabular}
\end{table}

\vspace{0.5cm}
\begin{exampleblock}{改进方向}
\begin{itemize}
  \item 引入更强大的生成先验(扩散模型、换脸模型)
  \item 设计高效的微调策略(参数高效)
  \item 建立全面的评估体系
\end{itemize}
\end{exampleblock}
\end{frame}

%% ============================================================
%% 第二部分:研究内容
%% ============================================================
\section{研究内容与方法}

\begin{frame}
\frametitle{研究内容概览}
\begin{figure}
\centering
\begin{tikzpicture}[
  node distance=1.5cm,
  box/.style={rectangle, draw, fill=blue!20, text width=3cm, text centered, rounded corners, minimum height=1cm},
  arrow/.style={->, >=stealth, thick}
]
  \node[box] (problem) {问题形式化};
  \node[box, below of=problem] (tia) {TIA方法设计};
  \node[box, below of=tia] (mia) {MIA方法设计};
  \node[box, below of=mia] (exp) {实验验证与分析};

  \draw[arrow] (problem) -- (tia);
  \draw[arrow] (tia) -- (mia);
  \draw[arrow] (mia) -- (exp);
\end{tikzpicture}
\caption{研究技术路线}
\end{figure}
\end{frame}

\subsection{TIA方法:基于扩散模型}

\begin{frame}
\frametitle{TIA方法架构}
\begin{figure}
  \centering
  \includegraphics[width=0.9\textwidth]{figures/tia_architecture.png}
  \caption{基于扩散模型的TIA方法架构}
\end{figure}

\begin{itemize}
  \item \textbf{生成先验:}明晰扩散模型(EDM)
  \item \textbf{条件引导:}交叉注意力 + Classifier-free guidance
  \item \textbf{微调策略:}LoRA参数高效微调(r=8)
\end{itemize}
\end{frame}

\begin{frame}
\frametitle{TIA关键技术}
\begin{columns}[T]
  \column{0.5\textwidth}
  \textbf{1. 条件注入机制}
  \begin{itemize}
    \item 目标模板嵌入
    \item 交叉注意力融合
    \item 引导强度调节
  \end{itemize}

  \vspace{0.3cm}
  \textbf{2. LoRA微调}
  \begin{itemize}
    \item 仅微调8.4M参数(5.9\%)
    \item 性能接近全参数微调
    \item 计算成本降低60-70\%
  \end{itemize}

  \column{0.5\textwidth}
  \textbf{3. 损失函数设计}
  \begin{equation*}
  \mathcal{L} = \mathcal{L}_{\text{denoise}} + 0.5\mathcal{L}_{\text{id}} + 0.01\mathcal{L}_{\text{reg}}
  \end{equation*}

  \vspace{0.3cm}
  其中:
  \begin{itemize}
    \item $\mathcal{L}_{\text{denoise}}$: 去噪损失
    \item $\mathcal{L}_{\text{id}}$: 身份一致性损失
    \item $\mathcal{L}_{\text{reg}}$: 正则化损失
  \end{itemize}
\end{columns}
\end{frame}

\subsection{MIA方法:基于换脸先验}

\begin{frame}
\frametitle{MIA方法架构}
\begin{figure}
  \centering
  \includegraphics[width=0.9\textwidth]{figures/mia_architecture.png}
  \caption{基于换脸先验的MIA方法架构}
\end{figure}

\begin{itemize}
  \item \textbf{生成先验:}换脸模型(SimSwap)
  \item \textbf{身份编码:}基于textual inversion的可学习嵌入
  \item \textbf{微调策略:}编码器+解码器联合微调(r=16)
\end{itemize}
\end{frame}

\begin{frame}
\frametitle{MIA关键技术}
\begin{block}{核心创新}
首次将换脸模型的身份解耦能力应用于模型反演任务
\end{block}

\vspace{0.5cm}

\begin{columns}[T]
  \column{0.5\textwidth}
  \textbf{身份嵌入学习}
  \begin{itemize}
    \item 512维可学习向量
    \item 直接在特征空间优化
    \item 相比文本嵌入提升14.1\%
  \end{itemize}

  \column{0.5\textwidth}
  \textbf{LoRA层级选择}
  \begin{itemize}
    \item 同时微调编码器和解码器
    \item 秩r=16, α=32
    \item 适配特征提取和图像生成
  \end{itemize}
\end{columns}

\vspace{0.5cm}

\begin{exampleblock}{对抗训练增强}
引入鉴别器进一步提升生成真实性,Top-1准确率从82\%提升至84\%
\end{exampleblock}
\end{frame}

%% ============================================================
%% 第三部分:实验结果
%% ============================================================
\section{实验结果与分析}

\begin{frame}
\frametitle{实验配置}
\begin{columns}[T]
  \column{0.5\textwidth}
  \textbf{数据集}
  \begin{itemize}
    \item CelebA-HQ (30K)
    \item LFW (13K)
    \item MegaFace (100K)
    \item VGGFace2 (3.3M)
  \end{itemize}

  \vspace{0.3cm}
  \textbf{基准方法}
  \begin{itemize}
    \item TIA: NBNet, DeepInversion
    \item MIA: BREP-MI, GMI
  \end{itemize}

  \column{0.5\textwidth}
  \textbf{评估指标}
  \begin{itemize}
    \item 识别一致性:TAR@FAR, Top-k Acc
    \item 视觉质量:FID, LPIPS, IS
    \item 身份保持度:余弦相似度
    \item 生成多样性:Div-LPIPS
  \end{itemize}

  \vspace{0.3cm}
  \textbf{统计分析}
  \begin{itemize}
    \item 5次重复实验
    \item 配对t检验(p<0.01)
    \item 效应量分析(Cohen's d)
  \end{itemize}
\end{columns}
\end{frame}

\begin{frame}
\frametitle{TIA实验结果}
\begin{table}
\centering
\footnotesize
\caption{TIA方法定量性能对比(LFW测试集)}
\vspace{-0.2cm}
\begin{tabular}{lcccc}
\toprule
\textbf{方法} & \textbf{TAR@FAR} & \textbf{FID↓} & \textbf{LPIPS↓} & \textbf{IS↑} \\
\midrule
GAN Inversion & 0.54 & 41.2 & 0.398 & 3.67 \\
DeepInversion & 0.61 & 38.9 & 0.376 & 3.82 \\
NBNet & 0.73 & 32.1 & 0.342 & 4.15 \\
\midrule
TIA (无微调) & 0.43 & 45.1 & 0.401 & 3.48 \\
TIA (LoRA) & \textbf{0.92} & \textbf{22.3} & \textbf{0.313} & \textbf{4.58} \\
\bottomrule
\end{tabular}
\end{table}

\vspace{-0.1cm}
\begin{block}{关键发现}
\begin{itemize}
  \item TAR@FAR提升\textbf{26\%} (0.73→0.92),FID降低\textbf{30\%}
  \item LoRA微调效果显著:TAR@FAR从43\%→92\% (相对提升\textbf{114\%})
\end{itemize}
\end{block}
\end{frame}

\begin{frame}
\frametitle{TIA定性结果}
\begin{figure}
  \centering
  \includegraphics[width=0.85\textwidth]{figures/tia_qualitative.png}
  \caption{TIA生成示例:真实图像、NBNet、TIA (LoRA r=8)}
\end{figure}

\vspace{-0.4cm}
\begin{itemize}
  \item TIA图像真实感更强,面部细节清晰
  \item 光照合理,身份特征保持准确
\end{itemize}
\end{frame}

\begin{frame}
\frametitle{MIA实验结果}
\begin{table}
\centering
\footnotesize
\caption{MIA方法定量性能对比(VGGFace2测试集)}
\vspace{-0.2cm}
\begin{tabular}{lcccc}
\toprule
\textbf{方法} & \textbf{Top-1↑} & \textbf{Top-5↑} & \textbf{FID↓} & \textbf{Div↑} \\
\midrule
GMI & 0.42 & 0.68 & 48.7 & 0.287 \\
PLGMI & 0.61 & 0.81 & 38.1 & 0.318 \\
BREP-MI & 0.68 & 0.85 & 35.4 & 0.329 \\
\midrule
MIA (文本) & 0.64 & 0.82 & 39.5 & 0.358 \\
MIA (身份) & 0.73 & 0.88 & 33.7 & 0.372 \\
MIA (LoRA) & \textbf{0.82} & \textbf{0.93} & \textbf{27.2} & \textbf{0.394} \\
\bottomrule
\end{tabular}
\end{table}

\vspace{-0.1cm}
\begin{block}{关键发现}
\begin{itemize}
  \item Top-1准确率提升\textbf{20.6\%} (0.68→0.82),FID降低\textbf{23.2\%}
  \item 生成多样性提升\textbf{19.8\%},同时保持高质量
\end{itemize}
\end{block}
\end{frame}

\begin{frame}
\frametitle{MIA定性结果}
\begin{figure}
  \centering
  \includegraphics[width=0.85\textwidth]{figures/mia_qualitative.png}
  \caption{MIA生成示例:真实图像、BREP-MI、MIA (LoRA r=16)}
\end{figure}

\vspace{-0.4cm}
\begin{itemize}
  \item 面部特征重建准确(眼睛、鼻子、嘴唇)
  \item 多样性合理,避免模式崩溃
\end{itemize}
\end{frame}

\begin{frame}
\frametitle{消融研究:LoRA秩的影响}
\begin{columns}[T]
  \column{0.5\textwidth}
  \begin{figure}
    \centering
    \includegraphics[width=\textwidth]{figures/ablation_lora_rank.png}
    \caption{不同LoRA秩的性能对比}
  \end{figure}

  \column{0.5\textwidth}
  \textbf{TIA方法}
  \begin{itemize}
    \item r=4: TAR@FAR=0.84
    \item r=8: TAR@FAR=0.92 ✓
    \item r=16: TAR@FAR=0.93
  \end{itemize}

  \vspace{0.3cm}
  \textbf{MIA方法}
  \begin{itemize}
    \item r=8: Top-1=0.78
    \item r=16: Top-1=0.82 ✓
    \item r=32: Top-1=0.83
  \end{itemize}
\end{columns}

\vspace{0.3cm}
\begin{block}{结论}
TIA最佳秩r=8,MIA最佳秩r=16,继续增大秩收益递减但成本显著增加
\end{block}
\end{frame}

\begin{frame}
\frametitle{跨识别器泛化能力}
\begin{table}
\centering
\small
\caption{TIA在不同识别器下的性能}
\begin{tabular}{lccc}
\toprule
\textbf{训练/测试识别器} & \textbf{TAR@FAR(1e-3)} & \textbf{相对下降} \\
\midrule
ArcFace / ArcFace & 0.92 ± 0.01 & -- \\
ArcFace / CosFace & 0.78 ± 0.02 & 15\% \\
ArcFace / AdaFace & 0.81 ± 0.02 & 12\% \\
\midrule
多识别器联合训练 / ArcFace & 0.90 ± 0.01 & 2\% \\
多识别器联合训练 / CosFace & 0.86 ± 0.02 & 7\% \\
多识别器联合训练 / AdaFace & 0.87 ± 0.01 & 5\% \\
\bottomrule
\end{tabular}
\end{table}

\begin{exampleblock}{改进策略}
通过多识别器联合训练,可显著提升跨识别器泛化性
\end{exampleblock}
\end{frame}

%% ============================================================
%% 第四部分:主要贡献
%% ============================================================
\section{主要贡献与创新}

\begin{frame}
\frametitle{主要贡献}
\begin{enumerate}
  \item<1->  \textbf{威胁建模系统化}
  \begin{itemize}
    \item 首次明确区分并形式化TIA与MIA两类攻击
    \item 建立统一的威胁建模框架
  \end{itemize}

  \item<2-> \textbf{方法创新}
  \begin{itemize}
    \item 首个基于扩散模型的TIA方法
    \item 首个利用换脸先验的MIA方法
    \item 系统应用LoRA参数高效微调技术
  \end{itemize}

  \item<3-> \textbf{实验贡献}
  \begin{itemize}
    \item 建立全面的多维度评估体系
    \item 提供开源实现与可复现性保障
    \item 系统化的消融研究和泛化评估
  \end{itemize}

  \item<4-> \textbf{理论洞见}
  \begin{itemize}
    \item 揭示生成先验的决定性作用
    \item 验证参数高效微调的有效性
    \item 分析身份特征与视觉质量的内在联系
  \end{itemize}
\end{enumerate}
\end{frame}

\begin{frame}
\frametitle{核心创新点}
\begin{block}{1. 生成先验的创新选择}
\begin{itemize}
  \item 扩散模型用于TIA:强大的生成能力 + 训练稳定性
  \item 换脸模型用于MIA:天然的身份解耦和保持能力
\end{itemize}
\end{block}

\begin{block}{2. 参数高效微调的系统应用}
\begin{itemize}
  \item TIA:仅5.9\%参数,TAR@FAR提升114\%
  \item MIA:同时微调编码器和解码器,Top-1提升20.6\%
  \item 计算成本降低60-70\%,显存需求减少50-60\%
\end{itemize}
\end{block}

\begin{block}{3. 条件引导机制的精细设计}
\begin{itemize}
  \item TIA:交叉注意力 + Classifier-free guidance
  \item MIA:基于textual inversion的身份嵌入学习
\end{itemize}
\end{block}
\end{frame}

\begin{frame}
\frametitle{实验方法论贡献}
\begin{columns}[T]
  \column{0.5\textwidth}
  \textbf{评估体系}
  \begin{itemize}
    \item 识别一致性指标
    \item 视觉质量指标
    \item 身份保持与多样性
  \end{itemize}

  \vspace{0.2cm}
  \textbf{统计分析}
  \begin{itemize}
    \item 配对t检验
    \item 效应量分析
    \item 多重比较校正
  \end{itemize}

  \column{0.5\textwidth}
  \textbf{消融研究}
  \begin{itemize}
    \item LoRA秩选择
    \item 采样步数与引导强度
    \item 嵌入策略与层级选择
  \end{itemize}

  \vspace{0.2cm}
  \textbf{泛化评估}
  \begin{itemize}
    \item 跨数据集测试
    \item 跨识别器测试
  \end{itemize}
\end{columns}

\vspace{0.3cm}
\begin{exampleblock}{开源贡献}
所有代码、配置、评估脚本已开源,确保可复现性
\end{exampleblock}
\end{frame}

%% ============================================================
%% 第五部分:总结与展望
%% ============================================================
\section{总结与展望}

\begin{frame}
\frametitle{研究成果总结}
\begin{block}{核心发现}
即使被认为安全的特征模板和深度模型,在面对精心设计的攻击时仍可能泄露大量隐私信息
\end{block}

\vspace{0.3cm}

\begin{columns}[T]
  \column{0.5\textwidth}
  \textbf{TIA方法}
  \begin{itemize}
    \item TAR@FAR: 92\%
    \item 相比NBNet提升26\%
    \item FID降低30\%
  \end{itemize}

  \column{0.5\textwidth}
  \textbf{MIA方法}
  \begin{itemize}
    \item Top-1 Acc: 82\%
    \item 相比BREP-MI提升20.6\%
    \item FID降低23.2\%
  \end{itemize}
\end{columns}

\vspace{0.5cm}

\begin{exampleblock}{实践意义}
\begin{itemize}
  \item 可作为安全评估工具,帮助识别系统漏洞
  \item 建立了标准化评估基准和可复现实验平台
  \item 为后续研究提供了方法论指导
\end{itemize}
\end{exampleblock}
\end{frame}

\begin{frame}
\frametitle{研究局限}
\begin{alertblock}{当前局限}
\begin{enumerate}
  \item \textbf{数据代表性}:主要基于高质量公开数据集,在野外低质量数据上的泛化能力有待验证

  \item \textbf{计算资源需求}:尽管引入LoRA降低成本,但训练推理仍需较大计算资源(10-20 GPU小时/实验)

  \item \textbf{可解释性不足}:主要从实证角度验证有效性,对深层机制缺乏充分理论解释

  \item \textbf{跨架构泛化}:针对特定识别器优化的方法在其他识别器上性能下降15-17\%
\end{enumerate}
\end{alertblock}
\end{frame}

\begin{frame}
\frametitle{未来研究方向}
\begin{columns}[T]
  \column{0.5\textwidth}
  \textbf{1. 扩展到更真实场景}
  \begin{itemize}
    \item 野外低质量数据
    \item 跨年龄、跨姿态、跨光照
    \item 视频人脸识别
    \item 其他生物特征
  \end{itemize}

  \vspace{0.3cm}
  \textbf{2. 提升效率与泛化}
  \begin{itemize}
    \item 代理模型迁移攻击
    \item 少样本/零样本反演
    \item 模型蒸馏、剪枝、量化
    \item 使用更大预训练模型
  \end{itemize}

  \column{0.5\textwidth}
  \textbf{3. 增强可解释性}
  \begin{itemize}
    \item 分析扩散时间步贡献
    \item 可视化特征空间演化
    \item 识别关键身份区域
    \item 建立可解释映射模型
  \end{itemize}

  \vspace{0.3cm}
  \textbf{4. 伦理与理论深化}
  \begin{itemize}
    \item 隐私风险量化框架
    \item 法律法规研究
    \item 隐私影响评估工具
    \item 攻击成功率理论界
    \item 博弈论分析框架
  \end{itemize}
\end{columns}
\end{frame}

\begin{frame}
\frametitle{结束语}
\begin{block}{研究意义}
本研究通过系统性的理论分析、方法创新和实验验证,揭示了人脸识别系统面临的严重隐私威胁,为构建更安全、更可信的生物识别系统提供了重要参考。
\end{block}

\vspace{0.5cm}

\begin{alertblock}{未来展望}
隐私保护是一个持续演进的过程。随着生成模型技术的不断进步,攻击方法也会变得更加强大和隐蔽。安全社区需要保持警惕,在隐私保护与系统可用性之间寻求动态平衡。
\end{alertblock}

\vspace{0.5cm}

\begin{exampleblock}{呼吁}
研究人员、系统开发者、政策制定者和用户应共同关注人脸识别技术的隐私风险,在技术创新的同时加强伦理约束与法律监管。
\end{exampleblock}
\end{frame}

%% 致谢页
\begin{frame}
\frametitle{}
\begin{center}
{\Huge 感谢各位老师和专家的聆听!}

\vspace{1cm}

{\Large 欢迎批评指正}

\vspace{2cm}

\begin{tabular}{ll}
答辩人:& 俞磊 \\
导师:& 王莘 教授 \\
单位:& 哈尔滨工业大学 计算学部 \\
\end{tabular}
\end{center}
\end{frame}

\end{document}
