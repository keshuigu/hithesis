% !Mode:: "TeX:UTF-8"

\hitsetup{
  %******************************
  % 注意:
  %   1. 配置里面不要出现空行
  %   2. 不需要的配置信息可以删除
  %******************************
  %
  %=====
  % 秘级
  %=====
  statesecrets={公开},
  natclassifiedindex={TP393.0},
  intclassifiedindex={004.8},
  %
  %=========
  % 中文信息
  %=========
  ctitleone={局部多孔质气体静压},%本科生封面使用
  ctitletwo={轴承关键技术的研究},%本科生封面使用
  ctitlecover={面向人脸识别模型的逆向重建方法研究},%放在封面中使用,自由断行
  ctitle={面向人脸识别模型的逆向重建方法研究},%放在原创性声明中使用
  % csubtitle={一条副标题}, %一般情况没有,可以注释掉
  cxueke={工学},
  csubject={网络空间安全},
  caffil={计算学部},
  cauthor={俞磊},
  csupervisor={王莘教授},
  % cassosupervisor={某某某教授}, % 副指导老师
  % ccosupervisor={某某某教授}, % 联合指导老师
  % 如果是第一封面的日期要手动设置,需要取消注释下一行,并将内容改为“规范”中要求的封面第一页最下方的日期
  % firstpagecdate={2022年6月},
  % 日期自动使用当前时间,若需指定按如下方式修改:
  % cdate={超新星纪元},
  cstudentid={9527},
  cstudenttype={学术学位论文}, %非全日制教育申请学位者
  cnumber={no9527}, %编号
  cpositionname={哈铁西站}, %博士后站名称
  cfinishdate={20XX年X月---20XX年X月}, %到站日期
  csubmitdate={20XX年X月}, %出站日期
  cstartdate={3050年9月10日}, %到站日期
  cenddate={3090年10月10日}, %出站日期
  %(同等学力人员)、(工程硕士)、(工商管理硕士)、
  %(高级管理人员工商管理硕士)、(公共管理硕士)、(中职教师)、(高校教师)等
  %
  %
  %=========
  % 英文信息
  %=========
  etitle={Inverse Reconstruction Methods for Face Recognition Models},
  % esubtitle={This is the sub title},
  exueke={Engineering},
  esubject={Cyberspace Security},
  eaffil={\emultiline[t]{Faculty of Computing}},
  eauthor={Lei Yu},
  esupervisor={Prof. Shen Wang},
  % eassosupervisor={XXX},
  % 日期自动生成,若需指定按如下方式修改:
  % edate={December, 2017},
  estudenttype={Master of Engineering},
  %
  % 关键词用“英文逗号”分割
  ckeywords={人脸识别, 模板逆向攻击, 模型反演攻击, 扩散生成模型, 隐私泄露},
  ekeywords={face recognition, template inversion attack, model inversion attack, diffusion models, privacy leakage},
}
  % 关键词是为了文献标引工作、用以表示全文主要内容信息的单词或术语。关键词不超过 5
  % 个,每个关键词中间用分号分隔。(模板作者注:关键词分隔符不用考虑,模板会自动处
  % 理。英文关键词同理。)
\begin{cabstract}
随着深度学习技术的快速发展,人脸识别系统已在身份认证、访问控制等关键领域广泛部署。然而,系统在实现高识别性能的同时,不可避免地在特征表示与模型输出中暴露了与原始图像高度相关的语义信息,为隐私攻击提供了可能。本文针对人脸识别系统面临的隐私泄露风险,系统研究了模板逆向攻击与模型反演攻击两类典型威胁,提出了基于扩散模型的高效逆向重建方法。

本文建立了攻击的形式化框架,明确了攻击者的知识边界与能力假设。针对模板逆向攻击,提出了基于扩散生成模型的方法,通过角度约束特征匹配对齐超球面特征空间,引入任务不确定性加权框架自动平衡像素去噪与特征感知的协同优化,并采用多样性正则化防止特征空间崩塌。针对模型反演攻击,本文将换脸模型的先验能力引入该任务,通过标签条件嵌入层将类别标签映射为身份向量,采用低秩适配技术仅需微调少量参数即可实现对目标分类器决策边界的精确匹配,并采用渐进式三阶段训练策略实现从图像到标签的平滑模态迁移。

系统性实验验证了所提方法的有效性与先进性。针对模板逆向攻击,所提方法在误识率$10^{-2}$阈值下平均攻击成功率达到87.50\%,显著优于次优方法Shahreza et al.的85.38\%。在误识率$10^{-3}$的严格阈值下,平均攻击成功率达到73.33\%,优于Shahreza et al.的68.81\%,在MOBIO数据集上达到88.00\%,在AgeDB和IJB-C数据集上分别达到65.67\%和55.42\%,均显著优于所有对比方法。消融实验充分验证了各模块的有效性:角度约束特征匹配显著提升了特征空间对齐精度,任务不确定性加权机制实现了像素重建与特征匹配的动态平衡,多样性正则化有效防止了模式崩塌。针对模型反演攻击,所提方法在VGGFace2数据集训练的ArcFace分类器上实现94.87\%的目标准确率和83.15\%的评估准确率,FID降至23.26,相比基线方法PLG-MI显著提升。在IR152和Face.evoLVe架构上分别达到92.45\%和91.23\%的目标准确率,展现出优异的跨架构泛化能力。渐进式三阶段训练策略使目标准确率从单阶段的70.85\%提升至94.87\%,低秩适配仅需微调1\%参数即可实现目标分类器的高效适配。实验充分证明了所提方法在攻击有效性与生成质量上的优势,揭示了人脸识别系统面临的隐私泄露风险。
\end{cabstract}

\begin{eabstract}
With the rapid development of deep learning, face recognition systems have been widely deployed in identity authentication and access control. However, these systems inevitably expose semantic information correlated with original images, providing opportunities for privacy attacks. This dissertation investigates template inversion attack and model inversion attack, proposing efficient inverse reconstruction methods based on diffusion models.

For template inversion attack, a method based on Elucidating Diffusion Models aligns with hyperspherical feature space through angle-constrained feature matching, introduces task uncertainty weighting to balance pixel denoising and feature perception, and employs diversity regularization. For model inversion attack, a label-conditioned embedding layer maps class labels to identity vectors, Low-Rank Adaptation matches target classifier decision boundaries by fine-tuning only a small fraction of parameters, and a progressive three-stage training strategy achieves smooth modal transition from images to labels.

Experiments validate the effectiveness of proposed methods. For template inversion attack, the method achieves X\% attack success rate at FMR=$10^{-3}$ on LFW, improving by X\% over GaFaR, with significant advantages on MOBIO, AgeDB, and IJB-C. Ablation studies show angle-constrained feature matching improves attack success rate from X\% to X\%, further enhanced to X\% with task uncertainty weighting, and the complete model reaches X\%. For model inversion attack, the method achieves X\% Top-1 accuracy and X FID on VGG16 classifier, reducing FID by X\% compared to PLG-MI. It achieves X\% Top-1 accuracy on IR152 and X\% on Face.evoLVe, demonstrating cross-architecture generalization. Progressive training improves Top-1 accuracy from X\% to X\%, and LoRA achieves efficient adaptation with only 0.18\% parameters. The experiments demonstrate advantages in attack effectiveness and generation quality, revealing privacy leakage risks in face recognition systems.
\end{eabstract}
