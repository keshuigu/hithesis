% !Mode:: "TeX:UTF-8"

\hitsetup{
  %******************************
  % 注意:
  %   1. 配置里面不要出现空行
  %   2. 不需要的配置信息可以删除
  %******************************
  %
  %=====
  % 秘级
  %=====
  statesecrets={公开},
  natclassifiedindex={TM301.2},
  intclassifiedindex={62-5},
  %
  %=========
  % 中文信息
  %=========
  ctitleone={局部多孔质气体静压},%本科生封面使用
  ctitletwo={轴承关键技术的研究},%本科生封面使用
  ctitlecover={面向人脸识别模型的逆向重建方法研究},%放在封面中使用,自由断行
  ctitle={面向人脸识别模型的逆向重建方法研究},%放在原创性声明中使用
  % csubtitle={一条副标题}, %一般情况没有,可以注释掉
  cxueke={工学},
  csubject={网络空间安全},
  caffil={计算学部},
  cauthor={俞磊},
  csupervisor={王莘},
  % cassosupervisor={某某某教授}, % 副指导老师
  % ccosupervisor={某某某教授}, % 联合指导老师
  % 如果是第一封面的日期要手动设置,需要取消注释下一行,并将内容改为“规范”中要求的封面第一页最下方的日期
  % firstpagecdate={2022年6月},
  % 日期自动使用当前时间,若需指定按如下方式修改:
  % cdate={超新星纪元},
  cstudentid={9527},
  cstudenttype={学术学位论文}, %非全日制教育申请学位者
  cnumber={no9527}, %编号
  cpositionname={哈铁西站}, %博士后站名称
  cfinishdate={20XX年X月---20XX年X月}, %到站日期
  csubmitdate={20XX年X月}, %出站日期
  cstartdate={3050年9月10日}, %到站日期
  cenddate={3090年10月10日}, %出站日期
  %(同等学力人员)、(工程硕士)、(工商管理硕士)、
  %(高级管理人员工商管理硕士)、(公共管理硕士)、(中职教师)、(高校教师)等
  %
  %
  %=========
  % 英文信息
  %=========
  etitle={Inverse Reconstruction Methods for Face Recognition Models},
  % esubtitle={This is the sub title},
  exueke={Engineering},
  esubject={Cyberspace Security},
  eaffil={\emultiline[t]{Faculty of Computing}},
  eauthor={Lei Yu},
  esupervisor={Prof. Shen Wang},
  % eassosupervisor={XXX},
  % 日期自动生成,若需指定按如下方式修改:
  % edate={December, 2017},
  estudenttype={Master of Engineering},
  %
  % 关键词用“英文逗号”分割
  ckeywords={人脸识别, 模板逆向攻击, 模型反演, 扩散生成模型, 隐私泄露},
  ekeywords={face recognition, template inversion attack, model inversion attack, diffusion models, privacy leakage},
}
  % 关键词是为了文献标引工作、用以表示全文主要内容信息的单词或术语。关键词不超过 5
  % 个,每个关键词中间用分号分隔。(模板作者注:关键词分隔符不用考虑,模板会自动处
  % 理。英文关键词同理。)
\begin{cabstract}
本论文研究针对人脸识别系统中两类重要的隐私威胁:一是基于模板匹配的人脸特征模板逆向重建(Template Inversion Attack,TIA);二是面向分类模型的模型反演(Model Inversion Attack,MIA)。针对TIA,提出了一种基于扩散生成模型(EDM)的条件生成方法,并结合像素级去噪与特征一致性损失,采用两阶段训练(先训练核心生成能力,再进行一致性微调)以在图像质量与特征对齐之间取得平衡;针对MIA,提出将预训练的换脸(Deepfake)生成模型作为人脸先验,通过标签嵌入与局部微调实现条件化生成,从而在仅知类别标签的情形下重建能被目标分类器识别的图像样本。实验与分析表明:在白盒或特征可查询的条件下,上述基于高质量生成先验与特征约束的方法可显著提高攻击成功率并保持较好的视觉质量;同时,研究也指出了方法在黑盒访问限制、计算代价和防御抵抗方面的局限性,并给出相应的防御建议与未来研究方向。
\end{cabstract}

\begin{eabstract}
This thesis investigates two critical privacy threats in face recognition systems: template inversion attacks (TIA) against template-matching models and model inversion attacks (MIA) against classification models. For TIA, we propose a conditional diffusion-based reconstruction method (EDM) that combines pixel-level denoising with feature-consistency loss and employ a two-stage training strategy (core EDM training followed by consistency fine-tuning) to balance visual fidelity and feature alignment. For MIA, we leverage pretrained face-swapping (deepfake) generators as priors, introduce label embeddings and localized fine-tuning to enable conditional generation that produces images recognized by the target classifier as the specified class. Empirical analysis demonstrates that, under white-box or feature-queryable settings, the proposed approaches can substantially increase attack success rates while preserving high visual quality. We also identify limitations in strict black-box scenarios, computational cost, and robustness to defenses, and provide practical countermeasures and directions for future work.
\end{eabstract}
