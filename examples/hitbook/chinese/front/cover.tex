% !Mode:: "TeX:UTF-8"

\hitsetup{
  %******************************
  % 注意:
  %   1. 配置里面不要出现空行
  %   2. 不需要的配置信息可以删除
  %******************************
  %
  %=====
  % 秘级
  %=====
  statesecrets={公开},
  natclassifiedindex={TM301.2},
  intclassifiedindex={62-5},
  %
  %=========
  % 中文信息
  %=========
  ctitleone={局部多孔质气体静压},%本科生封面使用
  ctitletwo={轴承关键技术的研究},%本科生封面使用
  ctitlecover={面向人脸识别模型的逆向重建方法研究},%放在封面中使用,自由断行
  ctitle={面向人脸识别模型的逆向重建方法研究},%放在原创性声明中使用
  % csubtitle={一条副标题}, %一般情况没有,可以注释掉
  cxueke={工学},
  csubject={网络空间安全},
  caffil={计算学部},
  cauthor={俞磊},
  csupervisor={王莘},
  % cassosupervisor={某某某教授}, % 副指导老师
  % ccosupervisor={某某某教授}, % 联合指导老师
  % 如果是第一封面的日期要手动设置,需要取消注释下一行,并将内容改为“规范”中要求的封面第一页最下方的日期
  % firstpagecdate={2022年6月},
  % 日期自动使用当前时间,若需指定按如下方式修改:
  % cdate={超新星纪元},
  cstudentid={9527},
  cstudenttype={学术学位论文}, %非全日制教育申请学位者
  cnumber={no9527}, %编号
  cpositionname={哈铁西站}, %博士后站名称
  cfinishdate={20XX年X月---20XX年X月}, %到站日期
  csubmitdate={20XX年X月}, %出站日期
  cstartdate={3050年9月10日}, %到站日期
  cenddate={3090年10月10日}, %出站日期
  %(同等学力人员)、(工程硕士)、(工商管理硕士)、
  %(高级管理人员工商管理硕士)、(公共管理硕士)、(中职教师)、(高校教师)等
  %
  %
  %=========
  % 英文信息
  %=========
  etitle={Inverse Reconstruction Methods for Face Recognition Models},
  % esubtitle={This is the sub title},
  exueke={Engineering},
  esubject={Cyberspace Security},
  eaffil={\emultiline[t]{Faculty of Computing}},
  eauthor={Lei Yu},
  esupervisor={Prof. Shen Wang},
  % eassosupervisor={XXX},
  % 日期自动生成,若需指定按如下方式修改:
  % edate={December, 2017},
  estudenttype={Master of Engineering},
  %
  % 关键词用“英文逗号”分割
  ckeywords={人脸识别, 模板逆向攻击, 模型反演, 扩散生成模型, 隐私泄露},
  ekeywords={face recognition, template inversion attack, model inversion attack, diffusion models, privacy leakage},
}
  % 关键词是为了文献标引工作、用以表示全文主要内容信息的单词或术语。关键词不超过 5
  % 个,每个关键词中间用分号分隔。(模板作者注:关键词分隔符不用考虑,模板会自动处
  % 理。英文关键词同理。)
\begin{cabstract}
随着深度学习技术的快速发展,人脸识别系统已广泛应用于身份验证、访问控制等关键领域。然而,系统中暴露的特征表示与模型输出为攻击者提供了可乘之机。本文针对人脸识别系统的两类隐私威胁展开研究:模板逆向攻击(TIA)从泄露的生物特征模板重建人脸图像,模型反演攻击(MIA)从分类模型重建特定身份样本。

本文建立了统一的白盒威胁建模框架。基于扩散概率模型理论,为逆向重建提供了数学工具与算法支撑。针对TIA,提出基于明晰扩散模型(EDM)的方法,通过混合目标函数融合像素空间重建与特征空间感知,采用梯度引导机制动态调整生成轨迹,实现精确的特征匹配。针对MIA,首次将换脸模型应用于模型反演,设计标签条件嵌入层将类别标签映射为身份向量,采用LoRA技术进行参数高效微调。

实验验证了方法的有效性。TIA方法在LFW数据集误识率$10^{-3}$阈值下攻击成功率达98.50\%,相比现有方法提升6.33\%。MIA方法Top-1准确率达93\%以上,FID值降至23.82{\textasciitilde}28.16,相比最佳基线降低约28\%。本文建立了多维度评估体系,建立了完整的理论框架与方法论规范,为人脸识别系统的隐私风险评估与防御策略设计提供了理论支撑。
\end{cabstract}

\begin{eabstract}
Face recognition systems have been widely deployed in identity verification and access control applications. However, exposed feature representations and model outputs provide opportunities for privacy attacks. This dissertation investigates two core privacy threats: Template Inversion Attack (TIA) reconstructs face images from leaked biometric templates; Model Inversion Attack (MIA) reconstructs identity samples from classification model outputs.

This dissertation establishes a unified white-box threat model where attackers have model access but no training data. Based on diffusion probabilistic model theory, it provides mathematical tools for inverse reconstruction. For TIA, an EDM-based method is proposed with hybrid objective function combining pixel-space reconstruction and feature-space perception, employing gradient guidance mechanisms to dynamically adjust generation trajectories for precise template matching. For MIA, face-swapping models are first applied to model inversion through label-conditioned embedding layers that map labels to identity vectors, with LoRA-based parameter-efficient fine-tuning.

Experimental results validate method effectiveness. The TIA method achieves 98.50\% attack success rate at FMR threshold of $10^{-3}$ on LFW, improving 6.33\% over state-of-the-art. The MIA method maintains Top-1 accuracy above 93\% with FID values of 23.82{\textasciitilde}28.16, approximately 28\% lower than the best baseline. This dissertation establishes comprehensive evaluation framework and method standards, providing theoretical support for privacy risk assessment and defense strategy design in face recognition systems.
\end{eabstract}
