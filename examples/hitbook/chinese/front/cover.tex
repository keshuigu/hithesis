% !Mode:: "TeX:UTF-8"

\hitsetup{
  %******************************
  % 注意:
  %   1. 配置里面不要出现空行
  %   2. 不需要的配置信息可以删除
  %******************************
  %
  %=====
  % 秘级
  %=====
  statesecrets={公开},
  natclassifiedindex={TP393.0},
  intclassifiedindex={004.8},
  %
  %=========
  % 中文信息
  %=========
  ctitleone={局部多孔质气体静压},%本科生封面使用
  ctitletwo={轴承关键技术的研究},%本科生封面使用
  ctitlecover={面向人脸识别模型的逆向重建方法研究},%放在封面中使用,自由断行
  ctitle={面向人脸识别模型的逆向重建方法研究},%放在原创性声明中使用
  % csubtitle={一条副标题}, %一般情况没有,可以注释掉
  cxueke={工学},
  csubject={网络空间安全},
  caffil={计算学部},
  cauthor={俞磊},
  csupervisor={王莘教授},
  % cassosupervisor={某某某教授}, % 副指导老师
  % ccosupervisor={某某某教授}, % 联合指导老师
  % 如果是第一封面的日期要手动设置,需要取消注释下一行,并将内容改为“规范”中要求的封面第一页最下方的日期
  % firstpagecdate={2022年6月},
  % 日期自动使用当前时间,若需指定按如下方式修改:
  % cdate={超新星纪元},
  cstudentid={9527},
  cstudenttype={学术学位论文}, %非全日制教育申请学位者
  cnumber={no9527}, %编号
  cpositionname={哈铁西站}, %博士后站名称
  cfinishdate={20XX年X月---20XX年X月}, %到站日期
  csubmitdate={20XX年X月}, %出站日期
  cstartdate={3050年9月10日}, %到站日期
  cenddate={3090年10月10日}, %出站日期
  %(同等学力人员)、(工程硕士)、(工商管理硕士)、
  %(高级管理人员工商管理硕士)、(公共管理硕士)、(中职教师)、(高校教师)等
  %
  %
  %=========
  % 英文信息
  %=========
  etitle={Inverse Reconstruction Methods for Face Recognition Models},
  % esubtitle={This is the sub title},
  exueke={Engineering},
  esubject={Cyberspace Security},
  eaffil={\emultiline[t]{Faculty of Computing}},
  eauthor={Lei Yu},
  esupervisor={Prof. Shen Wang},
  % eassosupervisor={XXX},
  % 日期自动生成,若需指定按如下方式修改:
  % edate={December, 2017},
  estudenttype={Master of Engineering},
  %
  % 关键词用“英文逗号”分割
  ckeywords={人脸识别, 模板逆向攻击, 模型反演攻击, 扩散生成模型, 隐私泄露},
  ekeywords={face recognition, template inversion attack, model inversion attack, diffusion models, privacy leakage},
}
  % 关键词是为了文献标引工作、用以表示全文主要内容信息的单词或术语。关键词不超过 5
  % 个,每个关键词中间用分号分隔。(模板作者注:关键词分隔符不用考虑,模板会自动处
  % 理。英文关键词同理。)
\begin{cabstract}
随着人脸识别系统在身份认证、访问控制等关键领域的广泛部署,其特征模板与模型输出的隐私泄露风险日益凸显。攻击者可能通过逆向重建手段从抽象特征或类别标签恢复出用户的面部图像,对个人隐私构成严重威胁。本研究针对模板逆向攻击与模型反演攻击两类典型威胁,系统研究了人脸识别系统的隐私脆弱性,提出两种高效逆向重建方法。

针对基于模板匹配的检索型人脸识别系统,本研究提出基于角度约束对比学习的模板逆向重建方法。该方法针对人脸识别系统在单位超球面上的几何特性,设计角度约束对比学习损失使特征优化方向与识别器决策边界精确对齐。引入任务不确定性加权框架自动平衡像素重建与特征匹配损失,避免手动调参并提升训练稳定性。采用类内多样性约束防止模式崩塌,推理阶段结合模板条件梯度引导机制提升特征匹配精度。

针对基于分类的端到端人脸识别系统,本研究提出基于换脸先验迁移的多目标自适应模型反演方法。该方法利用扩散换脸模型的身份属性解耦机制实现高保真重建。设计标签条件嵌入层将类别标签映射为身份嵌入向量,采用低秩适配技术仅需训练原模型1\%至5\%参数量即可实现参数高效微调。构建多目标损失框架涵盖扩散先验保真度、分类引导有效性等五个维度,通过任务不确定性加权机制实现自动平衡。采用渐进式三阶段训练策略,通过图像条件预热、混合条件过渡与纯标签条件适配实现平滑模态迁移。

实验验证了所提方法的有效性。针对模板逆向攻击,该方法在MOBIO数据集上于误识率为$10^{-2}$的场景下达到97.38\%的攻击成功率,平均攻击成功率87.54\%,在LFW数据集上FID降至18.27。针对模型反演攻击,该方法在ArcFace目标分类器上达到94.87\%目标准确率和83.15\%评估准确率,FID为23.26,在多个分类器架构上表现稳定。消融实验验证了各核心模块的有效性,渐进式三阶段训练策略使目标准确率从单阶段的70.85\%提升至94.87\%。研究成果揭示了人脸识别系统在特征表示与模型输出层面存在的隐私脆弱性,证明当前系统在面对逆向重建攻击时缺乏足够的防护能力,为系统安全评估与防护机制设计提供了重要参考。
\end{cabstract}

\begin{eabstract}
With the widespread deployment of face recognition systems in identity authentication and access control, privacy leakage risks in feature templates and model outputs have become increasingly prominent. Attackers may recover users' facial images from abstract features or class labels through inverse reconstruction, posing serious threats to personal privacy. This research systematically investigates privacy vulnerabilities of face recognition systems by studying template inversion attack and model inversion attack, proposing two efficient inverse reconstruction methods.

For retrieval-based face recognition systems, this research proposes a template inversion method based on angle-constrained contrastive learning. The method designs angle-constrained contrastive learning loss to align feature optimization with recognizer decision boundaries on the unit hypersphere. Task uncertainty weighting automatically balances pixel reconstruction and feature matching losses, avoiding manual tuning. Diversity constraint prevents mode collapse, while template-conditioned gradient guidance enhances feature matching precision during inference.

For classification-based face recognition systems, this research proposes a model inversion method based on face-swapping prior transfer. The method uses diffusion face-swapping models to achieve high-fidelity reconstruction. A label-conditioned embedding layer maps class labels to identity embeddings, while Low-Rank Adaptation enables efficient fine-tuning with only 1\% to 5\% parameters. A multi-objective loss framework balances five dimensions including diffusion prior fidelity and classification guidance through task uncertainty weighting. Progressive three-stage training achieves smooth transition from image conditions to label conditions through warm-up, hybrid transition, and pure label adaptation.

Experiments validate the effectiveness of proposed methods. For template inversion attack, the method achieves 97.38\% attack success rate on MOBIO at FMR=$10^{-2}$, with 87.54\% average success rate and 18.27 FID on LFW. For model inversion attack, the method achieves 94.87\% target accuracy and 83.15\% evaluation accuracy on ArcFace, with 23.26 FID and stable performance across multiple architectures. Ablation studies confirm the effectiveness of each module, with progressive training improving accuracy from 70.85\% to 94.87\%. The research reveals privacy vulnerabilities in feature representation and model output, demonstrating insufficient protection in current systems against inverse reconstruction attacks, providing important reference for security evaluation and defense design.
\end{eabstract}
