% !Mode:: "TeX:UTF-8"

\hitsetup{
  %******************************
  % 注意:
  %   1. 配置里面不要出现空行
  %   2. 不需要的配置信息可以删除
  %******************************
  %
  %=====
  % 秘级
  %=====
  statesecrets={公开},
  natclassifiedindex={TM301.2},
  intclassifiedindex={62-5},
  %
  %=========
  % 中文信息
  %=========
  ctitleone={局部多孔质气体静压},%本科生封面使用
  ctitletwo={轴承关键技术的研究},%本科生封面使用
  ctitlecover={面向人脸识别模型的逆向重建方法研究},%放在封面中使用,自由断行
  ctitle={面向人脸识别模型的逆向重建方法研究},%放在原创性声明中使用
  % csubtitle={一条副标题}, %一般情况没有,可以注释掉
  cxueke={工学},
  csubject={网络空间安全},
  caffil={计算学部},
  cauthor={俞磊},
  csupervisor={王莘},
  % cassosupervisor={某某某教授}, % 副指导老师
  % ccosupervisor={某某某教授}, % 联合指导老师
  % 如果是第一封面的日期要手动设置,需要取消注释下一行,并将内容改为“规范”中要求的封面第一页最下方的日期
  % firstpagecdate={2022年6月},
  % 日期自动使用当前时间,若需指定按如下方式修改:
  % cdate={超新星纪元},
  cstudentid={9527},
  cstudenttype={学术学位论文}, %非全日制教育申请学位者
  cnumber={no9527}, %编号
  cpositionname={哈铁西站}, %博士后站名称
  cfinishdate={20XX年X月---20XX年X月}, %到站日期
  csubmitdate={20XX年X月}, %出站日期
  cstartdate={3050年9月10日}, %到站日期
  cenddate={3090年10月10日}, %出站日期
  %(同等学力人员)、(工程硕士)、(工商管理硕士)、
  %(高级管理人员工商管理硕士)、(公共管理硕士)、(中职教师)、(高校教师)等
  %
  %
  %=========
  % 英文信息
  %=========
  etitle={Inverse Reconstruction Methods for Face Recognition Models},
  % esubtitle={This is the sub title},
  exueke={Engineering},
  esubject={Cyberspace Security},
  eaffil={\emultiline[t]{Faculty of Computing}},
  eauthor={Lei Yu},
  esupervisor={Prof. Shen Wang},
  % eassosupervisor={XXX},
  % 日期自动生成,若需指定按如下方式修改:
  % edate={December, 2017},
  estudenttype={Master of Engineering},
  %
  % 关键词用“英文逗号”分割
  ckeywords={人脸识别, 模板逆向攻击, 模型反演, 扩散生成模型, 隐私泄露},
  ekeywords={face recognition, template inversion attack, model inversion attack, diffusion models, privacy leakage},
}
  % 关键词是为了文献标引工作、用以表示全文主要内容信息的单词或术语。关键词不超过 5
  % 个,每个关键词中间用分号分隔。(模板作者注:关键词分隔符不用考虑,模板会自动处
  % 理。英文关键词同理。)
\begin{cabstract}
随着深度学习技术的快速发展,人脸识别系统已广泛应用于身份验证、访问控制等关键领域,但同时面临严重的隐私泄露风险。本文针对人脸识别系统中的两类核心隐私威胁展开研究:模板逆向攻击(TIA)旨在从泄露的生物特征模板重建人脸图像;模型反演攻击(MIA)旨在从分类模型重建训练数据中的特定身份样本。

本文建立了统一的白盒威胁建模框架,攻击者掌握模型架构与参数但无法访问训练数据。基于扩散概率模型的理论基础,为逆向重建方法提供了数学工具与算法支撑。针对TIA,本文提出基于明晰扩散模型(EDM)的方法,通过混合目标函数融合像素空间重建损失与特征空间感知损失,实现视觉质量与识别一致性的协同优化。推理阶段采用梯度引导机制动态调整生成轨迹,确保生成图像精确匹配目标模板。针对MIA,本文首次将换脸模型应用于模型反演任务,设计标签条件嵌入层将类别标签映射为身份向量。采用Low-Rank Adaptation (LoRA)技术对预训练换脸模型进行参数高效微调,通过分类器引导损失实现目标类别的精确匹配。

实验结果表明,TIA方法在LFW数据集上于误识率$10^{-3}$阈值下攻击成功率达98.50\%,相比GaFaR提升6.33\%。身份损失使攻击成功率从13\%提升至98\%以上。MIA方法Top-1准确率达93\%以上,FID值降至23.82{\textasciitilde}28.16,相比最佳基线降低约28\%,在攻击准确率与生成质量间取得优异平衡。

本文的主要贡献包括:首次系统区分并形式化了TIA与MIA两类攻击,建立了统一的威胁建模框架;提出了首个基于明晰扩散模型的TIA方法和首个利用换脸先验的MIA方法,在攻击成功率和生成质量上均实现突破;系统性应用参数高效微调技术大幅降低攻击成本;建立了全面的评估基准,提供开源实现与可复现性保障。研究成果揭示了人脸识别系统的隐私脆弱性,为安全评估和防御策略设计提供了理论与技术支撑。
\end{cabstract}

\begin{eabstract}
With the rapid development of deep learning, face recognition systems have been widely deployed in identity verification and access control, but also face serious privacy leakage risks. This dissertation investigates two core privacy threats: Template Inversion Attack (TIA) reconstructs face images from leaked biometric templates; Model Inversion Attack (MIA) reconstructs identity samples from classification model outputs.

This dissertation establishes a unified white-box threat modeling framework where attackers have model architecture and parameters but no training data access. Based on diffusion probabilistic model theory, it provides mathematical tools for inverse reconstruction methods.For TIA, this dissertation proposes an EDM-based method with hybrid objective function combining pixel-space reconstruction loss and feature-space perceptual loss, achieving coordinated optimization of visual quality and recognition consistency. The inference stage employs gradient guidance mechanisms to ensure generated images precisely match target templates.For MIA, this dissertation first applies face-swapping models to model inversion by designing a label-conditioned embedding layer mapping labels to identity vectors. Using Low-Rank Adaptation (LoRA) for parameter-efficient fine-tuning, it achieves precise target matching through classifier guidance loss.

Experimental results show that the TIA method achieves 98.50\% attack success rate at FMR threshold of $10^{-3}$, improving 6.33\% over GaFaR. Identity loss improves success rate from 13\% to over 98\%. The MIA method maintains Top-1 accuracy above 93\% with FID values of 23.82{\textasciitilde}28.16, approximately 28\% lower than the best baseline, achieving excellent balance between attack accuracy and generation quality.

The main contributions of this dissertation include: systematic formalization of TIA and MIA threat modeling framework, innovative EDM-based TIA method and face-swapping prior-based MIA method achieving breakthrough performance through hybrid loss functions and parameter-efficient fine-tuning, and establishment of comprehensive evaluation benchmarks with reproducibility guarantees. The research outcomes reveal privacy vulnerabilities of face recognition systems, providing theoretical and technical support for security assessment and defense strategy design.
\end{eabstract}
