% !Mode:: "TeX:UTF-8"

% TODO (写作提示):
% - 在方法开头形式化输入/输出:明确 $x$ (图像) 的尺寸、特征模板 $t$ 的维度、y 的含义等。
% - 在训练/推理小节后增加超参数表(batch size、lr、迭代次数、lambda 调度、随机种子)。
% - 在推理流程图下方给出示例推理命令(便于复现),例如:\texttt{python infer.py --ckpt model.pt --template tpl.npy --steps 18}
% - 若有关键实现文件或脚本,补充路径或仓库链接。

\chapter[面向人脸特征提取模型的逆向重建方法]{面向人脸特征提取模型的逆向重建方法}[Reconstruction Method for Face Feature Extraction Models]\label{chap:TIA}
\section{引言}[Introduction]

本章针对基于模板匹配的人脸识别系统,系统研究将已泄露或可获取的模板信息逆向重建为可感知图像的技术方法。具体而言,本章首先对"模板—比对"验证流程及相应的威胁模型进行严格的形式化表述,明确攻击者的能力边界、攻击目标以及成功判据;在此基础上,提出一种基于明晰扩散模型的模板逆向重建方法。该方法的核心思想是将扩散模型的生成过程与人脸识别的特征匹配目标相融合,通过设计包含像素空间重建损失与特征空间感知损失的混合目标函数,实现视觉质量与识别一致性的协同优化。为解决计算资源受限条件下的高效训练问题,本章采用了条件引导、动态权重平衡与采样投影等工程化策略,使方法在有限资源下仍能获得较高的重建质量与攻击成功率。

此外,本章还建立了一套统一的评估协议,从识别一致性、视觉感知质量以及攻击成功率等多个维度对模板逆向重建方法进行综合评价,为后续方法的对比分析提供客观、可靠的评估标准。本章所提出的方法框架、实现细节与评估体系,将为第五章的关于模板逆向重建部分的实验研究与性能分析提供完整的方法论支撑与可复现的工程流程。

\section{攻击任务与假设}[Attack Task and Assumptions]

\subsection{问题形式化}

本节对面向人脸模板逆向重建的攻击任务进行形式化描述,并明确攻击者的能力边界与成功判据。考虑一个典型的基于模板匹配的人脸识别系统,其验证流程可抽象为以下数学框架:

令原始图像空间为 $\mathcal{X} \subseteq \mathbb{R}^{H \times W \times C}$,其中 $H$、$W$ 和 $C$ 分别表示图像的高度、宽度和通道数。识别器定义为映射 $F:\mathcal{X}\to\mathbb{R}^d$,其将输入图像 $x \in \mathcal{X}$ 编码为 $d$ 维特征向量(模板)。在深度学习框架下,$F$ 通常由预训练的深度卷积神经网络实现,如 ArcFace、CosFace 或 FaceNet 等。目标用户的模板记为 $t\in\mathbb{R}^d$,该模板在注册阶段由用户的一张或多张人脸图像经特征提取后得到。

在实际系统中,模板的存储形式可能存在多样性。最简单的情况是将单次提取的特征向量直接存储,即 $t = F(x_{\text{enroll}})$,其中 $x_{\text{enroll}}$ 为注册时的人脸图像。更复杂的情况下,系统可能存储多个样本的平均特征,即 $t = \frac{1}{n}\sum_{i=1}^{n} F(x_i)$,或者保留完整的模板集合 $\{t_1, t_2, \ldots, t_n\}$。此外,模板可能携带额外的元信息,例如归一化方式($L_2$ 归一化或白化变换)、量化位宽(如将浮点数量化为定点数以节省存储空间)、加密状态等。本章假设攻击者已通过某种途径获取了目标模板 $t$ 及其相关元信息,攻击方法需要针对具体的存储格式做相应的适配处理。

对于任意待验证图像 $x \in \mathcal{X}$,系统首先计算其特征 $F(x)$,然后通过相似度函数 $d(F(x), t)$ 与预设阈值 $\tau$ 进行比对以判定身份。在人脸识别领域,相似度度量通常采用余弦相似度(Cosine Similarity),其定义为:
\begin{equation}
  d(F(x), t) = \frac{F(x)\cdot t}{\|F(x)\|_2\,\|t\|_2} = \cos\theta,
\end{equation}
其中 $\theta$ 表示两个特征向量之间的夹角。当相似度超过阈值时,即 $d(F(x),t) > \tau$,系统判定 $x$ 与注册用户为同一身份,验证通过;否则验证失败。阈值 $\tau$ 的选择通常需要在系统的误识率(False Accept Rate, FAR)与拒识率(False Reject Rate, FRR)之间进行权衡。

\subsection{威胁模型与攻击目标}

威胁模型的构建涉及攻击者的知识水平、能力边界以及攻击目标等关键要素的界定。本章考虑以下威胁场景:

(1)攻击者的知识与能力:攻击者已通过数据泄露、非法访问、内部人员泄密或传输截获等方式获得目标用户的特征模板 $t$。同时,攻击者对识别系统的体系结构具有一定了解,包括特征提取模型 $F$ 的结构、预训练权重、归一化方式以及相似度计算方法。攻击者可以通过逆向工程、模型窃取或公开渠道获取这些信息。此外,攻击者还拥有一定的计算资源用于训练生成模型和执行重建算法。

(2)攻击者的目标:攻击者的主要目标是从获得的模板信息 $t$ 构造一个生成算法 $\mathcal{G}$,输出候选重构图像 $\hat{x} = \mathcal{G}(t; \mathcal{I})$,其中 $\mathcal{I}$ 表示攻击者可获取的其他辅助信息,如识别器 $F$、或训练数据分布的先验知识等。重构图像 $\hat{x}$ 需要满足识别一致性和视觉可信性。识别一致性指的是重构图像 $\hat{x}$ 在特征空间中应与目标模板 $t$ 高度一致,即 $F(\hat{x}) \approx t$。这一要求确保重构图像能够通过身份验证,实现逆向重建攻击。视觉可信性指的是重构图像 $\hat{x}$ 在视觉上应具有合理的人脸特征和自然的外观,使其能够被人类观察者识别为真实人脸,从而在需要人工审核的场景中不易被察觉。

形式化地,攻击者的优化目标可表述为:
\begin{equation}\label{eq:attack_goal}
  \hat{x}^* = \mathop{\arg\max}_{x \in \mathcal{X}} \, \mathrm{sim}(F(x),t) \quad \text{subject to} \quad x \in \mathcal{M}_{\text{natural}},
\end{equation}
其中 $\mathrm{sim}(\cdot,\cdot)$ 为相似度函数,通常选用余弦相似度或负欧氏距离;$\mathcal{M}_{\text{natural}}$ 表示自然人脸图像分布,用于约束生成图像的视觉合理性。

\subsection{攻击成功判据}

为量化评估攻击效果,本章定义了多层次的攻击成功判据:

(1)强攻击成功:若重构图像 $\hat{x}$ 满足 $\mathrm{sim}(F(\hat{x}),t) \ge \tau$,其中 $\tau$ 为系统的验证阈值,则称该攻击在验证层面获得强成功。此时,$\hat{x}$ 可直接用于欺骗识别系统,通过身份验证。
(2)弱攻击成功:若 $\mathrm{sim}(F(\hat{x}),t)$ 虽未达到阈值 $\tau$,但显著高于随机基线(即与随机图像的平均相似度),则称该攻击获得弱成功。在此情况下,重构图像虽不能直接用于验证,但已在特征空间中接近目标模板,可能泄露目标用户的部分生物特征信息,如面部轮廓、五官布局等,从而对隐私构成威胁。
(3)视觉质量评估:除识别一致性外,还需评估重构图像的视觉质量。常用指标包括 Fréchet Inception Distance (FID)、结构相似性指数(SSIM)以及人工主观评分等。高质量的重构图像不仅能够欺骗自动识别系统,还能在需要人工审核的场景中降低被发现的风险。

本章在后续方法设计与实验评估中,将以上述判据为基础,全面衡量模板逆向重建方法的有效性与威胁程度。通过明确的形式化定义与多维度评估标准,为人脸识别系统的安全性分析提供了坚实的理论基础。

\section{基于明晰扩散模型的模板逆向重建方法}[Method Architecture Based on Elucidating Diffusion Models]
\label{sec:tia_architecture}

本章提出一种基于明晰扩散模型(Elucidating Diffusion Models, EDM)的模板逆向重建方法。该方法将扩散模型的强大生成能力与人脸识别的特征匹配目标相结合,通过设计混合损失函数和条件引导机制,实现从特征模板到可感知图像的高质量重建。

扩散模型是一类基于迭代去噪过程的深度生成模型,其核心思想是通过学习数据分布的逆向扩散过程,从纯噪声逐步生成目标样本。相较于生成对抗网络,扩散模型具有训练稳定、模式崩塌风险低、生成质量高等优势,在图像生成、图像修复、超分辨率等任务中展现出优异性能。明晰扩散模型是对传统扩散模型的改进框架,通过重新参数化噪声调度、优化采样策略以及引入可学习的噪声水平预测,进一步提升了生成质量和采样效率。

本章提出的模板逆向重建方法以EDM为基础架构,通过引入人脸识别特征约束,使生成过程不仅学习自然人脸图像的分布,还能够精确匹配目标特征模板。方法的整体框架包括两个关键阶段:

(1)特征引导的训练阶段:在标准扩散模型的去噪目标基础上,增加特征空间的感知损失,使模型在重建图像的同时保持与原始图像在特征空间中的一致性。通过动态调整损失权重,在训练过程中逐步增强特征约束的作用。(2)模板条件的推理阶段:给定目标模板,从随机噪声出发,通过条件引导的迭代去噪过程生成与模板匹配的人脸图像。在采样过程中引入梯度引导机制,动态调整生成轨迹以增强特征一致性。

图\ref{fig:edm_tia_train}展示了训练阶段的整体流程,该流程将像素空间重建与特征空间匹配有机结合,为后续的模板逆向攻击奠定了基础。

\section{混合损失函数设计}[Hybrid Loss Function Design]
\label{sec:tia_loss}

损失函数的设计是影响模型性能的关键因素。本章提出的方法采用混合损失函数,兼顾像素空间的视觉保真度与特征空间的识别一致性。

\subsection{像素空间重建损失}

像素空间的重建损失采用EDM的标准去噪目标,其形式为:
\[
  \mathcal{L}_{\text{pixel}}(\theta) = \mathbb{E}_{x_0 \sim p_{\text{data}}, \sigma \sim p_{\sigma}, \epsilon \sim \mathcal{N}(0,I)} \left[ w(\sigma) \left\|  f_\theta(x_0 + \sigma \epsilon, y, \sigma) - x_0 \right\|^2 \right],
\]
其中,$x_0 \in \mathbb{R}^{H \times W \times C}$ 表示从数据分布 $p_{\text{data}}$ 中采样的原始清晰图像,$\sigma$ 为噪声水平参数,从预设分布 $p_{\sigma}$ 中采样,$\epsilon$ 为标准正态噪声,$y$ 为条件信息,$f_\theta$ 为参数化的去噪网络,$w(\sigma)$ 为噪声水平相关的权重函数。

权重函数 $w(\sigma)$ 的设计对模型性能有重要影响。EDM采用的权重函数形式为:
\[
  w(\sigma) = \frac{1}{\sigma^2 + \sigma_{\text{data}}^2},
\]
其中 $\sigma_{\text{data}}$ 是数据的固有噪声水平。这种权重设计使得模型在不同噪声水平下都能得到均衡的训练,避免了传统扩散模型中高噪声阶段主导训练的问题。

\subsection{特征空间感知损失}

为使生成图像能够匹配目标特征模板,本章在损失函数中引入特征空间约束。令 $F: \mathbb{R}^{H \times W \times C} \to \mathbb{R}^d$ 表示预训练的人脸识别特征提取器,特征空间的感知损失定义为:
\[
  \mathcal{L}_{\text{feat}}(\theta) = \mathbb{E}_{x_0, \sigma, \epsilon} \left[ \left\| F\left( f_\theta(x_0 + \sigma \epsilon, y, \sigma) \right) - F(x_0) \right\|^2 \right].
\]

该损失项通过最小化生成图像与原始图像在特征空间中的欧氏距离,确保去噪过程不仅恢复视觉细节,还保持身份相关的语义信息。由于 $F$ 是在大规模人脸数据集上预训练的判别模型,其特征表示具有较强的身份判别能力,因此该损失项能够有效引导模型学习身份相关的特征分布。

\subsection{混合损失与动态权重调整}

综合上述两项损失,完整的训练目标函数为:
\begin{equation}\label{eq:total_loss}
  \mathcal{L}(\theta) = \mathcal{L}_{\text{pixel}}(\theta) + \lambda(t) \cdot \mathcal{L}_{\text{feat}}(\theta),
\end{equation}
其中 $\lambda(t)$ 为随训练步数 $t$ 动态调整的平衡系数。

平衡系数的设置需要在视觉质量与特征一致性之间进行权衡。若 $\lambda$ 过小,模型主要关注像素级重建,可能无法有效学习身份相关特征;若 $\lambda$ 过大,则可能导致生成图像过度拟合特征空间,出现视觉伪影或非自然纹理。为解决这一问题,本章采用渐进式的动态调整策略:
\[
  \lambda(t) = \begin{cases}
    0, & t < t_{\text{warmup}}, \\
    \lambda_{\max} \cdot \frac{t - t_{\text{warmup}}}{t_{\text{total}} - t_{\text{warmup}}}, & t_{\text{warmup}} \le t \le t_{\text{total}}, \\
    \lambda_{\max}, & t > t_{\text{total}},
  \end{cases}
\]
其中 $t_{\text{warmup}}$ 为预热步数,$t_{\text{total}}$ 为总训练步数,$\lambda_{\max}$ 为最大权重值。在训练初期($t < t_{\text{warmup}}$),模型仅优化像素空间重建,学习图像的基础结构和纹理;随后线性增加特征损失的权重,使模型逐步关注身份一致性;最终稳定在固定权重值进行精细调优。

\section{训练与推理流程}[Training and Inference Procedure]
\label{sec:tia_training_inference}

\subsection{训练流程}

基于上述损失函数,模板逆向攻击模型的训练流程如算法\ref{alg:edm_tia_train}所示。训练过程采用标准的随机梯度下降优化,每次迭代从训练集中随机采样一批数据,生成带噪样本,计算混合损失并更新模型参数。

\begin{algorithm}[H]
  \caption{模板逆向攻击模型训练}
  \label{alg:edm_tia_train}
  \begin{algorithmic}[1]
    \REQUIRE 训练样本集 $\mathcal{D} = \{(x_i, y_i)\}_{i=1}^N$,扩散生成模型 $f_\theta$,特征提取网络 $F$,学习率 $\eta$,批大小 $B$
    \ENSURE 训练后的生成模型参数 $\theta$
    \STATE 初始化模型参数 $\theta$
    \FOR{训练步数 $t = 1$ 到 $T_{\text{max}}$}
    \STATE 从 $\mathcal{D}$ 中随机采样批数据 $\{(x_0^{(j)}, y^{(j)})\}_{j=1}^B$
    \FOR{批内样本 $j = 1$ 到 $B$}
    \STATE 采样噪声水平 $\sigma^{(j)} \sim p_{\sigma}$(如对数均匀分布)
    \STATE 采样标准噪声 $\epsilon^{(j)} \sim \mathcal{N}(0, I)$
    \STATE 构造带噪输入 $x_t^{(j)} = x_0^{(j)} + \sigma^{(j)} \epsilon^{(j)}$
    \STATE 通过去噪网络预测清晰图像 $\hat{x}_0^{(j)} = f_\theta(x_t^{(j)}, y^{(j)}, \sigma^{(j)})$
    \STATE 计算像素损失 $\ell_{\text{pixel}}^{(j)} = w(\sigma^{(j)}) \| \hat{x}_0^{(j)} - x_0^{(j)} \|^2$
    \STATE 计算特征损失 $\ell_{\text{feat}}^{(j)} = \| F(\hat{x}_0^{(j)}) - F(x_0^{(j)}) \|^2$
    \ENDFOR
    \STATE 计算批平均损失 $\mathcal{L}(\theta) = \frac{1}{B} \sum_{j=1}^B \left[ \ell_{\text{pixel}}^{(j)} + \lambda(t) \cdot \ell_{\text{feat}}^{(j)} \right]$
    \STATE 计算梯度 $g = \nabla_\theta \mathcal{L}(\theta)$
    \STATE 更新参数 $\theta \leftarrow \theta - \eta \cdot g$
    \ENDFOR
    \RETURN $\theta$
  \end{algorithmic}
\end{algorithm}

在实际训练中,噪声水平 $\sigma$ 通常从对数均匀分布中采样,例如 $\log \sigma \sim \mathcal{U}(\log \sigma_{\min}, \log \sigma_{\max})$。这种采样策略确保模型在不同噪声尺度下都能得到充分训练。

\subsection{推理流程}

模型训练完成后,即可用于从目标模板重建人脸图像。推理阶段的核心是求解条件扩散过程的逆向随机微分方程(reverse SDE),从纯噪声逐步去噪至清晰图像。

\subsubsection{基础采样过程}

EDM采用确定性的ODE求解器进行采样,其离散化形式为:
\[
  x_{i-1} = x_i + (\sigma_{i-1} - \sigma_i) \cdot \frac{f_\theta(x_i, y, \sigma_i) - x_i}{\sigma_i},
\]
其中 $\{\sigma_N > \sigma_{N-1} > \cdots > \sigma_0\}$ 为预设的噪声调度序列,$N$ 为采样步数。初始状态 $x_N$ 从标准正态分布采样,即 $x_N \sim \mathcal{N}(0, \sigma_N^2 I)$。条件信息 $y$ 可以是身份标签或目标特征模板。

\subsubsection{模板条件引导}

为增强生成图像与目标模板的匹配度,在采样过程中引入梯度引导机制。具体而言,在每个去噪步骤中,除了沿着学习到的分数函数方向更新外,还额外施加一个朝向目标模板的梯度项:
\[
  x_{i-1} = x_i + (\sigma_{i-1} - \sigma_i) \cdot \left[ \frac{f_\theta(x_i, y, \sigma_i) - x_i}{\sigma_i} + \alpha \cdot \nabla_{x_i} \mathrm{sim}(F(x_i), t) \right],
\]
其中 $t \in \mathbb{R}^d$ 为目标特征模板,$\alpha$ 为引导强度超参数,$\mathrm{sim}(\cdot, \cdot)$ 为相似度函数(如余弦相似度)。引导梯度通过反向传播计算:
\[
  \nabla_{x_i} \mathrm{sim}(F(x_i), t) = \nabla_{x_i} \left[ \frac{F(x_i) \cdot t}{\|F(x_i)\|_2 \|t\|_2} \right].
\]

引导强度 $\alpha$ 的选择需要权衡生成质量与模板匹配度。较大的 $\alpha$ 能够提升特征相似度,但可能导致图像失真;较小的 $\alpha$ 则保持较好的视觉质量,但匹配度可能不足。

\subsubsection{完整推理算法}

完整的模板条件推理流程如算法\ref{alg:edm_tia_infer}所示。

\begin{algorithm}[H]
  \caption{基于目标模板的推理重建}
  \label{alg:edm_tia_infer}
  \begin{algorithmic}[1]
    \REQUIRE 目标模板 $t \in \mathbb{R}^d$,训练好的去噪网络 $f_\theta$,特征提取器 $F$,噪声调度 $\{\sigma_i\}_{i=0}^N$,引导强度 $\alpha$
    \ENSURE 重建图像 $\hat{x}_0$
    \STATE 从标准正态分布采样初始噪声 $x_N \sim \mathcal{N}(0, \sigma_N^2 I)$
    \FOR{$i = N$ 到 $1$}
    \STATE 通过去噪网络预测 $\hat{x}_0^{(i)} = f_\theta(x_i, t, \sigma_i)$
    \STATE 计算基础更新方向 $d_{\text{base}} = \frac{\hat{x}_0^{(i)} - x_i}{\sigma_i}$
    \STATE 计算特征相似度 $s = \mathrm{sim}(F(x_i), t)$
    \STATE 计算引导梯度 $g = \nabla_{x_i} s$
    \STATE 合成更新方向 $d = d_{\text{base}} + \alpha \cdot g$
    \STATE 更新状态 $x_{i-1} = x_i + (\sigma_{i-1} - \sigma_i) \cdot d$
    \STATE 可选:对 $x_{i-1}$ 进行像素值裁剪,确保在有效范围内
    \ENDFOR
    \STATE 最终输出 $\hat{x}_0 = x_0$
    \RETURN $\hat{x}_0$
  \end{algorithmic}
\end{algorithm}

图\ref{fig:edm_tia_infer}直观展示了从随机噪声到重建图像的完整推理过程,其中每一步都在去噪方向与特征引导方向的联合作用下逐步逼近目标。

\begin{figure}[!htbp]
  \centering
  \includegraphics[width=0.8\textwidth]{images/infer.drawio.pdf}
  \caption{基于EDM的模板逆向攻击模型推理流程示意图。从纯噪声 $x_N$ 出发,通过 $N$ 步迭代去噪,每步结合去噪网络预测与特征引导梯度,最终生成与目标模板匹配的人脸图像 $\hat{x}_0$。}
  \label{fig:edm_tia_infer}
\end{figure}

\section{方法优势与适用性分析}[Method Advantages and Applicability Analysis]
\label{sec:tia_advantages}

相较于传统的基于优化的逆向重建方法(如直接在像素空间进行梯度优化),本章提出的基于扩散模型的方法具有以下优势:
(1)生成质量高:扩散模型通过学习数据分布的先验知识,能够生成符合自然人脸的高质量图像,避免了直接优化方法常见的伪影和非自然纹理问题。
(2)收敛稳定:训练过程采用标准的回归目标,无需对抗训练,避免了生成对抗网络形式模型的模式崩塌和训练不稳定问题。
(3)灵活性强:通过调整引导强度 $\alpha$ 和采样步数 $N$,可以灵活平衡生成速度、视觉质量与特征匹配度,适应不同的攻击场景需求。
(4)可扩展性好:方法框架具有良好的可扩展性,可以方便地引入其他约束(如属性控制、多模板融合等)或替换为不同的扩散模型变体。

该方法适用于攻击者已知特征提取器结构与权重的白盒攻击场景。在黑盒场景下,可以使用替代模型或迁移攻击策略。此外,方法对训练数据的分布有一定依赖,当目标模板来自训练分布外的人群(如特殊族裔、极端年龄段等)时,重建质量可能下降,需要通过数据增强或领域适应技术加以改善。

\section{本章小结}[Summary]

本章针对人脸识别系统中基于模板匹配的验证机制,系统研究了面向人脸特征提取模型的逆向重建方法。通过对模板信息逆向重建为可感知图像的技术路径进行深入探讨,本章为后续章节的实验验证奠定了理论与方法论基础。

首先,本章对攻击任务与假设进行了形式化表述,明确定义了原始图像空间、识别器映射以及目标模板的数学形式,并给出了基于余弦相似度的验证机制。在此基础上,将攻击者的目标形式化为一个优化问题,即最大化重构图像与目标模板在特征空间中的相似度,为攻击成功判据提供了清晰的数学刻画。这一形式化框架不仅有助于理解模板逆向重建的本质,也为后续方法的设计与评估提供了统一的理论基础。

其次,本章提出了一种基于明晰扩散模型的模板逆向重建方法。该方法的核心创新在于将扩散生成模型的去噪过程与人脸识别的特征匹配目标相结合,通过设计包含像素空间重建损失与特征空间感知损失的混合目标函数,实现了视觉质量与识别一致性的双重优化。为平衡这两项损失的作用,本章采用了动态调整策略,在训练过程中逐步增强特征空间约束的权重,使模型在学习基础图像结构的同时,能够有效捕获目标模板的特征信息。此外,本章还详细阐述了训练与推理的完整流程,并通过算法伪码与流程图对关键步骤进行了可视化说明,为方法的实现与复现提供了清晰的指导。

综上所述,本章通过形式化的问题建模与基于扩散模型的方法设计,为面向人脸特征提取模型的逆向重建提供了系统的解决方案。该方法不仅在理论层面揭示了模板信息与图像空间之间的映射关系,也在工程层面展示了如何利用现代生成模型技术实现高质量的模板逆向重建。本章所提出的方法框架与评估标准,将为第五章的实验研究与性能分析提供重要的理论支撑与方法指导。
