% !Mode:: "TeX:UTF-8"

\chapter{基于换脸先验的模型反演攻击方法}[Model Inversion Attack Based on Face Swapping Prior]\label{chap:MIA}

\section{引言}

模型反演攻击旨在从训练好的分类模型中重建其训练数据的敏感特征,对生物特征识别系统的隐私安全构成严重威胁。传统基于梯度优化的方法直接在像素空间进行优化,面临生成质量低下、缺乏语义约束、优化效率低等问题。近年来,生成式先验模型的引入为模型反演攻击提供了新的技术路径,其中换脸模型因其独特的身份控制能力而成为理想选择。

然而,换脸模型通常需要目标图像作为身份输入,而在模型反演攻击场景中,攻击者仅拥有类别标签而无法获取目标身份的真实图像。本章针对这一核心挑战,提出了一种基于换脸先验与标签条件嵌入的模型反演攻击方法。该方法通过设计标签条件嵌入层将类别标签映射为身份表示,并采用低秩适配技术对换脸模型进行参数高效微调,在保持生成质量的同时实现对目标分类器的高效适配。

本章首先对模型反演攻击问题进行形式化定义,明确威胁模型与评估标准;随后详细阐述基于换脸先验的方法架构,包括标签条件嵌入层设计、LoRA微调策略以及多目标优化损失函数框架;最后给出完整的训练与推理流程,为方法的实现与复现提供系统化指导。

\section{问题定义与威胁模型}
\label{sec:mia_problem}

\subsection{模型反演攻击的形式化定义}

模型反演攻击旨在从训练好的分类模型中重建其训练数据的敏感特征。形式化地,设目标分类器为 $F_\theta:\mathcal{X}\to\mathbb{R}^C$,其中 $\mathcal{X}$ 为输入空间(如 $\mathbb{R}^{H\times W\times 3}$ 表示RGB图像),$C$ 为类别数量,$\theta$ 为模型参数。分类器 $F_\theta$ 通常在训练集 $\mathcal{D}_{\text{train}}=\{(x_i, y_i)\}_{i=1}^N$ 上训练,其中 $x_i\in\mathcal{X}$ 为输入样本,$y_i\in\{1,2,\ldots,C\}$ 为对应的类别标签。

模型反演攻击的目标是:给定目标类别标签 $y_{\text{target}}\in\{1,2,\ldots,C\}$ 以及对分类器 $F_\theta$ 的特定访问权限,攻击者希望生成一组图像 $\{\hat{x}_1, \hat{x}_2, \ldots, \hat{x}_K\}$,使得这些图像满足以下条件:

(1)分类器认同度(Classifier Confidence)。生成的图像 $\hat{x}_k$ 应被目标分类器 $F_\theta$ 以高置信度识别为目标类别 $y_{\text{target}}$,即:
\begin{equation}\label{eq:mia_classifier_conf}
    \Pr[F_\theta(\hat{x}_k)_{y_{\text{target}}} \geq \tau] \geq 1-\epsilon,
\end{equation}
其中 $F_\theta(\hat{x})_{y}$ 表示分类器对输入 $\hat{x}$ 预测为类别 $y$ 的置信度或logit值,$\tau$ 为置信度阈值,$\epsilon$ 为容忍的失败率。

(2)感知真实性(Perceptual Realism)。生成的图像 $\hat{x}_k$ 应具有高感知质量,在视觉上与真实人脸图像无明显差异,即:
\begin{equation}\label{eq:mia_perceptual}
    d_{\text{perc}}(\hat{x}_k, \mathcal{X}_{\text{real}}) \leq \delta,
\end{equation}
其中 $d_{\text{perc}}$ 为感知距离度量(如LPIPS、FID),$\mathcal{X}_{\text{real}}$ 为真实图像分布,$\delta$ 为可接受的感知误差上界。

(3)身份相关性(Identity Relevance)。生成的图像 $\hat{x}_k$ 应包含目标类别 $y_{\text{target}}$ 对应身份的特征信息,而非与该类别无关的随机人脸。这一要求可通过与目标类别训练样本的相似度进行量化:
\begin{equation}\label{eq:mia_identity_rel}
    \text{sim}(\hat{x}_k, \mathcal{X}_{y_{\text{target}}}) \geq \gamma,
\end{equation}
其中 $\mathcal{X}_{y_{\text{target}}}$ 为目标类别的真实训练样本(攻击者通常无法访问,仅用于评估),$\text{sim}(\cdot,\cdot)$ 为相似度度量(如人脸识别嵌入的余弦相似度),$\gamma$ 为相似度下界。

(4)多样性(Diversity)。生成的多个图像 $\{\hat{x}_k\}_{k=1}^K$ 应具有合理的多样性,覆盖目标类别在不同姿态、表情、光照、年龄等属性下的变化,避免模式崩溃(mode collapse):
\begin{equation}\label{eq:mia_diversity}
    \text{Var}(\{\hat{x}_k\}_{k=1}^K) \geq \rho,
\end{equation}
其中 $\text{Var}(\cdot)$ 为多样性度量(如特征空间的方差、聚类数量),$\rho$ 为多样性下界。

综合上述四个目标,模型反演攻击可形式化为以下约束优化问题:
\begin{equation}\label{eq:mia_objective}
\begin{aligned}
    \max_{\{\hat{x}_k\}_{k=1}^K} \quad & \sum_{k=1}^K F_\theta(\hat{x}_k)_{y_{\text{target}}} \\
    \text{s.t.} \quad & d_{\text{perc}}(\hat{x}_k, \mathcal{X}_{\text{real}}) \leq \delta, \quad \forall k, \\
    & \text{sim}(\hat{x}_k, \mathcal{X}_{y_{\text{target}}}) \geq \gamma, \quad \forall k, \\
    & \text{Var}(\{\hat{x}_k\}_{k=1}^K) \geq \rho.
\end{aligned}
\end{equation}

在实际应用中,这一多目标优化问题通常通过加权损失函数进行求解,各约束条件以正则化项或软约束的形式融入优化目标。

根据攻击者对目标分类器 $F_\theta$ 的访问权限不同,模型反演攻击可分为白盒、灰盒和黑盒三类威胁模型。本章主要关注白盒威胁模型,即攻击者拥有目标分类器的完整信息,包括模型架构、参数 $\theta$ 以及完整的梯度信息。该模型为评估模型的最大隐私泄露风险提供了上界,对应于模型开源、内部泄露或逆向工程成功的场景。在白盒威胁模型下,攻击者可以直接利用梯度信息进行基于优化的攻击,将分类器的损失函数融入生成模型的训练目标,实现端到端的优化。这一设定为深入研究模型反演攻击的方法设计与性能边界提供了理想的实验环境。

\section{基于换脸先验的模型反演架构}
\label{sec:mia_architecture}

本节详细阐述所提模型反演攻击方法的总体架构。如图~\ref{fig:mia_architecture}所示,该架构由三个核心模块组成:(1)预训练换脸模型(Face Swapping Prior Model),作为生成先验提供高质量的人脸图像生成能力;(2)标签条件嵌入层(Label-Conditioned Embedding Layer),将目标类别标签映射为换脸模型可接受的身份嵌入向量;(3)LoRA微调模块(LoRA Adaptation Module),对换脸模型进行参数高效的微调,使其适配目标分类器的决策边界。

\begin{figure}[htbp]
    \centering
    % TODO: 添加架构示意图
    % \includegraphics[width=0.9\textwidth]{figures/mia_architecture.pdf}
    \caption{基于换脸先验的模型反演攻击架构示意图。左侧为标签条件嵌入层,将类别标签映射为身份嵌入;中间为换脸模型,接受身份嵌入与源图像生成目标图像;右侧为目标分类器,提供分类损失用于微调。}
    \label{fig:mia_architecture}
\end{figure}

\textbf{架构概览}:整体架构采用三模块级联设计。(1)\textit{标签条件嵌入模块}:输入为类别标签$y\in\{1,\ldots,C\}$的one-hot编码,通过3层MLP映射为512维归一化身份嵌入$e_{\text{id}}\in\mathbb{R}^{512}$,各层配备ReLU激活与LayerNorm;(2)\textit{换脸生成模块}:基于REFace的扩散模型,输入身份嵌入$e_{\text{id}}$与随机采样的源图像$x_{\text{src}}$,通过参考注意力机制将身份信息注入U-Net去噪网络,在潜在空间执行50步DDIM采样后解码生成目标图像$\hat{x}$;(3)\textit{分类器反馈模块}:目标分类器$F_\theta$接收生成图像,输出logit向量用于计算分类损失$\mathcal{L}_{\text{cls}}$,梯度反向传播至嵌入层与LoRA参数。训练阶段冻结分类器与换脸主干,仅优化嵌入层(阶段1)与LoRA低秩矩阵(阶段2),推理阶段可合并LoRA权重实现单次前向生成。数据流:标签$y\xrightarrow{\text{Embed}}e_{\text{id}}\xrightarrow{\text{FaceSwap}}\hat{x}\xrightarrow{\text{Classify}}\ell_y$,损失流:$\mathcal{L}_{\text{total}}\xrightarrow{\nabla}\psi,\{A,B\}$。

整个系统的前向传播流程可形式化描述如下:

(1)标签到嵌入的映射。给定目标类别标签 $y_{\text{target}}\in\{1,2,\ldots,C\}$,标签条件嵌入层 $\mathcal{E}_{\psi}$ 将其映射为身份嵌入向量 $e_{\text{id}}\in\mathbb{R}^{d_e}$:
\begin{equation}\label{eq:mia_label_to_emb}
    e_{\text{id}} = \mathcal{E}_{\psi}(y_{\text{target}}),
\end{equation}
其中 $\psi$ 为嵌入层的可学习参数,$d_e$ 为嵌入维度(通常为512或1024)。

(2)换脸生成。预训练换脸模型 $G_{\phi}$ 接受身份嵌入 $e_{\text{id}}$ 与源图像 $x_{\text{src}}$(提供姿态、表情、光照等属性信息)作为输入,生成目标图像 $\hat{x}$:
\begin{equation}\label{eq:mia_face_swap}
    \hat{x} = G_{\phi}(e_{\text{id}}, x_{\text{src}}),
\end{equation}
其中 $\phi$ 为换脸模型的参数。在标准换脸任务中,$e_{\text{id}}$ 通常由目标图像通过身份编码器提取;而在本章的模型反演任务中,$e_{\text{id}}$ 由标签条件嵌入层直接生成,无需真实的目标图像。

(3)分类器评估与反馈。生成的图像 $\hat{x}$ 输入目标分类器 $F_\theta$,获得对目标类别的置信度:
\begin{equation}\label{eq:mia_classifier_eval}
    p_{y_{\text{target}}} = F_\theta(\hat{x})_{y_{\text{target}}}.
\end{equation}
该置信度用于计算分类器引导损失,并通过反向传播更新标签条件嵌入层 $\mathcal{E}_{\psi}$ 与换脸模型 $G_{\phi}$的参数。

\subsection{换脸先验模型的选择与分析}

换脸模型作为本章方法的核心生成先验,其选择对最终攻击性能有决定性影响。理想的换脸模型应满足以下要求:

(1)高质量的身份迁移能力。换脸模型应能够在保持源图像属性的前提下,精确地将目标身份特征迁移至生成图像中,确保生成图像的身份一致性。

(2)解耦的身份-属性表示。换脸模型应具有显式的身份-属性解耦机制,使得身份信息与姿态、表情、光照等属性信息分离,便于通过修改身份嵌入实现对生成结果的精确控制。

(3)高感知质量与真实感。生成的图像应在视觉上与真实人脸无明显差异,包括皮肤纹理、光照一致性、边界融合等细节。

(4)计算效率与可微性。换脸模型应具有合理的计算开销,支持端到端的梯度反向传播,以便与目标分类器联合训练。

基于上述要求,本章选择REFace~\cite{sanoojan2024reface}作为换脸先验模型。REFace(Reference-based Face Swapping)是一种基于扩散模型的换脸方法,通过参考注意力机制实现高保真的身份迁移。相比传统GAN方法,REFace具有生成质量高、身份-属性解耦能力强、训练过程稳定且支持端到端梯度传播等优势,使其成为模型反演攻击的理想生成先验。

REFace的核心机制包括:(1)身份编码器:使用预训练ArcFace模型将输入图像映射为512维身份嵌入$e_{\text{id}} = E_{\text{id}}(x) \in \mathbb{R}^{512}$;(2)参考注意力:通过交叉注意力将身份嵌入注入U-Net去噪网络各层,计算为$\text{RefAttn}(Q, K_{\text{ref}}, V_{\text{ref}}) = \text{Softmax}(QK_{\text{ref}}^T/\sqrt{d}) V_{\text{ref}}$,其中$K_{\text{ref}}, V_{\text{ref}}$由身份嵌入投影得到;(3)扩散去噪:基于潜在扩散模型在VAE潜在空间进行条件生成。在标准REFace流程中,身份嵌入由参考图像提取;而在本章的模型反演任务中,该步骤被标签条件嵌入层替代,无需真实目标图像即可实现身份控制。REFace的预训练模型在大规模人脸数据集上学习到强大的生成先验与身份迁移能力,为模型反演攻击提供了高质量的基础。

\subsection{标签条件嵌入层设计}

标准换脸模型接受目标图像作为输入,通过身份编码器提取身份嵌入。然而,在模型反演攻击场景中,攻击者仅拥有目标类别标签,无法获取真实的目标图像。因此,需要设计标签条件嵌入层来替代身份编码器,实现从离散标签到连续嵌入的映射。本文采用基于多层感知机(MLP)的嵌入层,将类别标签的one-hot编码映射为身份嵌入向量:
\begin{equation}\label{eq:mia_mlp_emb}
    e_{\text{id}} = \text{MLP}_{\psi}(\text{OneHot}(y_{\text{target}})),
\end{equation}
其中 $\text{MLP}_{\psi}$ 为多层感知机,$\psi$ 为其参数。MLP通常包含2-3个隐藏层,配备ReLU激活函数与LayerNorm:
\begin{equation}\label{eq:mia_mlp_structure}
    \text{MLP}_{\psi}(x) = W_3 \cdot \text{ReLU}(\text{LN}(W_2 \cdot \text{ReLU}(\text{LN}(W_1 \cdot x + b_1)) + b_2)) + b_3,
\end{equation}
其中 $W_1\in\mathbb{R}^{d_h\times C}, W_2\in\mathbb{R}^{d_h\times d_h}, W_3\in\mathbb{R}^{d_e\times d_h}$ 为权重矩阵,$d_h$ 为隐藏层维度,$d_e$ 为输出嵌入维度。基于MLP的嵌入层通过非线性变换能够学习类别标签与身份嵌入之间的复杂映射关系,相比查找表方法具有更强的表达能力,且参数量适中便于优化。

为确保标签条件嵌入层生成的身份嵌入 $e_{\text{id}}$ 与换脸模型的ID注入模块兼容,需要保证维度匹配与分布匹配。首先,MLP的输出维度 $d_e$ 应与换脸模型的身份编码器输出维度一致。对于REFace,该维度为512(ArcFace嵌入维度),因此MLP的最后一层输出维度设置为512。其次,嵌入向量的数值分布应与真实身份嵌入的分布相近,以保证换脸模型的正常工作。实践中,通过归一化约束对嵌入向量进行 $L_2$ 归一化,使其位于单位超球面上,与ArcFace嵌入的归一化特性一致:
\begin{equation}\label{eq:mia_emb_norm}
    e_{\text{id}} \leftarrow \frac{e_{\text{id}}}{\|e_{\text{id}}\|_2}.
\end{equation}
此外,在损失函数中添加范数正则项,鼓励嵌入向量的范数接近真实嵌入的平均范数:
\begin{equation}\label{eq:mia_norm_reg}
    \mathcal{L}_{\text{norm}} = (\|e_{\text{id}}\|_2 - \mu_{\text{norm}})^2,
\end{equation}
其中 $\mu_{\text{norm}}$ 为真实嵌入范数的统计平均值(对于归一化后的ArcFace嵌入,$\mu_{\text{norm}}=1$)。

\section{基于LoRA的换脸模型微调方法}
\label{sec:mia_lora}

本节详细阐述如何使用低秩适配(LoRA)技术对预训练换脸模型进行参数高效的微调。

\subsection{LoRA应用于换脸模型的动机}

直接对换脸模型进行全参数微调面临参数量巨大、过拟合风险高、可能破坏预训练知识等挑战。LoRA技术通过在冻结预训练权重的前提下引入低秩可训练增量,有效解决了上述问题:仅需训练与存储低秩矩阵,参数量降低至原模型的1\%-5\%;低秩约束起到隐式正则化作用,降低过拟合风险;原始权重冻结,预训练的生成先验得以保留。模块化部署:多个任务共享基础模型,仅需保存各自的LoRA权重,存储开销极低。

\subsection{LoRA在换脸模型中的应用位置}

根据换脸模型的架构特点与模型反演任务的需求,LoRA应优先应用于以下关键模块:

\subsubsection{参考注意力的投影矩阵}

REFace的参考注意力机制通过键值投影将身份嵌入映射到注意力空间。对于第 $\ell$ 层的参考注意力,键值投影定义为:
\begin{equation}\label{eq:mia_lora_ref_attn}
    K_{\text{ref}} = W_K^{\ell} \cdot e_{\text{id}}, \quad V_{\text{ref}} = W_V^{\ell} \cdot e_{\text{id}},
\end{equation}
其中 $W_K^{\ell}, W_V^{\ell}$ 为键值投影矩阵。

对这些投影矩阵应用LoRA:
\begin{equation}\label{eq:mia_lora_ref_proj}
\begin{aligned}
    W_K^{\ell\prime} &= W_K^{\ell} + \frac{\alpha}{r} B_K^{\ell} A_K^{\ell}, \\
    W_V^{\ell\prime} &= W_V^{\ell} + \frac{\alpha}{r} B_V^{\ell} A_V^{\ell},
\end{aligned}
\end{equation}
其中 $A_K^{\ell}, A_V^{\ell}\in\mathbb{R}^{r\times 512}$,$B_K^{\ell}, B_V^{\ell}\in\mathbb{R}^{d_h\times r}$。

应用LoRA到参考注意力投影矩阵的优势在于:这些矩阵直接控制身份嵌入如何被编码到注意力空间中,是身份信息注入的核心机制。微调这些参数能够调整模型对身份特征的关注程度,使生成图像更易被目标分类器识别。

\subsubsection{U-Net的自注意力和交叉注意力层}

REFace的去噪U-Net包含多层自注意力和交叉注意力模块。自注意力用于特征的全局建模,交叉注意力(除参考注意力外)用于融合时间步条件等信息。应在这些注意力的投影矩阵上应用LoRA:
\begin{equation}\label{eq:mia_lora_attn}
\begin{aligned}
    W_Q' &= W_Q + \frac{\alpha}{r} B_Q A_Q, \\
    W_K' &= W_K + \frac{\alpha}{r} B_K A_K, \\
    W_V' &= W_V + \frac{\alpha}{r} B_V A_V, \\
    W_O' &= W_O + \frac{\alpha}{r} B_O A_O,
\end{aligned}
\end{equation}
其中 $W_Q, W_K, W_V, W_O$ 分别为Query、Key、Value与输出投影矩阵。

注意力机制负责特征的全局聚合与重组,对生成图像的整体结构与细节有重要影响。微调注意力投影可以调整模型关注的特征区域,优先生成被分类器敏感的身份特征(如眼睛、鼻子、嘴巴的形状)。

\subsubsection{U-Net的残差块卷积层}

REFace的U-Net架构包含多个残差块(ResNet Block),每个残差块含有多个卷积层。可以在残差块中的$1\times1$点卷积层上应用LoRA。点卷积负责通道间的特征混合,其权重矩阵 $W\in\mathbb{R}^{C_{\text{out}}\times C_{\text{in}}\times 1\times 1}$ 可重塑为二维矩阵后应用LoRA分解:
\begin{equation}\label{eq:mia_lora_conv}
    W' = W + \frac{\alpha}{r} \text{reshape}(B A, [C_{\text{out}}, C_{\text{in}}, 1, 1]),
\end{equation}
其中 $A\in\mathbb{R}^{r\times C_{\text{in}}}, B\in\mathbb{R}^{C_{\text{out}}\times r}$。

\section{多目标优化损失函数设计}
\label{sec:mia_loss}

本节系统设计面向模型反演攻击的多目标优化损失函数。该损失函数需要平衡五个相互制约的优化目标:扩散先验保真度、分类器攻击有效性、身份一致性、生成质量与模型正则化。为实现各目标的高效协同,本文采用任务不确定性加权框架进行自动权重调整,并针对每个损失项设计针对性的优化策略。

\subsection{总体损失架构}

本文采用任务不确定性加权框架统一各损失项,避免手动权重调优的复杂性:
\begin{equation}\label{eq:mia_total_loss}
    \mathcal{L}_{\text{total}} = \sum_{i \in \{\text{prior, cls, id, perc, reg}\}} \left( \frac{1}{2\sigma_i^2} \mathcal{L}_i + \frac{1}{2}\log\sigma_i^2 \right),
\end{equation}
其中 $\sigma_i$ 为第 $i$ 个任务的可学习不确定性参数,与网络参数联合优化。该框架自动将权重分配偏向于较难的任务($\sigma_i$ 较小),确保各目标均衡发展。训练时 $\sigma_i$ 采用较小学习率(主学习率的10\%)以避免早期过度衰减。

\subsection{时间自适应的扩散先验损失}
\label{subsec:mia_prior}

扩散先验损失确保换脸模型的生成质量不被过度破坏。传统L2范数损失在高相似度时梯度消失,且未考虑不同噪声阶段的特性差异。本文采用余弦相似度度量并引入时间自适应权重:
\begin{equation}\label{eq:mia_prior_loss}
    \mathcal{L}_{\text{prior}} = \mathbb{E}_{z_0, \epsilon\sim\mathcal{N}(0,I), t} \left[ w(t) \cdot \left(1 - \frac{\langle \epsilon, \epsilon_\theta(z_t, t, e_{\text{id}}) \rangle}{\|\epsilon\|_2 \|\epsilon_\theta(z_t, t, e_{\text{id}})\|_2} \right) \right],
\end{equation}
其中时间步权重 $w(t)$ 根据扩散阶段动态调整:在高噪声阶段($t<0.2T$)采用 $w(t)=0.5$,中间阶段($0.2T \le t < 0.8T$)采用 $w(t)=1.0$,低噪声阶段($t \ge 0.8T$)采用 $w(t)=1.5$。该设计反映了低噪声阶段细节塑造的重要性,同时避免尺度敏感性。在分阶段训练中,第1阶段使用较大先验权重($\lambda_{\text{prior}}=2.0$)强力保护预训练知识,第2阶段衰减至0.5以增强任务适配能力。

\subsection{自适应分类器引导损失}

分类器引导损失驱动生成图像被目标分类器识别为指定类别。本文采用自适应top-k max-margin损失替代传统交叉熵,并引入可学习特征中心:
\begin{equation}\label{eq:mia_topk_loss}
    \mathcal{L}_{\text{cls}} = -\ell_y + \frac{1}{k_{\text{adapt}}} \sum_{j \in \text{top-}k_{\text{adapt}}(J \setminus \{y\})} \ell_j,
\end{equation}
其中 $k_{\text{adapt}} = \max(2, \min(5, \lfloor C / 20 \rfloor))$ 根据类别总数 $C$ 自适应调整,$\ell_y$ 为目标类别logit。采用均值聚合(mean)替代最大值聚合(max)提供更稳定的梯度流。同时引入基于可学习特征中心的p-reg约束:
\begin{equation}\label{eq:mia_preg_loss}
    \mathcal{L}_{\text{p-reg}} = \left\|p_x - c_y\right\|_2^2,
\end{equation}
其中特征中心 $c_y$ 通过动量更新自动学习:$c_y \leftarrow (1-\rho) c_y + \rho \cdot E_{\text{id}}(\hat{x})$,$\rho=0.01$。该设计直接学习类别特征原型,避免伪标签噪声的影响。

\subsection{对比学习的身份一致性损失}

身份一致性损失确保生成图像包含目标身份特征。本文采用对比学习框架,充分利用面部识别系统的单位超球面几何性质,显式增强真实身份特征与负样本特征的角度分离:
\begin{equation}\label{eq:mia_identity_loss}
    \mathcal{L}_{\text{id}} = \mathbb{E}\left[\max\left(0, m + \frac{\langle E_{\text{id}}(\hat{x}), e_{\text{neg}} \rangle}{\|E_{\text{id}}(\hat{x})\|_2 \|e_{\text{neg}}\|_2} - \frac{\langle E_{\text{id}}(\hat{x}), e_{\text{id}} \rangle}{\|E_{\text{id}}(\hat{x})\|_2 \|e_{\text{id}}\|_2}\right)\right],
\end{equation}
其中 $m \in [0.3, 0.5]$ 为角度裕度,$e_{\text{neg}}$ 为从内存库均匀采样的负样本特征。

\textbf{负样本采样策略}:维护包含$N=10000$个归一化身份嵌入的内存库$\mathcal{M}=\{e_1, \ldots, e_N\}$,通过对大规模人脸数据集随机采样并经ArcFace编码初始化。训练时,对批内每个样本,从$\mathcal{M}$中均匀随机采样$K_{\text{neg}}=5$个负样本计算式~\ref{eq:mia_identity_loss},取平均作为批损失。内存库每隔100步更新:将当前批生成图像的身份嵌入$E_{\text{id}}(\hat{x})$以0.1的概率替换随机选中的内存库条目,保持负样本分布的时效性。该策略与ArcFace的负样本策略对齐,通过大规模负样本集合增强判别性,同时动态更新避免分布漂移。相比批内负采样,独立内存库提供更丰富的负样本多样性;相比固定负样本集,动态更新机制适应训练过程中嵌入空间的演化。

该损失与ArcFace架构对齐,显式强制学习特征在单位超球面上保持较大角度分离,相比简单余弦相似度具有更强的判别性。综合分类与身份约束:
\begin{equation}\label{eq:mia_cls_id_combined}
    \mathcal{L}_{\text{cls+id}} = \mathcal{L}_{\text{cls}} + \alpha \mathcal{L}_{\text{p-reg}} + \lambda_{\text{id}} \mathcal{L}_{\text{id}},
\end{equation}
其中 $\alpha \in [0.5, 1.0]$,$\lambda_{\text{id}} \in [0.3, 1.0]$。

\subsection{属性保持与多样性感知损失}

感知质量损失确保生成图像的视觉真实性。本文在深度特征LPIPS基础上,加入显式属性保持约束与多样性约束。属性保持损失确保源图像的姿态、表情、光照等属性正确迁移:
\begin{equation}\label{eq:mia_attr_loss}
    \mathcal{L}_{\text{attr}} = \sum_{k=1}^K w_k \|\text{Attr}_k(\hat{x}) - \text{Attr}_k(x_{\text{src}})\|_2,
\end{equation}
其中 $\text{Attr}_k$ 为第 $k$ 个属性估计器,$w_k$ 为属性权重。多样性约束防止批内样本坍缩到相同生成结果:
\begin{equation}\label{eq:mia_diversity_loss}
    \mathcal{L}_{\text{diversity}} = -\text{Var}(E_{\text{id}}(\{\hat{x}_1, \ldots, \hat{x}_B\})),
\end{equation}
即最大化批内嵌入向量的方差。采用分层感知权重加强关键特征:
\begin{equation}\label{eq:mia_perc_hierarchical}
\begin{split}
    \mathcal{L}_{\text{perc}} = & w_{\text{shallow}} \|\phi_{\text{early}}(\hat{x}) - \phi_{\text{early}}(x)\|_2 \\
    & + w_{\text{mid}} \|\phi_{\text{mid}}(\hat{x}) - \phi_{\text{mid}}(x)\|_2 \\
    & + w_{\text{deep}} \|\phi_{\text{deep}}(\hat{x}) - \phi_{\text{deep}}(x)\|_2,
\end{split}
\end{equation}
推荐权重比例为 $w_{\text{shallow}} : w_{\text{mid}} : w_{\text{deep}} = 0.2 : 0.5 : 0.3$。综合感知与属性约束:
\begin{equation}\label{eq:mia_perception_combined}
    \mathcal{L}_{\text{perc+attr}} = \mathcal{L}_{\text{perc}} + \beta \mathcal{L}_{\text{attr}} + \gamma \mathcal{L}_{\text{diversity}},
\end{equation}
其中 $\beta \in [0.1, 0.3]$,$\gamma \in [0.05, 0.15]$。

\subsection{正则化与训练策略}
\label{subsec:mia_regularization}

正则化损失确保嵌入质量与模型稳定性。本文对嵌入范数、LoRA参数及嵌入分离施加约束:
\begin{equation}\label{eq:mia_reg_loss}
    \mathcal{L}_{\text{reg}} = \sum_{y=1}^C (\|e_y\|_2 - 1)^2 + \lambda_{\text{lora}} \sum_{\ell} (\|A_{\ell}\|_F^2 + \|B_{\ell}\|_F^2) + \lambda_{\text{sep}} \sum_{y \neq y'} \max(0, m - \|e_y - e_{y'}\|_2),
\end{equation}
其中第一项确保嵌入归一化,第二项防止LoRA过拟合,第三项强制不同类别嵌入保持最小间隔 $m$。

本文采用分阶段训练策略实现稳定优化。阶段1(嵌入预训练,500-1000步):冻结换脸与扩散主干,仅训练标签嵌入与少量LoRA参数,关注分类与身份约束的建立,使用较弱先验约束($\lambda_{\text{prior}}=0.3$)与强身份约束($\lambda_{\text{id}}=1.0$)。阶段2(LoRA微调与保真,1000-2000步):开启更多LoRA层训练,加强先验保护,使用完整损失函数,权重通过任务不确定性框架自动学习。LoRA学习率设为主学习率的10-20\%以保持预训练知识。

损失函数中各权重参数的具体取值范围与最优配置将在第\ref{chap:Results}章的实验部分详细讨论,并通过消融实验验证各组件的有效性。

\section{训练与推理流程}
\label{sec:mia_training}

\subsection{训练流程}

模型反演攻击的训练过程分为两个阶段:阶段1专注于标签条件嵌入层的学习,使其建立类别标签到身份嵌入的初步映射;阶段2联合优化嵌入层与LoRA参数,微调换脸模型以适配目标分类器的决策边界。两阶段策略避免了早期过度破坏预训练知识,确保生成质量与攻击有效性的平衡。

\subsubsection{阶段1:嵌入预训练}

阶段1的目标是学习标签到嵌入的映射,同时保持换脸模型的生成能力。算法~\ref{alg:mia_stage1}详细描述了该阶段的训练流程。

\begin{algorithm}[H]
  \caption{MIA阶段1:标签条件嵌入预训练}
  \label{alg:mia_stage1}
  \begin{algorithmic}[1]
    \REQUIRE 训练集$\mathcal{D}=\{(x_i, y_i)\}$,换脸模型$G_\phi$(冻结),分类器$F_\theta$,嵌入层$\mathcal{E}_\psi$,学习率$\eta_{\text{emb}}$
    \ENSURE 训练后的嵌入层参数$\psi$
    \STATE 初始化嵌入层参数$\psi$,初始化任务不确定性参数$\{\sigma_i\}$
    \FOR{训练步数$t=1$ 到 $T_1$ (500-1000步)}
    \STATE 从$\mathcal{D}$中采样批数据$\{(x_i, y_i)\}_{i=1}^B$
    \STATE 采样源图像批$\{x_{\text{src}}^{(i)}\}_{i=1}^B$(从$\mathcal{D}$随机采样)
    \FOR{批内样本$i=1$ 到 $B$}
    \STATE 通过嵌入层生成身份嵌入:$e_{\text{id}}^{(i)} = \mathcal{E}_\psi(y_i)$
    \STATE 通过换脸模型生成图像:$\hat{x}^{(i)} = G_\phi(e_{\text{id}}^{(i)}, x_{\text{src}}^{(i)})$
    \STATE 计算分类logit:$\ell^{(i)} = F_\theta(\hat{x}^{(i)})$
    \STATE 计算身份嵌入:$e_{\text{gen}}^{(i)} = E_{\text{id}}(\hat{x}^{(i)})$
    \STATE 计算分类损失:$\mathcal{L}_{\text{cls}}^{(i)}$(式~\ref{eq:mia_topk_loss})
    \STATE 计算身份损失:$\mathcal{L}_{\text{id}}^{(i)}$(式~\ref{eq:mia_identity_loss})
    \STATE 计算正则化损失:$\mathcal{L}_{\text{reg}}^{(i)}$(嵌入归一化)
    \ENDFOR
    \STATE 计算批平均损失:$\mathcal{L} = \frac{1}{B}\sum_{i=1}^B [\mathcal{L}_{\text{cls}}^{(i)} + \lambda_{\text{id}}\mathcal{L}_{\text{id}}^{(i)} + \mathcal{L}_{\text{reg}}^{(i)}]$
    \STATE 其中$\lambda_{\text{id}}=1.0$(强身份约束)
    \STATE 更新嵌入层参数:$\psi \leftarrow \psi - \eta_{\text{emb}} \cdot \nabla_\psi \mathcal{L}$
    \ENDFOR
    \RETURN $\psi$
  \end{algorithmic}
\end{algorithm}

阶段1仅训练嵌入层,换脸模型权重保持冻结。损失函数聚焦于分类器引导与身份一致性,确保生成图像能被目标分类器识别为正确类别,且包含目标身份特征。

\subsubsection{阶段2:LoRA微调与保真优化}

阶段2在阶段1的基础上,开启LoRA参数训练,微调换脸模型以增强对目标分类器的适配能力。同时引入扩散先验损失与感知损失,保护生成质量不被过度破坏。算法~\ref{alg:mia_stage2}给出了完整的训练流程。

\begin{algorithm}[H]
  \caption{MIA阶段2:LoRA微调与多目标优化}
  \label{alg:mia_stage2}
  \begin{algorithmic}[1]
    \REQUIRE 阶段1训练的嵌入层$\mathcal{E}_\psi$,换脸模型$G_\phi$及其LoRA参数$\{\Delta W_\ell\}$,分类器$F_\theta$,学习率$\eta_{\text{emb}}, \eta_{\text{lora}}$
    \ENSURE 微调后的嵌入层$\psi$与LoRA参数$\{\Delta W_\ell\}$
    \STATE 初始化LoRA参数$\{A_\ell, B_\ell\}$,初始化任务不确定性参数$\{\sigma_i\}$
    \FOR{训练步数$t=1$ 到 $T_2$ (1000-2000步)}
    \STATE 从$\mathcal{D}$中采样批数据$\{(x_i, y_i)\}_{i=1}^B$与源图像$\{x_{\text{src}}^{(i)}\}$
    \FOR{批内样本$i=1$ 到 $B$}
    \STATE 生成身份嵌入:$e_{\text{id}}^{(i)} = \mathcal{E}_\psi(y_i)$
    \STATE 采样扩散时间步:$t^{(i)} \sim \mathcal{U}(0, T)$
    \STATE 编码源图像到潜在空间:$z_0^{(i)} = \text{VAE}_{\text{enc}}(x_{\text{src}}^{(i)})$
    \STATE 添加噪声:$z_t^{(i)} = \sqrt{\bar{\alpha}_{t^{(i)}}} z_0^{(i)} + \sqrt{1-\bar{\alpha}_{t^{(i)}}} \epsilon^{(i)}$,$\epsilon^{(i)} \sim \mathcal{N}(0,I)$
    \STATE 通过微调的去噪网络预测噪声:$\hat{\epsilon}^{(i)} = \epsilon_{\theta+\Delta}(z_t^{(i)}, t^{(i)}, e_{\text{id}}^{(i)})$
    \STATE 计算扩散先验损失:$\mathcal{L}_{\text{prior}}^{(i)} = w(t^{(i)}) \cdot (1 - \cos(\epsilon^{(i)}, \hat{\epsilon}^{(i)}))$
    \STATE 生成图像:$\hat{x}^{(i)} = G_{\phi+\Delta}(e_{\text{id}}^{(i)}, x_{\text{src}}^{(i)})$(完整采样)
    \STATE 计算分类损失$\mathcal{L}_{\text{cls}}^{(i)}$、身份损失$\mathcal{L}_{\text{id}}^{(i)}$、感知损失$\mathcal{L}_{\text{perc}}^{(i)}$、正则化$\mathcal{L}_{\text{reg}}^{(i)}$
    \ENDFOR
    \STATE 通过任务不确定性框架计算总损失:
    \STATE \quad $\mathcal{L}_{\text{total}} = \sum_{i\in\{\text{prior,cls,id,perc,reg}\}} \left(\frac{1}{2\sigma_i^2}\bar{\mathcal{L}}_i + \frac{1}{2}\log\sigma_i^2\right)$
    \STATE 其中$\bar{\mathcal{L}}_i = \frac{1}{B}\sum_{j=1}^B \mathcal{L}_i^{(j)}$为批平均
    \STATE 更新参数:
    \STATE \quad $\psi \leftarrow \psi - \eta_{\text{emb}} \cdot \nabla_\psi \mathcal{L}_{\text{total}}$
    \STATE \quad $\{A_\ell, B_\ell\} \leftarrow \{A_\ell, B_\ell\} - \eta_{\text{lora}} \cdot \nabla_{\{A_\ell, B_\ell\}} \mathcal{L}_{\text{total}}$
    \STATE \quad $\{\sigma_i\} \leftarrow \{\sigma_i\} - 0.1\eta_{\text{emb}} \cdot \nabla_{\{\sigma_i\}} \mathcal{L}_{\text{total}}$
    \ENDFOR
    \RETURN $\psi$, $\{A_\ell, B_\ell\}$
  \end{algorithmic}
\end{algorithm}

阶段2采用完整的多目标损失函数,通过任务不确定性加权框架自动平衡各损失项。LoRA学习率设为主学习率的10-20\%($\eta_{\text{lora}} = 0.1\sim 0.2 \cdot \eta_{\text{emb}}$),不确定性参数采用更小学习率($0.1\eta_{\text{emb}}$)以避免早期过度衰减。训练过程中定期在验证集上评估攻击成功率与FID指标,根据性能选择最佳检查点。

\subsection{推理策略}

推理阶段根据目标类别$y$生成嵌入$e_{\text{id}} = \mathcal{E}_\psi(y)$,结合源图像$x_{\text{src}}$通过换脸模型生成攻击样本$\hat{x} = G_{\phi+\Delta}(e_{\text{id}}, x_{\text{src}})$。为提升攻击成功率,可采用多源图像采样策略:从包含$K$个候选源图像的集合$\mathcal{X}_{\text{cand}}$中,为每个源图像生成攻击样本并计算目标类别置信度,选择使$F_\theta(\hat{x})_{y}$最大的样本作为最终输出。该策略通过候选集的多样性(不同姿态、表情、光照)增强攻击鲁棒性。实验中设置$K=5$,使用DDIM 50步采样。训练完成后可将LoRA权重合并到基础模型($W' = W + \frac{\alpha}{r}BA$)以消除推理开销。完整超参数配置详见第\ref{chap:Results}章§5.2.4表~\ref{tab:mia_hyperparams_detail},其中时间自适应扩散权重$w(t)$、任务不确定性初始化$\sigma_i^2$、LoRA秩$r$与缩放因子$\alpha$的最优设置均通过消融实验确定。

\section{本章小结}
\label{sec:mia_summary}

本章提出了基于换脸先验与标签条件嵌入的模型反演攻击方法,用于评估人脸分类模型的隐私泄露风险。该方法利用预训练换脸模型的生成能力,通过标签条件嵌入层将类别标签映射为身份嵌入,并采用LoRA技术进行参数高效微调,实现高质量的模型反演攻击。实验结果表明该方法在攻击成功率与图像质量方面均达到先进水平,为生物特征识别系统的安全性评估提供了有效工具。
% Local Variables:
% TeX-master: "../main"
% TeX-engine: xetex
% End:
