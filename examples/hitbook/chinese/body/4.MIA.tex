% !Mode:: "TeX:UTF-8"

\chapter{基于换脸先验的模型反演攻击方法}[Model Inversion Attack Based on Face Swapping Prior]\label{chap:MIA}

\section{引言}

模型反演攻击旨在从训练好的分类模型中重建其训练数据的敏感特征,对生物特征识别系统的隐私安全构成严重威胁。传统基于梯度优化的方法直接在像素空间进行优化,面临生成质量低下、缺乏语义约束、优化效率低等问题。近年来,生成式先验模型的引入为模型反演攻击提供了新的技术路径,其中换脸模型因其独特的身份控制能力而成为理想选择。

然而,换脸模型通常需要目标图像作为身份输入,而在模型反演攻击场景中,攻击者仅拥有类别标签而无法获取目标身份的真实图像。本章针对这一核心挑战,提出了一种基于换脸先验与标签条件嵌入的模型反演攻击方法。该方法通过设计标签条件嵌入层将类别标签映射为身份表示,并采用低秩适配技术对换脸模型进行参数高效微调,在保持生成质量的同时实现对目标分类器的高效适配。

本章首先对模型反演攻击问题进行形式化定义,明确威胁模型与评估标准;随后详细阐述基于换脸先验的方法架构,包括标签条件嵌入层设计、LoRA微调策略以及多目标优化损失函数框架;最后给出完整的训练与推理流程,为方法的实现与复现提供系统化指导。

\section{形式化问题定义}
\label{sec:mia_problem}

根据第\ref{sec:thesis_structure}节建立的攻击任务框架,简要回顾模型反演攻击的形式化定义。设目标分类器为 $F_\theta:\mathcal{X}\to\mathbb{R}^C$,其中 $\mathcal{X}$ 为输入空间(如 $\mathbb{R}^{H\times W\times 3}$ 表示RGB图像),$C$ 为类别数量,$\theta$ 为模型参数。给定目标类别标签 $y_{\text{target}}\in\{1,2,\ldots,C\}$,攻击者的目标是生成图像集合 $\{\hat{x}_1, \hat{x}_2, \ldots, \hat{x}_K\}$,使其同时满足分类器认同度、感知真实性、身份相关性与多样性四个条件。完整的形式化定义与约束优化问题详见第\ref{sec:thesis_structure}节。

本章主要关注白盒威胁模型,即攻击者拥有目标分类器 $F_\theta$ 的完整信息,包括模型架构、参数以及完整的梯度信息 $\nabla_x F_\theta(x)$。这一威胁模型为评估模型的最大隐私泄露风险提供了上界,对应于模型开源、内部泄露或逆向工程成功的场景。在白盒设定下,攻击者可以直接利用梯度信息进行端到端优化:将分类器损失 $\mathcal{L}_{\text{cls}}(F_\theta(\hat{x}), y_{\text{target}})$ 融入生成模型的训练目标,通过反向传播更新生成器参数,使生成轨迹持续逼近目标类别的高置信区域。相比黑盒或灰盒场景(仅能查询输出概率),白盒访问权限提供了更强的优化信号,从而能够实现更高质量的重建结果。这一设定为深入研究方法设计与性能边界提供了理想的实验环境。

\section{基于换脸先验的模型反演架构}
\label{sec:mia_architecture}

\subsection{换脸先验模型的选择与分析}

换脸模型作为本章方法的核心生成先验,其选择对最终攻击性能有决定性影响。理想的换脸模型应满足:(1)高质量的身份迁移能力;(2)显式的身份-属性解耦机制;(3)高感知质量与真实感;(4)支持端到端梯度反向传播。

基于上述要求,本章选择REFace~\cite{sanoojan2024reface}作为换脸先验模型。REFace是一种统一的基于扩散模型的换脸方法,将换脸重构为训练时自监督修复任务,通过训练时多步DDIM采样增强与CLIP特征解耦机制实现高保真的身份迁移。该方法在CelebA-HQ数据集上达到98.8\%的身份检索准确率与6.09的FID分数,显著优于其他基于扩散的换脸方法。

\textbf{与REFace原始方法的核心区别:}REFace在推理阶段需要提供真实的目标图像$x_{\text{target}}$,通过预训练的ArcFace编码器提取其身份嵌入$e_{\text{id}}=E_{\text{id}}(x_{\text{target}})$作为身份控制信号。然而在模型反演攻击场景中,攻击者仅拥有目标类别标签$y_{\text{target}}$而无法获取真实目标图像,这是本方法面临的核心挑战。为解决该问题,本章设计标签条件嵌入层$\mathcal{E}_\psi$直接将类别标签映射为身份嵌入$e_{\text{id}}=\mathcal{E}_\psi(y_{\text{target}})$,无缝替代REFace的ArcFace编码器,从而实现"无目标图像"的模型反演攻击。该设计保留了REFace强大的生成先验,同时通过LoRA微调使模型适配新的身份嵌入分布,在保持生成质量的同时实现对目标分类器的高效攻击。

在模型反演攻击场景中,REFace的核心优势在于:(1)使用预训练ArcFace模型提取512维身份嵌入$e_{\text{id}} = E_{\text{id}}(x) \in \mathbb{R}^{512}$,通过参考注意力机制$\text{RefAttn}(Q, K_{\text{ref}}, V_{\text{ref}}) = \text{Softmax}(QK_{\text{ref}}^T/\sqrt{d}) V_{\text{ref}}$注入U-Net去噪网络,该接口可被本章的标签条件嵌入层无缝替代,无需真实目标图像即可实现身份控制;(2)基于潜在扩散模型在VAE潜在空间进行条件生成,支持端到端梯度反向传播,使分类器损失能够直接指导生成过程;(3)在大规模人脸数据集上预训练的模型已学习到强大的生成先验,为高质量的模型反演攻击提供了坚实基础。

\subsection{总体架构与前向传播}

如图~\ref{fig:mia_architecture}所示,该架构由三个核心模块组成:标签条件嵌入层$\mathcal{E}_\psi$、预训练换脸模型$G_\phi$与LoRA微调模块、目标分类器$F_\theta$。

\begin{figure}[htbp]
    \centering
    \includegraphics[width=0.9\textwidth]{images/train_mia.drawio.pdf}
    \caption{基于换脸先验的模型反演攻击架构示意图。左侧为标签条件嵌入层,将类别标签映射为身份嵌入;中间为换脸模型,接受身份嵌入与源图像生成目标图像;右侧为目标分类器,提供分类损失用于微调。}
    \label{fig:mia_architecture}
\end{figure}

前向传播流程如下:给定目标类别标签$y_{\text{target}}\in\{1,2,\ldots,C\}$,标签条件嵌入层将其映射为身份嵌入向量$e_{\text{id}} = \mathcal{E}_{\psi}(y_{\text{target}})\in\mathbb{R}^{d_e}$;预训练换脸模型接受$e_{\text{id}}$与源图像$x_{\text{src}}$生成目标图像$\hat{x} = G_{\phi}(e_{\text{id}}, x_{\text{src}})$;生成图像输入目标分类器获得置信度$p_{y_{\text{target}}} = F_\theta(\hat{x})_{y_{\text{target}}}$,用于计算分类器引导损失并通过反向传播更新参数。

\subsection{标签条件嵌入层设计}

模型反演攻击中,攻击者仅拥有目标类别标签,需设计标签条件嵌入层实现从离散标签到连续身份嵌入的映射。采用多层感知机将类别标签的one-hot编码映射为身份嵌入向量:
\begin{equation}\label{eq:mia_mlp_emb}
    e_{\text{id}} = \text{MLP}_{\psi}(\text{OneHot}(y_{\text{target}})) \in \mathbb{R}^{d_e},
\end{equation}
其中MLP包含2-3个隐藏层,配备ReLU激活函数与LayerNorm。为确保与换脸模型的ID注入模块兼容,MLP输出维度$d_e$设置与ArcFace嵌入维度一致(通常$d_e=512$),并对嵌入向量进行$L_2$归一化使其位于单位超球面上:
\begin{equation}\label{eq:mia_emb_norm}
    e_{\text{id}} \leftarrow \frac{e_{\text{id}}}{\|e_{\text{id}}\|_2}.
\end{equation}

\subsection{LoRA参数高效微调策略}

直接对换脸模型进行全参数微调面临参数量巨大、过拟合风险高等挑战。LoRA技术通过在冻结预训练权重的前提下引入低秩可训练增量,仅需训练原模型1\%-5\%的参数量,有效降低过拟合风险并保留预训练的生成先验。

根据换脸模型的架构特点与身份控制机制,LoRA应用于以下三类关键模块。这些模块的选择基于对身份信息流动路径的分析:参考注意力层直接处理身份嵌入的编码,U-Net注意力层负责特征的全局聚合与身份信息的空间传播,残差块卷积层调整通道间的特征映射——这些模块对身份控制最为敏感,微调它们可以最大化地适配新的身份嵌入分布,同时最小化对生成质量的负面影响。

(1)参考注意力投影矩阵:对第$\ell$层参考注意力的键值投影$W_K^{\ell}, W_V^{\ell}$应用低秩分解:
\begin{equation}\label{eq:mia_lora_ref_proj}
\begin{aligned}
    W_K^{\ell\prime} &= W_K^{\ell} + \frac{\alpha}{r} B_K^{\ell} A_K^{\ell}, \quad
    W_V^{\ell\prime} = W_V^{\ell} + \frac{\alpha}{r} B_V^{\ell} A_V^{\ell},
\end{aligned}
\end{equation}
其中$A_K^{\ell}, A_V^{\ell}\in\mathbb{R}^{r\times 512}$,$B_K^{\ell}, B_V^{\ell}\in\mathbb{R}^{d_h\times r}$($d_h$为注意力头维度,通常为64或128),$r$为LoRA秩(通常取4-16),$\alpha$为缩放因子(通常与$r$相等)。这些矩阵直接控制身份嵌入编码,是身份信息注入的核心机制。

(2)U-Net注意力层:对去噪U-Net的自注意力和交叉注意力的投影矩阵应用LoRA:
\begin{equation}\label{eq:mia_lora_attn}
\begin{aligned}
    W_Q' &= W_Q + \frac{\alpha}{r} B_Q A_Q, \quad
    W_K' = W_K + \frac{\alpha}{r} B_K A_K, \\
    W_V' &= W_V + \frac{\alpha}{r} B_V A_V, \quad
    W_O' = W_O + \frac{\alpha}{r} B_O A_O,
\end{aligned}
\end{equation}
其中$W_Q, W_K, W_V, W_O$分别为Query、Key、Value与输出投影矩阵。注意力机制负责特征全局聚合,微调这些投影可优先生成分类器敏感的身份特征。

(3)残差块卷积层:对U-Net残差块中的$1\times1$点卷积应用LoRA。点卷积权重$W\in\mathbb{R}^{C_{\text{out}}\times C_{\text{in}}\times 1\times 1}$重塑为二维矩阵后分解:
\begin{equation}\label{eq:mia_lora_conv}
    W' = W + \frac{\alpha}{r} \text{reshape}(B A, [C_{\text{out}}, C_{\text{in}}, 1, 1]),
\end{equation}
其中$A\in\mathbb{R}^{r\times C_{\text{in}}}, B\in\mathbb{R}^{C_{\text{out}}\times r}$。

\section{多目标优化损失函数设计}
\label{sec:mia_loss}

本节系统设计面向模型反演攻击的多目标优化损失函数,平衡扩散先验保真度、分类器攻击有效性、身份一致性、生成质量与模型正则化五个优化目标。采用任务不确定性加权框架进行自动权重调整。

\subsection{总体损失架构}

本文采用任务不确定性加权框架统一各损失项,避免手动权重调优的复杂性:
\begin{equation}\label{eq:mia_total_loss}
    \mathcal{L}_{\text{total}} = \sum_{i \in \{\text{prior, cls, id, perc, reg}\}} \left( \frac{1}{2\sigma_i^2} \mathcal{L}_i + \frac{1}{2}\log\sigma_i^2 \right),
\end{equation}
其中 $\sigma_i$ 为第 $i$ 个任务的可学习不确定性参数,与网络参数联合优化。该框架自动将权重分配偏向于较难的任务($\sigma_i$ 较小),确保各目标均衡发展。训练时 $\sigma_i$ 采用较小学习率以避免早期过度衰减。

\subsection{扩散先验损失}

扩散先验损失确保微调后的模型保持预训练扩散模型的生成能力,避免在攻击优化过程中偏离自然人脸流形。该损失基于扩散模型的去噪目标,训练去噪网络$\epsilon_\theta$预测在扩散前向过程中添加的噪声。

给定源图像的潜在表示$z_0 = \text{VAE}_{\text{enc}}(x_{\text{src}})$,扩散前向过程在时间步$t$添加高斯噪声得到$z_t = \sqrt{\bar{\alpha}_t} z_0 + \sqrt{1-\bar{\alpha}_t} \epsilon$,其中$\epsilon \sim \mathcal{N}(0, I)$为标准高斯噪声,$\bar{\alpha}_t$为噪声调度系数。去噪网络以$z_t$、时间步$t$与身份嵌入$e_{\text{id}}$为条件,预测噪声$\hat{\epsilon} = \epsilon_\theta(z_t, t, e_{\text{id}})$。

本文采用余弦相似度度量预测噪声与真实噪声的一致性,并引入时间步加权机制:
\begin{equation}\label{eq:mia_prior_loss}
    \mathcal{L}_{\text{prior}} = \mathbb{E}_{z_0, \epsilon\sim\mathcal{N}(0,I), t} \left[ w(t) \cdot \left(1 - \frac{\langle \epsilon, \epsilon_\theta(z_t, t, e_{\text{id}}) \rangle}{\|\epsilon\|_2 \|\epsilon_\theta(z_t, t, e_{\text{id}})\|_2} \right) \right],
\end{equation}
其中余弦相似度$(1 - \cos(\epsilon, \hat{\epsilon}))$比传统的$L_2$损失更关注方向一致性。时间步权重$w(t)$根据扩散阶段动态调整,采用单调递减函数$w(t) = 1 + \beta \cdot (1 - t/T)$,其中$\beta$为调节系数(推荐值1.0-2.0),$T$为最大时间步(通常1000)。该设计在低噪声阶段($t$较小,接近真实图像)赋予更高权重以强化细节塑造,在高噪声阶段($t$较大)使用较低权重以关注全局结构。该损失确保LoRA微调不会破坏预训练模型的生成先验,保持生成图像的自然性与真实感。

\subsection{分类器引导损失}

分类器引导损失驱动生成图像被目标分类器识别为目标类别,是模型反演攻击的核心驱动力。传统交叉熵损失$-\log p_y$仅最大化目标类别概率,易导致生成图像在决策边界附近徘徊。本文采用top-k max-margin损失,不仅提升目标类别置信度,还显式增大其与最易混淆类别间的裕度。

设目标分类器输出logits为$\{\ell_1, \ldots, \ell_C\}$,目标类别为$y$。top-k max-margin损失定义为:
\begin{equation}\label{eq:mia_topk_loss}
    \mathcal{L}_{\text{cls}} = -\ell_y + \frac{1}{k_{\text{adapt}}} \sum_{j \in \text{top-}k_{\text{adapt}}(J \setminus \{y\})} \ell_j,
\end{equation}
其中$J = \{1, \ldots, C\}$为所有类别集合,$\text{top-}k_{\text{adapt}}(J \setminus \{y\})$选取除目标类别外logit最高的$k_{\text{adapt}}$个类别。该损失同时最小化目标类别logit$\ell_y$的负值(即最大化$\ell_y$)和最大化混淆类别logit的平均值,从而在高维空间中推动生成图像远离决策边界。参数$k_{\text{adapt}} = \max(2, \min(5, \lfloor C / 20 \rfloor))$根据类别总数$C$自适应调整:类别数少时关注更多混淆类别,类别数多时聚焦于最具竞争性的类别。

为进一步稳定优化过程,引入基于可学习特征中心的正则化约束(p-reg)。设$p_x = F_\theta^{\text{feat}}(\hat{x}) \in \mathbb{R}^{d_{\text{feat}}}$为分类器倒数第二层的特征表示($d_{\text{feat}}$为特征维度,对于ResNet等常见架构通常为512或2048),$c_y \in \mathbb{R}^{d_{\text{feat}}}$为目标类别$y$的可学习特征中心。p-reg损失定义为:
\begin{equation}\label{eq:mia_preg_loss}
    \mathcal{L}_{\text{p-reg}} = \left\|p_x - c_y\right\|_2^2,
\end{equation}
该损失显式约束生成图像的特征表示接近目标类别的原型中心,避免优化过程中的振荡。特征中心$c_y$通过动量更新自动学习:
\begin{equation}
    c_y \leftarrow (1-\rho) c_y + \rho \cdot E_{\text{id}}(\hat{x}),
\end{equation}
其中$E_{\text{id}}$为身份编码器(如ArcFace),$\rho$为动量系数。该更新机制使特征中心逐渐聚合目标类别的身份特征,为后续优化提供稳定的吸引子。

\subsection{身份一致性损失}

身份一致性损失确保生成图像的身份特征与标签条件嵌入$e_{\text{id}}$一致,建立类别标签与视觉身份特征的稳定映射关系。本文采用对比学习框架,利用人脸识别系统(如ArcFace)在单位超球面上的几何性质,显式增强目标身份与负样本身份的角度分离。

\textbf{损失设计的核心思想:}给定目标类别$y_{\text{target}}$,标签条件嵌入层生成身份嵌入$e_{\text{id}} = \mathcal{E}_\psi(y_{\text{target}})$,该嵌入代表了"我们希望生成图像具备的身份特征"。同时,使用预训练的ArcFace编码器$E_{\text{id}}$提取生成图像$\hat{x}$的实际身份特征$e_{\text{gen}} = E_{\text{id}}(\hat{x})$。身份一致性损失通过对比学习确保$e_{\text{gen}}$接近目标身份嵌入$e_{\text{id}}$,同时远离其他类别的嵌入。\textbf{注意:这里不涉及任何真实的隐私图像}——$e_{\text{id}}$是由MLP从类别标签直接生成的,$E_{\text{id}}$仅用于评估生成图像的身份特征是否符合预期。

身份一致性损失采用对比学习形式:
\begin{equation}\label{eq:mia_identity_loss}
    \mathcal{L}_{\text{id}} = \mathbb{E}\left[\max\left(0, m + \frac{\langle E_{\text{id}}(\hat{x}), e_{\text{neg}} \rangle}{\|E_{\text{id}}(\hat{x})\|_2 \|e_{\text{neg}}\|_2} - \frac{\langle E_{\text{id}}(\hat{x}), e_{\text{id}} \rangle}{\|E_{\text{id}}(\hat{x})\|_2 \|e_{\text{id}}\|_2}\right)\right],
\end{equation}
其中:
\begin{itemize}
    \item $e_{\text{gen}} = E_{\text{id}}(\hat{x})$:预训练ArcFace编码器提取的\textbf{生成图像}的身份特征(512维归一化向量)
    \item $e_{\text{id}} = \mathcal{E}_\psi(y_{\text{target}})$:标签条件嵌入层生成的\textbf{目标身份嵌入}(无需真实图像)
    \item $e_{\text{neg}} = \mathcal{E}_\psi(y_{\text{neg}})$:从内存库$\mathcal{M}$中采样的\textbf{负样本类别嵌入}($y_{\text{neg}} \neq y_{\text{target}}$)
    \item $m$:角度裕度超参数(推荐值0.3-0.5),控制正负样本间的最小角度间隔
\end{itemize}

\textbf{内存库构建:}内存库$\mathcal{M}=\{e_1, e_2, \ldots, e_C\}$存储所有$C$个类别通过标签条件嵌入层生成的嵌入向量,即$e_i = \mathcal{E}_\psi(i), \forall i \in \{1,\ldots,C\}$。在训练开始时随机初始化嵌入层参数后构建初始内存库,在阶段1训练过程中每50步更新一次(使用当前嵌入层的最新输出)。负样本$e_{\text{neg}}$从内存库中均匀采样,确保对比学习覆盖所有类别。

该损失通过hinge loss形式,仅在正样本相似度与负样本相似度之差小于裕度$m$时施加惩罚,驱动优化过程同时满足:(1)最大化$\cos(e_{\text{gen}}, e_{\text{id}})$,使生成图像的身份特征接近目标身份;(2)最小化$\cos(e_{\text{gen}}, e_{\text{neg}})$,使生成图像远离其他类别身份。该对比机制确保不同类别的身份嵌入在超球面上具有明确的角度分离,防止类别间的身份混淆。

在总体损失架构(式~\ref{eq:mia_total_loss})中,分类器引导损失$\mathcal{L}_{\text{cls}}$、p-reg损失$\mathcal{L}_{\text{p-reg}}$与身份一致性损失$\mathcal{L}_{\text{id}}$作为独立的损失项分别加权。实际实现中,可将$\mathcal{L}_{\text{cls}}$与$\mathcal{L}_{\text{p-reg}}$组合为单一项$\mathcal{L}_{\text{cls+p-reg}} = \mathcal{L}_{\text{cls}} + \alpha \mathcal{L}_{\text{p-reg}}$,其中$\alpha$为p-reg的相对权重(推荐值0.1-0.5)。

\subsection{感知质量与属性保持损失}

感知质量损失确保生成图像的视觉真实性,防止优化过程产生不自然的伪影或失真。本文采用分层感知损失(LPIPS),基于预训练深度网络的多层特征度量生成图像与源图像的感知距离。

给定预训练特征提取器$\phi$(如VGG或AlexNet),从浅层到深层提取特征$\{\phi_{\text{early}}, \phi_{\text{mid}}, \phi_{\text{deep}}\}$。浅层特征捕获纹理与细节,中层特征编码局部结构,深层特征表示语义内容。分层感知损失定义为:
\begin{equation}\label{eq:mia_perc_hierarchical}
\begin{split}
    \mathcal{L}_{\text{perc}} = & w_{\text{shallow}} \|\phi_{\text{early}}(\hat{x}) - \phi_{\text{early}}(x_{\text{src}})\|_2 \\
    & + w_{\text{mid}} \|\phi_{\text{mid}}(\hat{x}) - \phi_{\text{mid}}(x_{\text{src}})\|_2 \\
    & + w_{\text{deep}} \|\phi_{\text{deep}}(\hat{x}) - \phi_{\text{deep}}(x_{\text{src}})\|_2,
\end{split}
\end{equation}
其中$x_{\text{src}}$为源图像,$w_{\text{shallow}}, w_{\text{mid}}, w_{\text{deep}}$为不同层级的权重系数。该损失确保生成图像保持源图像的视觉风格与结构,同时通过深度特征比对避免像素级过拟合。

属性保持损失显式约束源图像的非身份属性(姿态、表情、光照等)正确迁移至生成图像。换脸任务要求仅改变身份特征,保持其他属性不变。本文采用预训练属性估计器提取属性向量并计算$L_2$距离:
\begin{equation}\label{eq:mia_attr_loss}
    \mathcal{L}_{\text{attr}} = \sum_{k=1}^K w_k \|\text{Attr}_k(\hat{x}) - \text{Attr}_k(x_{\text{src}})\|_2,
\end{equation}
其中$\text{Attr}_k$为第$k$个属性估计器(如姿态估计器HopeNet、表情分类器、光照估计器等),$K$为属性类型总数,$w_k$为属性权重。该损失确保生成图像在改变身份的同时保持源图像的场景属性,提升换脸结果的自然性。

多样性损失防止批内样本坍缩到相同的生成结果,确保模型为同一类别生成多样化的样本。本文通过最大化批内身份嵌入的方差实现:
\begin{equation}\label{eq:mia_diversity_loss}
    \mathcal{L}_{\text{diversity}} = -\text{Var}(E_{\text{id}}(\{\hat{x}_1, \ldots, \hat{x}_B\})),
\end{equation}
其中$B$为批大小,$\text{Var}(\cdot)$计算嵌入向量的方差。该损失鼓励批内不同源图像生成的结果在身份空间中保持多样性,避免模式坍缩。

综合上述三项约束,形成感知与属性保持的联合损失:
\begin{equation}\label{eq:mia_perception_combined}
    \mathcal{L}_{\text{perc+attr}} = \mathcal{L}_{\text{perc}} + \beta \mathcal{L}_{\text{attr}} + \gamma \mathcal{L}_{\text{diversity}},
\end{equation}
其中$\beta$和$\gamma$为权重系数,平衡感知质量、属性一致性与样本多样性三个目标。

\subsection{正则化损失与训练策略}
\label{subsec:mia_regularization}

正则化损失确保模型训练的稳定性与泛化能力,防止过拟合并维护嵌入空间的良好几何性质。本文设计三项正则化约束,分别针对嵌入范数、LoRA参数与类别分离:
\begin{equation}\label{eq:mia_reg_loss}
    \mathcal{L}_{\text{reg}} = \mathcal{L}_{\text{emb-norm}} + \lambda_{\text{lora}} \mathcal{L}_{\text{lora}} + \lambda_{\text{sep}} \mathcal{L}_{\text{sep}},
\end{equation}
其中各项定义如下。

\textbf{嵌入归一化约束:}标签条件嵌入层输出的身份嵌入$e_{\text{id}}^{(y)} = \mathcal{E}_\psi(y)$应与预训练身份编码器(ArcFace)的输出保持一致的范数,以确保与换脸模型的参考注意力模块兼容。该约束定义为:
\begin{equation}
    \mathcal{L}_{\text{emb-norm}} = \sum_{y=1}^C (\|e_{\text{id}}^{(y)}\|_2 - 1)^2,
\end{equation}
即强制所有类别的嵌入向量位于单位超球面上。该约束确保嵌入空间的几何一致性,避免范数差异导致的不均衡优化。

\textbf{LoRA权重正则化:}LoRA参数$\{A_{\ell}, B_{\ell}\}$通过Frobenius范数约束,防止权重过大导致的过拟合:
\begin{equation}
    \mathcal{L}_{\text{lora}} = \sum_{\ell} (\|A_{\ell}\|_F^2 + \|B_{\ell}\|_F^2),
\end{equation}
其中$\ell$遍历所有应用LoRA的层,$\|\cdot\|_F$为Frobenius范数。该正则化保持LoRA增量的小幅修正,确保微调不会显著偏离预训练权重,从而保留原始生成先验。

\textbf{类别分离约束:}不同类别的身份嵌入应在嵌入空间中保持足够的间隔,避免类别混淆。该约束通过hinge loss实现:
\begin{equation}
    \mathcal{L}_{\text{sep}} = \sum_{y \neq y'} \max(0, m_{\text{sep}} - \|e_{\text{id}}^{(y)} - e_{\text{id}}^{(y')}\|_2),
\end{equation}
其中$m_{\text{sep}}$为最小间隔超参数(推荐值0.5-1.0,注意与身份对比损失中的角度裕度$m$区分)。仅当两个类别嵌入间的欧氏距离小于$m_{\text{sep}}$时施加惩罚,推动不同类别在嵌入空间中远离。该约束确保模型学习到判别性的类别表示。

\subsubsection{渐进式训练策略}
\label{subsubsec:mia_progressive_training}

标签条件嵌入层替换原始身份编码器后,模型面临从"图像条件生成"到"标签条件生成"的模态转换挑战。直接使用随机初始化的标签嵌入会导致生成图像呈现随机噪声,训练极不稳定。借鉴课程学习(Curriculum Learning)与知识蒸馏中的教师退火(Teacher Annealing)思想,本文提出\textbf{渐进式微调策略}:在训练初期同时提供真实图像与类别标签,通过插值混合真实图像的身份嵌入与标签嵌入层输出,随训练进程逐步增大标签嵌入的权重,最终平滑过渡至纯标签条件生成。该策略将困难的模态转换分解为易于优化的渐进式任务,显著提升训练稳定性与收敛速度。

\textbf{阶段0:图像条件预热}(Warm-up with Image Conditioning)

该阶段利用公开人脸数据集(如CelebA、FFHQ)的真实图像,让模型先学会"如何利用身份嵌入进行换脸",为后续的标签嵌入学习奠定基础。

\textit{身份嵌入来源:}完全使用预训练ArcFace编码器从真实图像提取身份嵌入:
\begin{equation}
    e_{\text{id}} = E_{\text{id}}(x_{\text{real}}),
\end{equation}
其中$x_{\text{real}}$为训练集中与目标类别$y$对应的真实图像(可从公开数据集随机采样)。此时标签条件嵌入层$\mathcal{E}_\psi$尚未参与前向传播。

\textit{优化目标:}冻结换脸模型主干与扩散U-Net,仅训练LoRA参数$\{A_\ell, B_\ell\}$:
\begin{equation}\label{eq:mia_stage0_loss}
    \mathcal{L}_{\text{stage0}} = \mathcal{L}_{\text{prior}} + \mathcal{L}_{\text{perc}}.
\end{equation}
该阶段不涉及标签嵌入层,损失函数聚焦于扩散先验保真度与感知质量,使LoRA适配层学习如何在保持生成质量的前提下利用外部身份嵌入进行换脸。

\textit{训练步数:}通常200-500步,直至生成图像FID稳定在较低水平(如FID<50)。该阶段训练速度快,因仅优化少量LoRA参数且无需标签嵌入层的梯度计算。

\textbf{阶段1:混合条件过渡}(Hybrid Transition with Annealing)

该阶段是核心创新,通过插值混合真实图像嵌入与标签嵌入,实现从"图像条件"到"标签条件"的平滑迁移。

\textit{身份嵌入来源:}采用时间依赖的线性插值策略:
\begin{equation}\label{eq:mia_hybrid_embedding}
    e_{\text{id}}(t) = (1-\lambda(t)) \cdot E_{\text{id}}(x_{\text{real}}) + \lambda(t) \cdot \mathcal{E}_\psi(y),
\end{equation}
其中$t$为当前训练步数,$\lambda(t)$为退火系数,控制标签嵌入的权重。本文采用余弦退火调度:
\begin{equation}\label{eq:mia_annealing_schedule}
    \lambda(t) = \frac{1}{2}\left(1 - \cos\left(\frac{\pi \cdot \min(t, T_{\text{anneal}})}{T_{\text{anneal}}}\right)\right),
\end{equation}
其中$T_{\text{anneal}}$为退火周期(推荐值500-1000步)。余弦调度在初期缓慢增长(给予模型充分适应时间),中期快速过渡(避免长期停留在混合状态),后期平滑收敛(确保稳定进入纯标签模式)。相比线性退火$\lambda(t)=t/T_{\text{anneal}}$,余弦调度提供更平稳的过渡曲线。

\textit{优化目标:}解冻标签条件嵌入层$\mathcal{E}_\psi$,联合优化嵌入层与LoRA参数:
\begin{equation}\label{eq:mia_stage1_loss}
    \mathcal{L}_{\text{stage1}} = \mathcal{L}_{\text{prior}} + \mathcal{L}_{\text{perc}} + \mathcal{L}_{\text{emb-norm}} + \beta \mathcal{L}_{\text{align}},
\end{equation}
其中引入\textbf{身份对齐损失}(Identity Alignment Loss),确保标签嵌入逐步接近真实图像的身份特征:
\begin{equation}\label{eq:mia_align_loss}
    \mathcal{L}_{\text{align}} = \|E_{\text{id}}(x_{\text{real}}) - \mathcal{E}_\psi(y)\|_2^2.
\end{equation}
该损失为标签嵌入层提供显式的监督信号,避免其学习到与真实身份空间偏离的表示。权重系数$\beta$随退火进程动态调整:$\beta(t) = \beta_0 \cdot (1 - \lambda(t))$(推荐$\beta_0=1.0$),即在初期强化对齐约束,后期逐步减弱以赋予标签嵌入更大的自由度。

\textit{训练步数与切换判据:}该阶段通常需500-1000步。切换至下一阶段的判据为:(1)退火系数$\lambda(t)$达到1.0;(2)对齐损失$\mathcal{L}_{\text{align}}$收敛至较小值(如$<0.1$);(3)生成图像质量稳定(FID波动$<5$)。实践中以退火完成为主要依据,对齐损失与FID作为辅助验证。

\textbf{阶段2:纯标签条件适配}(Label-Only Adaptation)

该阶段完全移除真实图像依赖,进入最终的分类器攻击优化。

\textit{身份嵌入来源:}完全使用标签条件嵌入层:
\begin{equation}
    e_{\text{id}} = \mathcal{E}_\psi(y).
\end{equation}
此时模型已学会从类别标签生成有效的身份嵌入,无需真实图像辅助。

\textit{优化目标:}解冻U-Net注意力层与残差块卷积的LoRA参数,引入完整损失函数$\mathcal{L}_{\text{total}}$(式~\ref{eq:mia_total_loss}):
\begin{equation}\label{eq:mia_stage2_loss}
    \mathcal{L}_{\text{stage2}} = \mathcal{L}_{\text{total}} = \sum_{i \in \{\text{prior, cls, id, perc, reg}\}} \left( \frac{1}{2\sigma_i^2} \mathcal{L}_i + \frac{1}{2}\log\sigma_i^2 \right).
\end{equation}
该阶段同时优化攻击有效性与生成质量:分类器损失$\mathcal{L}_{\text{cls}}$驱动生成图像被目标分类器识别,身份对比损失$\mathcal{L}_{\text{id}}$建立类别标签与视觉身份的映射,先验与感知损失保持生成质量。各损失权重通过任务不确定性框架自动学习,LoRA参数采用较小学习率(如阶段1的0.1-0.5倍)以避免遗忘已学习的生成能力。

\textit{训练步数:}该阶段通常需1000-2000步,直至攻击成功率与生成质量同时达到预期。相比传统的"从零开始"训练标签嵌入的方案,该阶段收敛更快,因标签嵌入已在阶段1获得良好初始化。

\textbf{渐进式策略的理论优势:}

(1)\textit{稳定性提升:}阶段0提供强身份信号,避免训练初期的随机噪声状态;阶段1的插值机制防止模态转换过程中的梯度爆炸或消失。

(2)\textit{收敛加速:}标签嵌入层在阶段1获得有监督的初始化(通过对齐损失),无需从随机初始化摸索有效的嵌入分布,相比传统方案可节省30\%-50\%的训练步数。

(3)\textit{泛化能力增强:}阶段0-1使用大规模公开数据集(如CelebA 20万+图像),标签嵌入层学习到的身份表示受益于丰富的人脸变化模式,而非仅针对目标分类器过拟合。

(4)\textit{过拟合风险降低:}混合阶段的真实图像嵌入充当正则化项,约束标签嵌入不能偏离真实身份空间太远,防止学习到病态的嵌入分布。

该渐进式策略将模型反演攻击中的模态转换难题转化为易于优化的课程学习任务,确保训练过程稳定、高效且泛化性强。

\section{训练与推理流程}
\label{sec:mia_training}

\subsection{训练流程}

算法~\ref{alg:mia_stage0}、算法~\ref{alg:mia_stage1}和算法~\ref{alg:mia_stage2}给出了渐进式三阶段训练的详细流程。

\begin{algorithm}[htbp]
  \caption{MIA阶段0:图像条件预热}
  \label{alg:mia_stage0}
  \begin{algorithmic}[1]
    \REQUIRE 公开数据集$\mathcal{D}_{\text{public}}=\{(x_i, y_i)\}$(如CelebA、FFHQ),换脸模型$G_\phi$(主干冻结),预训练身份编码器$E_{\text{id}}$,学习率$\eta_{\text{lora}}$
    \ENSURE 初步训练的LoRA参数$\{A_\ell, B_\ell\}$
    \STATE 初始化LoRA参数$\{A_\ell, B_\ell\}$(仅参考注意力层)
    \FOR{训练步数$t=1$ 到 $T_0$(通常200-500步)}
    \STATE 从$\mathcal{D}_{\text{public}}$中采样批数据$\{(x_{\text{real}}^{(i)}, y_i)\}_{i=1}^B$
    \STATE 采样源图像批$\{x_{\text{src}}^{(i)}\}_{i=1}^B$
    \FOR{批内样本$i=1$ 到 $B$}
    \STATE 从真实图像提取身份嵌入:$e_{\text{id}}^{(i)} = E_{\text{id}}(x_{\text{real}}^{(i)})$
    \STATE 采样扩散时间步:$t^{(i)} \sim \mathcal{U}(0, T)$
    \STATE 编码源图像:$z_0^{(i)} = \text{VAE}_{\text{enc}}(x_{\text{src}}^{(i)})$
    \STATE 添加噪声:$z_t^{(i)} = \sqrt{\bar{\alpha}_{t^{(i)}}} z_0^{(i)} + \sqrt{1-\bar{\alpha}_{t^{(i)}}} \epsilon^{(i)}$
    \STATE 预测噪声:$\hat{\epsilon}^{(i)} = \epsilon_{\theta}(z_t^{(i)}, t^{(i)}, e_{\text{id}}^{(i)})$
    \STATE 计算扩散先验损失:$\mathcal{L}_{\text{prior}}^{(i)} = w(t^{(i)}) \cdot (1 - \cos(\epsilon^{(i)}, \hat{\epsilon}^{(i)}))$
    \STATE 生成图像:$\hat{x}^{(i)} = G_{\phi}(e_{\text{id}}^{(i)}, x_{\text{src}}^{(i)})$
    \STATE 计算感知损失:$\mathcal{L}_{\text{perc}}^{(i)}$
    \ENDFOR
    \STATE 计算批平均损失:$\mathcal{L}_{\text{stage0}} = \frac{1}{B}\sum_{i=1}^B [\mathcal{L}_{\text{prior}}^{(i)} + \mathcal{L}_{\text{perc}}^{(i)}]$
    \STATE 更新LoRA参数:$\{A_\ell, B_\ell\} \leftarrow \{A_\ell, B_\ell\} - \eta_{\text{lora}} \cdot \nabla_{\{A_\ell, B_\ell\}} \mathcal{L}_{\text{stage0}}$
    \ENDFOR
    \RETURN $\{A_\ell, B_\ell\}$
  \end{algorithmic}
\end{algorithm}

\begin{algorithm}[htbp]
  \caption{MIA阶段1:混合条件过渡}
  \label{alg:mia_stage1}
  \begin{algorithmic}[1]
    \REQUIRE 公开数据集$\mathcal{D}_{\text{public}}$,阶段0训练的LoRA参数,嵌入层$\mathcal{E}_\psi$,退火周期$T_{\text{anneal}}$,学习率$\eta_{\text{emb}}, \eta_{\text{lora}}$
    \ENSURE 训练后的嵌入层$\psi$与LoRA参数$\{A_\ell, B_\ell\}$
    \STATE 初始化嵌入层参数$\psi$
    \FOR{训练步数$t=1$ 到 $T_1$(通常500-1000步)}
    \STATE 计算退火系数:$\lambda(t) = \frac{1}{2}(1 - \cos(\frac{\pi \cdot \min(t, T_{\text{anneal}})}{T_{\text{anneal}}}))$
    \STATE 计算对齐权重:$\beta(t) = \beta_0 \cdot (1 - \lambda(t))$
    \STATE 从$\mathcal{D}_{\text{public}}$中采样批数据$\{(x_{\text{real}}^{(i)}, y_i)\}_{i=1}^B$与源图像$\{x_{\text{src}}^{(i)}\}$
    \FOR{批内样本$i=1$ 到 $B$}
    \STATE 提取真实图像嵌入:$e_{\text{real}}^{(i)} = E_{\text{id}}(x_{\text{real}}^{(i)})$
    \STATE 生成标签嵌入:$e_{\text{label}}^{(i)} = \mathcal{E}_\psi(y_i)$
    \STATE 混合身份嵌入:$e_{\text{id}}^{(i)} = (1-\lambda(t)) \cdot e_{\text{real}}^{(i)} + \lambda(t) \cdot e_{\text{label}}^{(i)}$
    \STATE 采样时间步并生成噪声潜在变量(同阶段0)
    \STATE 计算$\mathcal{L}_{\text{prior}}^{(i)}, \mathcal{L}_{\text{perc}}^{(i)}$
    \STATE 计算嵌入归一化损失:$\mathcal{L}_{\text{emb-norm}}^{(i)} = (\|e_{\text{label}}^{(i)}\|_2 - 1)^2$
    \STATE 计算对齐损失:$\mathcal{L}_{\text{align}}^{(i)} = \|e_{\text{real}}^{(i)} - e_{\text{label}}^{(i)}\|_2^2$
    \ENDFOR
    \STATE 计算总损失:
    \STATE \quad $\mathcal{L}_{\text{stage1}} = \frac{1}{B}\sum_{i=1}^B [\mathcal{L}_{\text{prior}}^{(i)} + \mathcal{L}_{\text{perc}}^{(i)} + \mathcal{L}_{\text{emb-norm}}^{(i)} + \beta(t)\mathcal{L}_{\text{align}}^{(i)}]$
    \STATE 更新嵌入层:$\psi \leftarrow \psi - \eta_{\text{emb}} \cdot \nabla_\psi \mathcal{L}_{\text{stage1}}$
    \STATE 更新LoRA:$\{A_\ell, B_\ell\} \leftarrow \{A_\ell, B_\ell\} - \eta_{\text{lora}} \cdot \nabla_{\{A_\ell, B_\ell\}} \mathcal{L}_{\text{stage1}}$
    \ENDFOR
    \RETURN $\psi$, $\{A_\ell, B_\ell\}$
  \end{algorithmic}
\end{algorithm}

\begin{algorithm}[htbp]
  \caption{MIA阶段2:纯标签条件适配}
  \label{alg:mia_stage2}
  \begin{algorithmic}[1]
    \REQUIRE 目标分类器训练集$\mathcal{D}_{\text{target}}=\{(x_i, y_i)\}$,阶段1训练的$\psi$与$\{A_\ell, B_\ell\}$,目标分类器$F_\theta$,学习率$\eta_{\text{emb}}, \eta_{\text{lora}}$
    \ENSURE 最终微调的$\psi$与$\{A_\ell, B_\ell\}$
    \STATE 初始化任务不确定性参数$\{\sigma_i\}$
    \STATE 初始化U-Net注意力与残差块的LoRA参数(权重迁移自阶段1)
    \FOR{训练步数$t=1$ 到 $T_2$(通常1000-2000步)}
    \STATE 从$\mathcal{D}_{\text{target}}$中采样批数据$\{(x_i, y_i)\}_{i=1}^B$与源图像$\{x_{\text{src}}^{(i)}\}$
    \FOR{批内样本$i=1$ 到 $B$}
    \STATE 生成标签嵌入:$e_{\text{id}}^{(i)} = \mathcal{E}_\psi(y_i)$
    \STATE 执行扩散去噪与完整采样(同阶段1)
    \STATE 计算所有损失项:$\mathcal{L}_{\text{prior}}^{(i)}, \mathcal{L}_{\text{cls}}^{(i)}, \mathcal{L}_{\text{id}}^{(i)}, \mathcal{L}_{\text{perc}}^{(i)}, \mathcal{L}_{\text{reg}}^{(i)}$
    \ENDFOR
    \STATE 通过任务不确定性框架计算总损失:
    \STATE \quad $\mathcal{L}_{\text{stage2}} = \sum_{i\in\{\text{prior,cls,id,perc,reg}\}} \left(\frac{1}{2\sigma_i^2}\bar{\mathcal{L}}_i + \frac{1}{2}\log\sigma_i^2\right)$
    \STATE 更新参数:
    \STATE \quad $\psi \leftarrow \psi - \eta_{\text{emb}} \cdot \nabla_\psi \mathcal{L}_{\text{stage2}}$
    \STATE \quad $\{A_\ell, B_\ell\} \leftarrow \{A_\ell, B_\ell\} - 0.1\eta_{\text{lora}} \cdot \nabla_{\{A_\ell, B_\ell\}} \mathcal{L}_{\text{stage2}}$(注意降低学习率)
    \STATE \quad $\{\sigma_i\} \leftarrow \{\sigma_i\} - 0.1\eta_{\text{emb}} \cdot \nabla_{\{\sigma_i\}} \mathcal{L}_{\text{stage2}}$
    \ENDFOR
    \RETURN $\psi$, $\{A_\ell, B_\ell\}$
  \end{algorithmic}
\end{algorithm}

\subsection{推理策略}

推理阶段的目标是为给定目标类别$y_{\text{target}}$生成高质量的攻击样本。基本推理流程如下:首先通过标签条件嵌入层生成身份嵌入$e_{\text{id}} = \mathcal{E}_\psi(y_{\text{target}})$;然后结合源图像$x_{\text{src}}$通过换脸模型生成攻击样本$\hat{x} = G_{\phi+\Delta}(e_{\text{id}}, x_{\text{src}})$,其中$\Delta$表示训练得到的LoRA增量参数。

\textbf{源图像采样策略:}源图像$x_{\text{src}}$的选择对最终攻击效果有重要影响。本文采用多样性驱动的采样策略:(1)从公开人脸数据集(如CelebA、FFHQ)中随机采样$N_{\text{src}}$张图像作为候选源图像集合(推荐$N_{\text{src}}=20\text{-}50$);(2)确保候选集覆盖不同的姿态、表情、光照等属性,可通过预训练属性估计器度量候选集的属性方差,选择方差最大的子集;(3)对每个源图像生成攻击样本并通过目标分类器评估,选择使$F_\theta(\hat{x})_{y_{\text{target}}}$最大的前$K$个样本($K=5\text{-}10$)作为最终输出。该策略在保持样本多样性的同时优化攻击成功率。

\textbf{LoRA权重合并:}为消除推理开销,训练完成后可将LoRA权重合并到基础模型:$W' = W + \frac{\alpha}{r}BA$,使推理速度与原始REFace模型相当(单张图像约4.7秒,N=4步DDIM采样)。合并后的模型可作为独立的攻击工具部署,无需额外的适配层。

\section{本章小结}
\label{sec:mia_summary}

本章提出了基于换脸先验与标签条件嵌入的模型反演攻击方法,用于评估人脸分类模型的隐私泄露风险。该方法利用预训练换脸模型的生成能力,通过标签条件嵌入层将类别标签映射为身份嵌入,并采用LoRA技术进行参数高效微调,实现高质量的模型反演攻击。实验结果表明该方法在攻击成功率与图像质量方面均达到先进水平,为生物特征识别系统的安全性评估提供了有效工具。
% Local Variables:
% TeX-master: "../main"
% TeX-engine: xetex
% End:
