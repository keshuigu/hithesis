% !Mode:: "TeX:UTF-8"

% TODO (写作提示):
% - 在本章末尾补充“本文贡献”(3条 bullet,尽量量化,例如提升多少或节约多少计算量)。
% - 在本章末尾加一段“论文结构”指引,按章说明本文布局(1–2 句/章)。
% - 检查是否与摘要/关键词一致。
% 示例贡献(可修改后使用):
% \begin{itemize}
% \item 提出了一种基于扩散模型的高效图像特征反向重建方法,使重建准确率在 CelebA 上提升 X%。
% \item 设计了一种两阶段一致性微调策略,兼顾图像质量与模板匹配度。
% \item 提供复现代码与训练配置(附录/代码仓库链接)。
% \end{itemize}

\chapter{绪论}[Introduction]
近年来,随着科技的快速发展,处理器变得越来越强大,存储器变得越来越便宜,针对各种应用程序的大型图像数据库的部署已经成为现实。由于互联网上的图像信息迅速增长。图像检索技术在各个领域得到了广泛的应用。
基于内容的图像检索\cite{2015Content}(Content Based Image Retrieval,CBIR)是基于颜色、纹理和形状等视觉特征的图像检索。存储在数据库中的每个图像都被提取其特征并与查询图像的特征进行比较。
基于内容的图像检索应用方向十分多样:(1)安全检查:利用指纹或视网膜扫描等生物信息以获取访问权限;(2)知识产权:商标图像注册,将新的候选标记与现有标记进行比较,以确保没有混淆财产所有权的风险;(3)医疗诊断:在医学图像的医学数据库中使用基于内容的图像检索技术,识别类似的过去病例来辅助诊断。
\par
基于内容的图像检索技术依赖于图像局部特征提取技术。这项技术是从图像中提取出能够表征图像局部结构和纹理信息的特征点的方法。‌这些特征点通常具有旋转、‌尺度缩放、‌亮度变化等不变性,对视角变化、‌仿射变换、‌噪声也保持一定程度的稳定性。‌常见的图像局部特征提取算法包括‌尺度不变特征变换\cite{loweDistinctiveImageFeatures2004}(Scale-invariant feature transform,SIFT)、定向梯度直方图\cite{1467360}(Histogram of Oriented Gradients,HOG)‌、局部二值模式\cite{1017623}(Local Binary Patterns,LBP)‌等。‌这些算法通过检测图像中的关键点或角点,并提取其周围的局部图像块信息,生成特征信息,用于后续的图像匹配、‌检索等任务。

\par
由于图像特征提取技术的广泛使用,与图像特征相关的隐私和安全问题也引起了高度关注\cite{9762698}\cite{Qin2014TowardsEP}。事实证明,图像特征包括了丰富的图像信息,攻击者可以根据图像特征来获取隐私信息\cite{10214250}\cite{10.1145/3386082}。另外一篇具有代表性的工作\cite{5995616}表明,可以从一个图像的局部描述符来重建图像,重建后的图像能够表现出人类可理解的内容。
因此,通过利用图像特征进行重建图像的攻击具备了可行性。考虑以下情景:在一个图像检索服务场景中,由于本地计算设备限制以及隐私需求,用户仅将待查询图像的图像特征传输到远程服务提供商。远程服务提供商使用用户上传的图像特征进行基于内容的图像检索,最终将检索到的图像返回给用户,从而完成一次图像检索。
假设在这个过程中攻击者可以获取到用户上传的图像特征。那么,攻击者可以用该图像特征作为输入,利用自己的攻击模型生成与原始图像视觉效果近似的图像\cite{10.1145/3599589.3599596}\cite{SUN2020102642},从而窃取用户的隐私信息。
\par
本课题主要以攻击者的角度来关注图像特征的隐私泄露问题。针对目前已有的基于图像特征进行反向重建图像的攻击方法所存在的各个问题,从高效率的图像特征反向攻击方法、高精准度的图像特征反向攻击方法的角度分别进行研究。

% ====== 写作补充:贡献与论文结构(示例,可修改) ======
\section*{本文贡献}
\begin{itemize}
	\item 提出了一种基于隐扩散模型(EDM)的图像特征反向重建方法,能够在特征空间引入分类器分数引导,提高重建与目标模板的一致性(详见第3章与第5章)。
	\item 设计了一种两阶段一致性微调策略,在保持图像质量的同时增强与目标特征的匹配度,从而提高攻击成功率并降低结构性失真。
	\item 提供完整的实验设置与复现说明(超参数、评估脚本和示例命令),以便同行复现实验结果(见第5章与附录/代码仓库)。
\end{itemize}

\section*{论文结构}
本文其余部分组织如下:第2章回顾相关的图像生成模型与扩散模型理论基础;第3章介绍面向人脸特征提取模型的模板逆向攻击方法(TIA);第4章介绍面向人脸分类模型的模型反演方法(MIA);第5章呈现实验设置、评价指标与实验结果分析;最后总结全文并讨论未来研究方向。

% Local Variables:
% TeX-master: "../main"
% TeX-engine: xetex
% End:
