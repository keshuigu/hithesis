% !Mode:: "TeX:UTF-8"

\chapter{绪论}[Introduction]\label{chap:introduction}
\section{课题背景及研究的目的和意义}\label{sec:background}

随着深度学习技术的快速发展与计算能力的持续提升,基于深度神经网络的人脸识别系统在性能上取得了突破性进展,已成为生物特征识别领域最为成熟和广泛部署的技术之一\cite{taigman2014deepface,schroff2015facenet,cao2018vggface2}。当前,人脸识别技术已深度渗透至身份认证、公共安全监控、金融支付、门禁控制、社交媒体等诸多安全敏感应用场景\cite{learned2016labeled,deng2019arcface},在全球范围内形成了庞大的用户基数和数据规模。这种大规模的应用部署在带来便捷性和高效性的同时,也使得人脸识别系统成为攻击者关注的重要目标,其安全性与隐私保护问题日益凸显\cite{sharif2016accessorize,komkov2021advhat}。

从技术架构角度审视,现代人脸识别系统的核心在于通过深度神经网络实现从原始人脸图像到身份判别信号的映射\cite{he2016deep,deng2019arcface}。根据系统的输出形式与后续处理流程,当前主流的人脸识别系统可细分为两种主要架构类型:

(1)基于模板匹配的检索型架构:该类系统在注册阶段为每个身份存储特征模板,在识别阶段通过计算查询特征与模板库中所有模板的相似度进行身份检索与验证。这种架构的优势在于开放集适应性与可扩展性,但特征模板长期存储于数据库中的安全隐患突出:一旦数据库遭到入侵或内部泄密,攻击者可直接获取大量用户的特征模板,进而通过逆向重建技术恢复用户的面部图像,对用户隐私构成严重威胁\cite{mai2018reconstruction,cole2017synthesizing}。

(2)基于分类的端到端架构:该类系统将分类层直接嵌入特征提取器之后,模型输出为身份类别的概率分布或标签。这种架构在封闭集场景下性能较高且实现相对简单,但模型输出通常包含丰富的置信度信息或概率分布,攻击者可利用这些信息,通过优化或生成模型技术重建训练样本的近似图像,从而泄露用户隐私\cite{fredrikson2015model,zhang2020secret}。

两种架构在安全性与隐私保护方面面临的威胁本质上源于同一核心问题:人脸识别系统在实现高性能识别的同时,不可避免地在中间层或输出层暴露了与原始图像高度相关的语义信息\cite{he2019model,chen2021knowledge}。这些信息虽然经过了非线性变换和降维,但仍保留了足够的身份判别能力,从而为攻击者实施逆向重建提供了可能。关于人脸识别系统的详细工作原理、特征提取流程及嵌入空间表示方法,将在第\ref{chap:theory}章第\ref{sec:face_recognition}节中系统阐述。基于上述隐私风险分析,针对人脸识别系统的逆向重建攻击主要可分为两大类:其一是针对特征模板的逆向重建攻击(Template Inversion Attack, TIA),攻击者在获得系统存储的特征嵌入向量后,试图重建出与该特征对应的人脸图像\cite{mai2018reconstruction,cole2017synthesizing};其二是针对分类模型的模型反演攻击(Model Inversion Attack, MIA),攻击者利用模型输出的置信度或概率分布等信息,重建训练数据的近似样本\cite{fredrikson2015model,zhang2020secret}。这两类攻击分别针对前述的检索型架构和分类型架构,反映了不同系统设计下的隐私泄露风险。

从技术发展脉络看,逆向重建方法经历了从基于优化的方法到基于生成对抗网络再到基于扩散模型等深度生成技术的演进,重建攻击的威胁能力得到显著提升。具体的技术进展与方法比较详见第\ref{sec:related_work}节的研究现状分析。尽管现有工作取得了显著进展,但仍存在关键瓶颈亟待突破:面向连续特征空间的引导机制设计不足、感知质量与身份一致性的协同优化困难、以及生成效率与资源消耗之间的平衡问题。基于上述分析与所述问题,本文拟从理论建模、方法设计与实验验证三个层面开展研究,针对TIA和MIA两类典型威胁场景,分别设计基于扩散模型的高效攻击方法,系统评估和揭示人脸识别系统在特征模板泄露或模型输出暴露情形下的隐私风险。


\section{国内外研究现状及分析}\label{sec:related_work}

人脸识别系统的隐私安全问题是学术界和工业界共同关注的前沿课题。本节从逆向重建攻击方法、生成模型技术发展以及评估体系三个维度,系统梳理国内外相关研究现状,分析现有工作的优势与不足,并明确本文研究的切入点与创新方向。

\subsection{逆向重建攻击方法研究}

逆向重建攻击旨在从深度神经网络的中间表示或输出信息中恢复原始输入数据,对人脸识别系统的隐私安全构成严重威胁。根据攻击目标的不同,可分为模板逆向重建攻击和模型反演攻击两类。从技术路径看,逆向重建方法经历了从基于优化到基于生成模型的演进,本节按照技术发展脉络展开综述。

\subsubsection{基于优化的逆向重建方法}

早期研究主要采用基于优化的方法实现特征到图像的逆向映射。Mahendran和Vedaldi\cite{Mahendran_2015_CVPR}首次系统研究了从深度卷积神经网络的中间层特征重建图像的可行性,提出通过最小化特征空间距离并结合自然图像先验来优化像素值。该工作表明,即使是经过多层非线性变换的深度特征,仍然保留了足够的结构信息用于重建可识别的图像。Dosovitskiy和Brox\cite{Dosovitskiy_2016_CVPR}进一步探索了从不同网络层级和不同类型特征进行逆向重建的能力,发现浅层特征包含更多纹理细节,而深层特征更侧重于语义和身份信息。

针对人脸识别系统,Mai等人\cite{5995616}证明了通过反向优化可以从局部二值模式、方向梯度直方图等传统图像特征中恢复原始人脸图像。进入深度学习时代后,针对深度特征嵌入的逆向重建成为研究重点。这类方法通常将重建问题建模为如下优化问题:
\begin{equation}
\hat{x} = \arg\min_{x} \, \|F(x) - t\|^2 + \lambda_{\text{TV}} \mathcal{R}_{\text{TV}}(x) + \lambda_{\text{norm}} \|x\|^2,
\end{equation}
其中 $F(\cdot)$ 为特征提取器,$t$ 为目标模板,$\mathcal{R}_{\text{TV}}(\cdot)$ 为全变分正则化项用于平滑图像,$\lambda_{\text{TV}}$ 和 $\lambda_{\text{norm}}$ 为权衡系数。Fredrikson等人\cite{fredrikson2015model}将类似思想应用于模型反演攻击,证明当分类模型返回完整的类别置信度分布时,攻击者可通过迭代优化重建出与训练样本高度相似的图像。

然而,基于优化的方法存在明显局限:(1)优化过程通常需要数千次迭代,计算开销大;(2)在信息受限场景下容易陷入局部极值;(3)生成的图像可能存在高频噪声或非自然纹理。这些局限促使研究者转向基于生成模型的学习式方法。

\subsubsection{基于生成对抗网络的方法}

生成对抗网络的出现为逆向重建提供了新的技术路径。Cole等人\cite{cole2017synthesizing}首次将生成对抗网络应用于模板逆向重建攻击,通过在大规模人脸数据集上训练条件生成对抗网络,学习从低维特征嵌入到高质量人脸图像的逆映射。该方法的核心优势在于利用了生成对抗网络学到的自然人脸先验,避免了显式定义正则化项的困难,且推理速度快。

在模型反演攻击方面,Zhang等人\cite{zhang2020secret}提出的生成式模型反演(Generative Model Inversion, GMI)方法将生成对抗网络与梯度引导相结合,在目标分类模型的置信度指导下训练条件生成对抗网络,实现了黑盒场景下的高保真反演。Chen等人\cite{chen2022mirror}进一步利用预训练的StyleGAN\cite{karras2019style}先验,结合分类损失和感知损失,在ImageNet和人脸数据集上实现了高保真度的模型反演。Yuan等人\cite{yuan2022pseudo}引入伪标签引导机制,通过自动生成辅助训练样本改善条件生成对抗网络的训练质量,进一步提升了黑盒场景下的反演成功率。

尽管基于生成对抗网络的方法在生成质量上取得了显著进展,但仍存在训练不稳定、模式崩塌以及多样性不足等固有问题,限制了其在某些场景下的应用。

\subsubsection{基于扩散模型的方法}

近年来,扩散模型在图像生成领域取得了突破性进展,相比GAN具有训练稳定、生成质量高、模式覆盖好等优势。Ho等人\cite{hoDenoisingDiffusionProbabilistic2020}提出的去噪扩散概率模型(Denoising Diffusion Probabilistic Models, DDPM)通过学习数据分布的逆向去噪过程,实现了高质量的图像生成。Song等人\cite{songScoreBasedGenerativeModeling2021}从随机微分方程的角度统一了扩散模型的理论框架,提出了基于分数的生成模型并设计了更高效的采样算法。Karras等人\cite{karrasEluvidatingDiffusionModels2022}进一步提出明晰扩散模型(Elucidating Diffusion Models, EDM),通过重新设计噪声调度、采样算法和网络架构,系统性地分析了扩散模型的各个组件,在生成质量和采样效率上取得了显著提升。Rombach等人\cite{rombachHighResolutionImageSynthesis2022}提出的隐空间扩散模型(Latent Diffusion Models, LDM)将扩散过程从像素空间转移到低维隐空间,显著提高了训练和采样效率,并支持灵活的条件注入机制。

在将扩散模型应用于逆向重建任务时,条件引导机制是关键技术。Dhariwal和Nichol\cite{dhariwalDiffusionModelsBeat2021}提出分类器引导(Classifier Guidance)方法,在采样过程中利用预训练分类器的梯度信息修正采样轨迹,显著提升了生成样本的条件一致性和视觉质量。Ho和Salimans\cite{ho2022classifierfree}进一步提出无分类器引导(Classifier-free Guidance)策略,通过在训练阶段同时学习条件生成和无条件生成,在采样时对比两者的分数函数差异来实现隐式引导,避免了额外分类器的计算开销。Song等人\cite{song2021denoising}提出的去噪扩散隐式模型(Denoising Diffusion Implicit Models, DDIM)通过引入确定性采样轨迹加速生成过程,为条件引导提供了更高效的基础框架。

针对特征匹配类逆问题,研究者提出了专门的引导策略。Chung等人\cite{chung2022diffusion}提出扩散后验采样方法,将条件约束建模为似然函数,通过贝叶斯推断框架在采样过程中融合先验和似然信息,为特征匹配任务提供了理论上更严谨的引导机制。Kawar等人\cite{kawar2022denoising}提出的去噪扩散修复模型(Denoising Diffusion Restoration Models, DDRM)通过在频域中分解观测算子实现高效的条件采样。Bansal等人\cite{bansal2023universal}提出通用引导方法,可以灵活地集成多种类型的条件约束(包括连续特征向量、离散类别标签、空间掩码等),通过统一的梯度计算接口实现对采样过程的精确控制。Chung等人\cite{chung2023comebacksampler}从随机收缩的角度分析了条件扩散模型的收敛性,提出了加速采样算法,在保证生成质量的同时显著减少了采样步数。

扩散模型及其引导策略为逆向重建攻击提供了新的技术范式。Struppek等人\cite{struppek2022plug}提出的Plug \& Play框架将攻击过程解耦为生成器训练和优化引导两个阶段,使得不同的预训练生成模型(生成对抗网络、扩散模型等)可以灵活应用于反演任务。这些工作表明,通过合理利用扩散模型的生成能力和条件控制机制,可以在保持高视觉保真度的同时精确满足身份匹配约束,显著提升了逆向重建攻击的威胁能力。


\subsection{国内外研究现状分析}

通过对国内外研究现状的系统梳理与深入分析,可以发现现有工作在以下关键方面仍存在局限性,亟待进一步突破:

(1)面向连续特征空间的扩散引导机制适配性不足。尽管扩散模型在条件图像生成领域表现卓越,但现有的引导策略主要针对离散类别标签或模态间对齐设计。人脸识别系统中的特征嵌入向量具有高维、连续及特定的度量空间结构,直接套用通用的离散引导机制往往难以捕捉其精细的几何特性。

(2)生成感知质量与身份一致性的协同优化机制缺失。现有逆向重建技术普遍面临“质量-一致性”的权衡困境:基于优化的方法虽然能精确拟合特征,但重建图像常伴随高频噪声与非自然伪影;基于生成对抗网络的方法虽能保证视觉真实性,却易受模式崩塌影响导致身份信息漂移。尽管扩散模型具备更强的分布建模能力,但如何在去噪采样过程中动态平衡感知先验与身份语义约束,避免顾此失彼,目前尚缺乏系统的理论指导与有效的控制策略。

基于上述分析,本研究将从以下两个维度切入,旨在构建高效、高保真的逆向重建攻击框架:

(1)设计适配连续特征嵌入的动态引导与生成机制。针对特征模板逆向重建任务,本研究将基于先进的明晰扩散模型架构,提出面向连续特征向量的嵌入一致性引导策略。通过深入分析扩散采样过程中的信噪比演变规律,设计自适应的动态引导算法,在采样轨迹中融合识别模型的特征空间约束与生成模型的自然图像先验,实现对目标特征的精确逆向映射。

(2)构建基于强生成先验与参数高效微调的协同反演框架。针对模型反演攻击任务,本研究将探索利用预训练扩散换脸模型中蕴含的丰富人脸结构与纹理先验,结合低秩适配技术进行参数高效微调。通过将通用的人脸生成能力快速迁移至目标身份域,并协同分类器置信度引导,实现在有限计算资源下对目标身份的高保真、高一致性重建,解决传统方法在质量与效率上的瓶颈。

通过上述研究,本文旨在为人脸识别系统的隐私安全评估提供更具威胁性、更系统的技术方案,从攻防对抗的角度推动生物特征保护技术的发展。

\section{本文的研究内容及章节安排}\label{sec:thesis_structure}

\subsection{主要研究内容}

为方便后文形式化与讨论,本节对本文研究的两类主要逆向重建攻击任务给出明确定义与形式化描述:

模板逆向重建攻击针对基于特征提取器的检索型人脸识别系统。设特征提取器为 $f: \mathbb{R}^{H \times W \times C} \to \mathbb{R}^d$,将输入图像 $x$ 映射为 $d$ 维特征嵌入 $e = f(x)$。攻击者通过数据库泄露等途径获得目标身份的特征模板 $t = f(x_0) \in \mathbb{R}^d$,其目标是构造逆映射函数 $g: \mathbb{R}^d \to \mathbb{R}^{H \times W \times C}$,生成图像 $\hat{x} = g(t)$,使得其满足两个特性:(1)身份一致性:$f(\hat{x}) \approx t$,即生成图像的特征嵌入与目标模板在度量空间中接近,满足 $\text{sim}(f(\hat{x}), t) > \tau$,其中 $\text{sim}(\cdot, \cdot)$ 为相似度度量函数(常采用余弦相似度),$\tau$ 为识别系统的验证阈值;(2)视觉真实性:$\hat{x}$ 具有自然的人脸特征和纹理细节,在感知质量上与真实人脸图像无明显差异。

形式化地,模板逆向重建攻击问题可建模为如下优化问题:
\begin{equation}
\hat{x} = \arg\min_x \; \mathcal{L}_{embed}(f(x),t) + \lambda_{perc}\mathcal{L}_{perc}(x) + \lambda_{reg}R(x),
\end{equation}
其中 $\mathcal{L}_{embed}$ 为嵌入空间距离损失,可采用余弦距离 $1-\cos(f(x),t)$ 或欧氏距离 $\|f(x)-t\|_2^2$;$\mathcal{L}_{perc}$ 为感知质量损失;$R(\cdot)$ 为正则化项或先验约束;$\lambda_{perc}$ 和 $\lambda_{reg}$ 为权衡系数。

模型反演攻击针对基于分类的端到端人脸识别系统。设分类模型为 $F_\theta: \mathcal{X} \to \mathbb{R}^C$,将输入图像 $x \in \mathcal{X}$ 映射为 $C$ 个类别的置信度向量。攻击者针对目标类别 $y^* \in \{1, 2, \ldots, C\}$,通过访问模型输出与梯度信息,生成图像集合 $\{\hat{x}_1, \hat{x}_2, \ldots, \hat{x}_K\}$,使其同时满足以下条件:(1)分类器认同度:生成图像 $\hat{x}_k$ 被目标分类器以高置信度识别为目标类别,即 $F_\theta(\hat{x}_k)_{y^*} \geq \tau$;(2)感知真实性:生成图像在视觉上与真实人脸无明显差异,感知距离 $d_{\text{perc}}(\hat{x}_k, \mathcal{X}_{\text{real}}) \leq \delta$;(3)身份相关性:生成图像包含目标类别对应身份的特征信息,与训练样本相似度 $\text{sim}(\hat{x}_k, \mathcal{X}_{y^*}) \geq \gamma$;(4)多样性:生成的多个样本覆盖不同姿态、表情、光照等变化,避免模式崩溃。综合上述目标,MIA可建模为约束优化问题:
\begin{equation}
\begin{aligned}
    \max_{\{\hat{x}_k\}_{k=1}^K} \quad & \sum_{k=1}^K F_\theta(\hat{x}_k)_{y^*} \\\
    \text{s.t.} \quad & d_{\text{perc}}(\hat{x}_k, \mathcal{X}_{\text{real}}) \leq \delta, \quad \forall k, \\\
    & \text{sim}(\hat{x}_k, \mathcal{X}_{y^*}) \geq \gamma, \quad \forall k, \\\
    & \text{Var}(\{\hat{x}_k\}_{k=1}^K) \geq \rho.
\end{aligned}
\end{equation}
本文采用生成模型驱动的攻击策略:利用预训练扩散换脸模型的先验知识,通过参数高效微调快速适配目标分类器,在采样过程中融入分类器梯度引导,在保证视觉质量的同时实现精确的身份匹配。

本研究围绕人脸识别系统中的逆向重建攻击问题展开,针对模板逆向重建攻击和模型反演攻击两类典型威胁场景,分别设计基于扩散模型的攻击方法。研究工作涵盖理论建模、方法设计与实验验证三个层面,具体内容如下:

\subsubsection{基于明晰扩散模型的模板逆向重建方法}

针对特征提取模型的模板逆向重建攻击,本文提出基于明晰扩散模型的攻击框架。尽管EDM、分类器引导等技术在图像生成领域已有研究,但将其系统应用于人脸识别系统的特征模板逆向重建场景,针对高维连续特征空间设计自适应引导策略,以及在检索型架构下进行隐私风险评估,现有研究在特征匹配精度与生成质量的平衡上仍有较大提升空间。本文方法针对特征空间的特殊性和实际应用中的威胁模式,设计了动态权衡参数控制机制,在保证生成质量的同时精确匹配目标模板。具体地,本文采样早期以EDM的强先验保证视觉自然度,后期通过自适应增强的特征约束实现身份一致性,通过这种渐进式的权衡策略在检索型人脸识别系统中有效评估特征模板泄露的隐私风险。该方法在第3章中详细阐述,并在第5章实验中进行全面的对比验证。

\subsubsection{基于扩散换脸与低秩适配微调的模型反演方法}

针对分类模型的模型反演攻击,本文将扩散换脸模型应用于MIA任务,提出融合换脸先验、标签条件嵌入与渐进式微调的攻击框架。区别于现有工作使用通用生成模型(GAN或扩散模型),本文发现换脸模型的显式身份-属性解耦机制和专用身份控制接口特别适合身份重建任务。然而,换脸模型通常需要真实目标图像作为身份输入,而MIA场景仅有类别标签,针对这一核心矛盾,本文设计标签条件嵌入层将类别标签映射为身份表示,并提出三阶段渐进式微调策略:通过图像预热、混合过渡、纯标签生成的递进式训练,结合余弦退火调度实现从图像条件到标签条件的平滑迁移,显著提升训练稳定性与收敛速度。同时采用LoRA技术进行参数高效微调,仅需1\%-5\%参数量即可适配目标分类器。该方法在第4章中详细阐述,并在第5章通过多个实验场景展示其在攻击准确率与生成保真度上的显著优势。

\subsection{章节安排}

本文共分为五章,各章节内容安排与逻辑关系如下:

第~\ref{chap:introduction}~章:绪论。本章阐述人脸识别技术的广泛应用现状与面临的安全威胁,阐明特征模板逆向重建和模型反演攻击的研究价值。通过系统综述逆向重建攻击、深度生成模型等领域的国内外研究进展,分析现有工作的不足与本文的创新切入点。最后阐明本文的研究内容、主要贡献与篇章组织结构。

第~\ref{chap:theory}~章:相关理论与技术基础。本章介绍支撑本研究的核心理论与关键技术。阐述人脸识别模型的基本原理,包括特征提取网络、嵌入空间表示与相似度度量方法;详细论述深度生成模型的发展脉络,重点涵盖明晰扩散模型的设计原理、噪声调度和采样机制,以及扩散换脸模型的架构特点;阐述参数高效微调技术LoRA的数学原理与在扩散模型中的应用;定义本研究采用的评估指标体系,包括嵌入相似度、识别成功率、感知质量与计算效率等。本章为后续方法设计与实验分析奠定理论基础。

第~\ref{chap:TIA}~章:基于EDM与动态权衡的模板逆向重建方法。本章阐述面向特征提取模型的模板逆向重建攻击方法。形式化定义TIA攻击的任务设定、威胁模型与约束条件;提出基于明晰扩散模型的逆向重建框架,阐述EDM的生成流程、噪声调度策略与采样机制;设计动态权衡参数控制机制,使引导强度 $\lambda(t)$ 根据采样阶段和特征匹配程度自适应调整;推导分类器引导损失 $\mathcal{L}_{cls}(x_t, t)$ 的梯度计算与融入方法。通过消融实验验证各模块的贡献度,分析关键超参数对生成质量的影响。本章构建了系统的TIA攻击方法,为后续实验验证提供理论基础。

第~\ref{chap:MIA}~章:基于扩散换脸与LoRA的模型反演攻击方法。本章面向分类型人脸识别系统的模型反演威胁,阐述与第3章TIA攻击的方法论联系与区别。分析两类攻击在威胁模型、信息访问权限与约束条件上的差异,揭示MIA攻击的特殊性;阐述预训练扩散换脸模型的迁移适配方法及其相比通用扩散模型的优势;设计基于LoRA的参数高效微调策略,说明关键设计决策如注入位置、秩选择与训练数据构建;推导分类器引导损失函数及其在采样过程中的融入机制;描述三阶段训练流程的实现与优化方法。本章在白盒攻击假设下系统论证了扩散换脸先验与参数高效微调在模型反演中的有效性。

第~\ref{chap:Results}~章:实验设计与结果分析。本章通过系统的实验验证所提方法的有效性与优越性。介绍实验环境配置、数据集选择、目标识别模型设置与基线方法的实现细节;针对TIA攻击评估EDM框架、参数控制机制与引导策略的有效性,分析采样步数、引导强度等关键超参数的影响;针对MIA攻击评估扩散换脸先验、LoRA微调与三阶段训练的贡献度,分析LoRA秩、微调数据量等超参数的设置影响。通过对比实验展示所提方法在身份匹配度、感知质量与计算效率上的优越性;通过消融实验量化各关键模块的作用,统计参数量、训练时间、推理时间等工程指标;通过可视化结果展示生成样本质量与身份一致性,分析成功与失败案例。本章以全面的实验数据与深入的分析论证了所提方法的科学性与先进性。

% Local Variables:
% TeX-master: "../main"
% TeX-engine: xetex
% End:
