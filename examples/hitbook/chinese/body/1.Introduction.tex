% !Mode:: "TeX:UTF-8"

\chapter{绪论}[Introduction]\label{chap:introduction}
\section{课题背景及研究的目的和意义}\label{sec:background}

随着深度学习技术的快速发展与计算能力的持续提升,基于深度神经网络的人脸识别系统在性能上取得了突破性进展,已成为生物特征识别领域最为成熟和广泛部署的技术之一\cite{taigman2014deepface,schroff2015facenet,cao2018vggface2}。当前,人脸识别技术已深度渗透至身份认证、公共安全监控、金融支付、门禁控制、社交媒体等诸多安全敏感应用场景\cite{learned2016labeled,deng2019arcface},在全球范围内形成了庞大的用户基数和数据规模。这种大规模的应用部署在带来便捷性和高效性的同时,也使得人脸识别系统成为攻击者关注的重要目标,其安全性与隐私保护问题日益凸显\cite{sharif2016accessorize,komkov2021advhat}。

从技术架构角度审视,现代人脸识别系统的核心在于通过深度神经网络实现从原始人脸图像到身份判别信号的映射\cite{he2016deep,deng2019arcface}。根据系统的输出形式与后续处理流程,当前主流的人脸识别系统可细分为两种主要架构类型:

(1)基于模板匹配的检索型架构:该类系统在注册阶段为每个身份存储特征模板,在识别阶段通过计算查询特征与模板库中所有模板的相似度进行身份检索与验证。这种架构的优势在于开放集适应性与可扩展性,但特征模板长期存储于数据库中的安全隐患突出:一旦数据库遭到入侵或内部泄密,攻击者可直接获取大量用户的特征模板,进而通过逆向重建技术恢复用户的面部图像,对用户隐私构成严重威胁\cite{mai2018reconstruction,cole2017synthesizing}。

(2)基于分类的端到端架构:该类系统将分类层直接嵌入特征提取器之后,模型输出为身份类别的概率分布或标签。这种架构在封闭集场景下性能较高且实现相对简单,但模型输出通常包含丰富的置信度信息或概率分布,攻击者可利用这些信息,通过优化或生成模型技术重建训练样本的近似图像,从而泄露用户隐私\cite{fredrikson2015model,zhang2020secret}。

两种架构在安全性与隐私保护方面面临的威胁本质上源于同一核心问题:人脸识别系统在实现高性能识别的同时,不可避免地在中间层或输出层暴露了与原始图像高度相关的语义信息\cite{he2019model,chen2021knowledge}。这些信息虽然经过了非线性变换和降维,但仍保留了足够的身份判别能力,从而为攻击者实施逆向重建提供了可能。关于人脸识别系统的详细工作原理、特征提取流程及嵌入空间表示方法,将在第\ref{chap:theory}章第\ref{sec:face_recognition}节中系统阐述。基于上述隐私风险分析,针对人脸识别系统的逆向重建攻击主要可分为两大类别:其一是模板逆向攻击(Template Inversion Attack, TIA),攻击者在获得系统存储的特征嵌入向量后,试图重建出与该特征对应的人脸图像\cite{mai2018reconstruction,cole2017synthesizing};其二是模型反演攻击(Model Inversion Attack, MIA),攻击者利用模型输出的置信度或概率分布等信息,重建训练数据的近似样本\cite{fredrikson2015model,zhang2020secret}。这两类攻击分别针对前述的检索型架构和分类型架构,反映了不同系统设计下的隐私泄露风险。

从技术发展脉络看,逆向重建方法经历了从基于优化的方法到基于生成对抗网络再到基于扩散模型等深度生成技术的演进,重建攻击的威胁能力得到显著提升。具体的技术进展与方法比较详见第\ref{sec:related_work}节的研究现状分析。尽管现有工作取得了显著进展,但仍存在关键瓶颈亟待突破:面向连续特征空间的引导机制设计不足、感知质量与身份一致性的协同优化困难、以及生成效率与资源消耗之间的平衡问题。基于上述分析与所述问题,本文拟从理论建模、方法设计与实验验证三个层面开展研究,针对模板逆向和模型反演两类典型威胁场景,分别提出基于角度约束扩散模型的模板逆向方法和基于渐进式训练的模型反演方法,系统评估和揭示人脸识别系统在特征模板泄露或模型输出暴露情形下的隐私风险。


\section{国内外研究现状及分析}\label{sec:related_work}

人脸识别系统的隐私安全问题是学术界和工业界共同关注的前沿课题。本节从逆向重建攻击方法、生成模型技术发展以及评估体系三个维度,系统梳理国内外相关研究现状,分析现有工作的优势与不足,并明确本文研究的切入点与创新方向。

\subsection{逆向重建攻击方法研究}

逆向重建攻击旨在从深度神经网络的中间表示或输出信息中恢复原始输入数据,对人脸识别系统的隐私安全构成严重威胁。根据攻击目标的不同,可分为模板逆向重建攻击和模型反演攻击两类。从技术路径看,逆向重建方法经历了从基于优化到基于生成模型的演进,本节按照技术发展脉络展开综述。

\subsubsection{基于优化的逆向重建方法}

早期研究主要采用基于优化的方法实现特征到图像的逆向映射。Mahendran和Vedaldi\cite{Mahendran_2015_CVPR}首次系统研究了从深度卷积神经网络的中间层特征重建图像的可行性,提出通过最小化特征空间距离并结合自然图像先验来优化像素值。该工作表明,即使是经过多层非线性变换的深度特征,仍然保留了足够的结构信息用于重建可识别的图像。Dosovitskiy和Brox\cite{Dosovitskiy_2016_CVPR}进一步探索了从不同网络层级和不同类型特征进行逆向重建的能力,发现浅层特征包含更多纹理细节,而深层特征更侧重于语义和身份信息。

针对人脸识别系统,Mai等人\cite{5995616}证明了通过反向优化可以从局部二值模式、方向梯度直方图等传统图像特征中恢复原始人脸图像。进入深度学习时代后,针对深度特征嵌入的逆向重建成为研究重点。这类方法通常将重建问题建模为如下优化问题:
\begin{equation}
\hat{x} = \arg\min_{x} \, \|F(x) - t\|^2 + \lambda_{\text{TV}} \mathcal{R}_{\text{TV}}(x) + \lambda_{\text{norm}} \|x\|^2
\end{equation}
其中 $F(\cdot)$ 为特征提取器,$t$ 为目标模板,$\mathcal{R}_{\text{TV}}(\cdot)$ 为全变分正则化项用于平滑图像,$\lambda_{\text{TV}}$ 和 $\lambda_{\text{norm}}$ 为权衡系数。Fredrikson等人\cite{fredrikson2015model}将类似思想应用于模型反演攻击,证明当分类模型返回完整的类别置信度分布时,攻击者可通过迭代优化重建出与训练样本高度相似的图像。

然而,基于优化的方法存在明显局限:(1)优化过程通常需要数千次迭代,计算开销大;(2)在信息受限场景下容易陷入局部极值;(3)生成的图像可能存在高频噪声或非自然纹理。这些局限促使研究者转向基于生成模型的学习式方法。

\subsubsection{基于生成对抗网络的方法}

生成对抗网络的出现为逆向重建提供了新的技术路径。Cole等人\cite{cole2017synthesizing}首次将生成对抗网络应用于模板逆向重建攻击,通过在大规模人脸数据集上训练条件生成对抗网络,学习从低维特征嵌入到高质量人脸图像的逆映射。该方法的核心优势在于利用了生成对抗网络学到的自然人脸先验,避免了显式定义正则化项的困难,且推理速度快。

Mai等人\cite{mai2019reconstruction}提出的NBNet系列方法通过邻域感知的模板逆向网络实现了高质量的人脸重建。该方法的核心思想是利用特征空间的邻域信息辅助生成过程:首先在特征空间中检索与目标模板相近的参考样本,然后通过注意力机制融合邻域特征的结构信息指导生成对抗网络的重建过程。NBNetA和NBNetB两个变体分别采用不同的邻域聚合策略,在多个人脸数据集上实现了较高的攻击成功率。Dong等人\cite{dong2021towards}提出了基于知识蒸馏的模板逆向重建方法,通过将教师网络学到的特征分布知识迁移至生成网络,增强了重建图像的特征一致性。该方法在保持视觉质量的同时显著提升了特征匹配精度。Vendrow等人\cite{vendrow2021realistic}关注特征逆向重建的真实性评估问题,提出了结合人类感知评价与自动指标的综合评估框架,并设计了基于感知约束的逆向重建方法。Dong等人\cite{dong2023reconstruct}进一步改进了基于知识蒸馏的框架,引入多尺度特征对齐机制与渐进式训练策略,在LFW、AgeDB等数据集上取得了更高的重建质量。

Shahreza等人\cite{shahreza2023template}提出的GaFaR(GANs for Face Reconstruction)方法代表了基于生成对抗网络的模板逆向重建技术的最新进展。该方法采用多判别器架构分别约束全局结构与局部细节,通过对抗训练同时优化身份一致性与视觉真实性。GaFaR引入了特征空间正则化机制防止模式崩塌,并设计了自适应权重调整策略动态平衡不同损失项的贡献。实验表明,GaFaR在误识率$10^{-3}$的严格阈值下仍能保持较高的攻击成功率。

在模型反演攻击方面,Zhang等人\cite{zhang2020secret}提出的生成式模型反演(Generative Model Inversion, GMI)方法将生成对抗网络与梯度引导相结合,在目标分类模型的置信度指导下训练条件生成对抗网络,实现了黑盒场景下的高保真反演。GMI方法首先在公开数据集上预训练生成器学习人脸先验分布,然后在白盒场景下利用目标分类器的梯度信息微调生成器,使生成图像在保持自然外观的同时被分类器高置信度识别为目标类别。Chen等人\cite{chen2021knowledge}提出的KED-MI(Knowledge-Enriched Distributional Model Inversion)方法通过知识蒸馏技术将教师模型的决策边界信息迁移至学生生成器,增强了生成样本的类别判别能力。该方法采用多阶段训练策略,逐步提升生成图像与目标类别的一致性。Yuan等人\cite{yuan2022pseudo}提出的PLG-MI(Pseudo Label Guided Model Inversion)方法引入伪标签引导机制,通过自动生成辅助训练样本改善条件生成对抗网络的训练质量。PLG-MI首先利用目标分类器为公开数据集中的样本生成伪标签,然后以这些伪标签样本作为额外的监督信号训练生成器,显著提升了黑盒场景下的反演成功率和生成多样性。

尽管基于生成对抗网络的方法在生成质量上取得了显著进展,但仍存在训练不稳定、模式崩塌以及多样性不足等固有问题,限制了其在某些场景下的应用。

\subsubsection{基于扩散模型的方法}

近年来,扩散模型在图像生成领域取得了突破性进展,相比GAN具有训练稳定、生成质量高、模式覆盖好等优势。Ho等人\cite{hoDenoisingDiffusionProbabilistic2020}提出的去噪扩散概率模型(Denoising Diffusion Probabilistic Models, DDPM)通过学习数据分布的逆向去噪过程,实现了高质量的图像生成。Song等人\cite{songScoreBasedGenerativeModeling2021}从随机微分方程的角度统一了扩散模型的理论框架,提出了基于分数的生成模型并设计了更高效的采样算法。Karras等人\cite{karrasEluvidatingDiffusionModels2022}进一步提出EDM,通过重新设计噪声调度、采样算法和网络架构,系统性地分析了扩散模型的各个组件,在生成质量和采样效率上取得了显著提升。Rombach等人\cite{rombachHighResolutionImageSynthesis2022}提出的隐空间扩散模型(Latent Diffusion Models, LDM)将扩散过程从像素空间转移到低维隐空间,显著提高了训练和采样效率,并支持灵活的条件注入机制。

在将扩散模型应用于逆向重建任务时,条件引导机制是关键技术。Dhariwal和Nichol\cite{dhariwalDiffusionModelsBeat2021}提出分类器引导(Classifier Guidance)方法,在采样过程中利用预训练分类器的梯度信息修正采样轨迹,显著提升了生成样本的条件一致性和视觉质量。Ho和Salimans\cite{ho2022classifierfree}进一步提出无分类器引导(Classifier-free Guidance)策略,通过在训练阶段同时学习条件生成和无条件生成,在采样时对比两者的分数函数差异来实现隐式引导,避免了额外分类器的计算开销。Song等人\cite{song2021denoising}提出的去噪扩散隐式模型(Denoising Diffusion Implicit Models, DDIM)通过引入确定性采样轨迹加速生成过程,为条件引导提供了更高效的基础框架。

针对特征匹配类逆问题,研究者提出了专门的引导策略。Chung等人\cite{chung2022diffusion}提出扩散后验采样方法,将条件约束建模为似然函数,通过贝叶斯推断框架在采样过程中融合先验和似然信息,为特征匹配任务提供了理论上更严谨的引导机制。Kawar等人\cite{kawar2022denoising}提出的去噪扩散修复模型(Denoising Diffusion Restoration Models, DDRM)通过在频域中分解观测算子实现高效的条件采样。Bansal等人\cite{bansal2023universal}提出通用引导方法,可以灵活地集成多种类型的条件约束(包括连续特征向量、离散类别标签、空间掩码等),通过统一的梯度计算接口实现对采样过程的精确控制。Chung等人\cite{chung2023comebacksampler}从随机收缩的角度分析了条件扩散模型的收敛性,提出了加速采样算法,在保证生成质量的同时显著减少了采样步数。

扩散模型及其引导策略为逆向重建攻击提供了新的技术范式。Struppek等人\cite{struppek2022plug}提出的Plug \& Play框架将攻击过程解耦为生成器训练和优化引导两个阶段,使得不同的预训练生成模型(生成对抗网络、扩散模型等)可以灵活应用于反演任务。这些工作表明,通过合理利用扩散模型的生成能力和条件控制机制,可以在保持高视觉保真度的同时精确满足身份匹配约束,显著提升了逆向重建攻击的威胁能力。


\subsection{国内外研究现状分析}

通过对国内外研究现状的系统梳理与深入分析,可以发现现有工作在以下关键方面仍存在局限性,亟待进一步突破:

(1)面向连续特征空间的扩散引导机制适配性不足。尽管扩散模型在条件图像生成领域表现卓越,但现有的引导策略主要针对离散类别标签或模态间对齐设计。人脸识别系统中的特征嵌入向量具有高维、连续及特定的度量空间结构,直接套用通用的离散引导机制往往难以捕捉其精细的几何特性。

(2)生成感知质量与身份一致性的协同优化机制缺失。现有逆向重建技术普遍面临质量和一致性的权衡困境:基于优化的方法虽然能精确拟合特征,但重建图像常伴随高频噪声与非自然伪影;基于生成对抗网络的方法虽能保证视觉真实性,却易受模式崩塌影响导致身份信息漂移。尽管扩散模型具备更强的分布建模能力,但如何在去噪采样过程中动态平衡感知先验与身份语义约束,避免顾此失彼,目前尚缺乏系统的理论指导与有效的控制策略。

基于上述分析,本研究将从以下两个维度切入,旨在构建高效、高保真的逆向重建攻击框架:

(1)设计适配连续特征嵌入的动态引导与生成机制。针对特征模板逆向重建任务,本研究将采用EDM\cite{karrasEluvidatingDiffusionModels2022}作为生成骨干网络,提出面向连续特征向量的嵌入一致性引导策略。通过深入分析扩散采样过程中的信噪比演变规律,设计自适应的动态引导算法,在采样轨迹中融合识别模型的特征空间约束与生成模型的自然图像先验,实现对目标特征的精确逆向映射。

(2)构建基于强生成先验与参数高效微调的协同反演框架。针对模型反演攻击任务,本研究将探索利用预训练扩散换脸模型中蕴含的丰富人脸结构与纹理先验,结合低秩适配技术进行参数高效微调。通过将通用的人脸生成能力快速迁移至目标身份域,并协同分类器置信度引导,实现在有限计算资源下对目标身份的高保真、高一致性重建,解决传统方法在质量与效率上的瓶颈。

通过上述研究,本文旨在为人脸识别系统的隐私安全评估提供更具威胁性、更系统的技术方案,从攻防对抗的角度推动生物特征保护技术的发展。

\section{本文的研究内容及章节安排}\label{sec:thesis_structure}

\subsection{主要研究内容}

为方便后文形式化与讨论,本节对本文研究的两类主要逆向重建攻击任务给出明确定义与形式化描述:

模板逆向重建攻击针对基于特征提取器的检索型人脸识别系统。设特征提取器为 $F: \mathbb{R}^{H \times W \times C} \to \mathbb{R}^d$,将输入图像 $x$ 映射为 $d$ 维特征嵌入 $e = F(x)$。攻击者通过数据库泄露等途径获得目标身份的特征模板 $t = F(x_0) \in \mathbb{R}^d$,其目标是构造逆映射函数 $g: \mathbb{R}^d \to \mathbb{R}^{H \times W \times C}$,生成图像 $\hat{x} = g(t)$,使得其满足两个特性:(1)身份一致性:$F(\hat{x}) \approx t$,即生成图像的特征嵌入与目标模板在度量空间中接近,满足 $\text{sim}(F(\hat{x}), t) > \tau$,其中 $\text{sim}(\cdot, \cdot)$ 为相似度度量函数(常采用余弦相似度),$\tau$ 为识别系统的验证阈值;(2)视觉真实性:$\hat{x}$ 具有自然的人脸特征和纹理细节,在感知质量上与真实人脸图像无明显差异。

形式化地,模板逆向重建攻击问题可建模为如下优化问题:
\begin{equation}
\hat{x} = \arg\min_x \; \mathcal{L}_{embed}(F(x),t) + \lambda_{perc}\mathcal{L}_{perc}(x) + \lambda_{reg}R(x)
\end{equation}
其中 $\mathcal{L}_{embed}$ 为嵌入空间距离损失,可采用余弦距离 $1-\cos(F(x),t)$ 或欧氏距离 $\|F(x)-t\|_2^2$;$\mathcal{L}_{perc}$ 为感知质量损失;$R(\cdot)$ 为正则化项或先验约束;$\lambda_{perc}$ 和 $\lambda_{reg}$ 为权衡系数。

模型反演攻击针对基于分类的端到端人脸识别系统。设分类模型为 $F_\theta: \mathcal{X} \to \mathbb{R}^C$,将输入图像 $x \in \mathcal{X}$ 映射为 $C$ 个类别的置信度向量。攻击者针对目标类别 $y_{\text{target}} \in \{1, 2, \ldots, C\}$,通过访问模型输出与梯度信息,生成图像集合 $\{\hat{x}_1, \hat{x}_2, \ldots, \hat{x}_K\}$,使其同时满足以下条件:(1)分类器认同度:生成图像 $\hat{x}_k$ 被目标分类器以高置信度识别为目标类别,即 $F_\theta(\hat{x}_k)_{y_{\text{target}}} \geq \tau$;(2)感知真实性:生成图像在视觉上与真实人脸无明显差异,感知距离 $d_{\text{perc}}(\hat{x}_k, \mathcal{X}_{\text{real}}) \leq \delta$;(3)身份相关性:生成图像包含目标类别对应身份的特征信息,与训练样本相似度 $\text{sim}(\hat{x}_k, \mathcal{X}_{y_{\text{target}}}) \geq \gamma$;(4)多样性:生成的多个样本覆盖不同姿态、表情、光照等变化,避免模式崩溃。综合上述目标,MIA可建模为约束优化问题:
\begin{equation}
\begin{aligned}
    \max_{\{\hat{x}_k\}_{k=1}^K} \quad & \sum_{k=1}^K F_\theta(\hat{x}_k)_{y_{\text{target}}} \\\
    \text{s.t.} \quad & d_{\text{perc}}(\hat{x}_k, \mathcal{X}_{\text{real}}) \leq \delta, \quad \forall k, \\\
    & \text{sim}(\hat{x}_k, \mathcal{X}_{y_{\text{target}}}) \geq \gamma, \quad \forall k, \\\
    & \text{Var}(\{\hat{x}_k\}_{k=1}^K) \geq \rho.
\end{aligned}
\end{equation}
本文采用生成模型驱动的攻击策略:利用预训练扩散换脸模型的先验知识,通过参数高效微调快速适配目标分类器,在采样过程中融入分类器梯度引导,在保证视觉质量的同时实现精确的身份匹配。

本研究围绕人脸识别系统中的逆向重建攻击问题展开,针对模板逆向重建攻击和模型反演攻击两类典型威胁场景,分别设计基于扩散模型的攻击方法。研究工作涵盖理论建模、方法设计与实验验证三个层面,具体内容如下:

\subsubsection{基于扩散模型的模板逆向重建方法}

针对特征提取模型的模板逆向重建攻击方法,本文提出基于扩散模型的政击框架,采用EDM\cite{karrasEluvidatingDiffusionModels2022}作为生成骨干网络。该方法通过融合像素空间重建与特征空间匹配的混合损失函数,实现了视觉质量与身份一致性的协同优化。具体而言,本文设计了包含三个核心组件的损失架构:(1)像素空间重建损失,采用EDM的标准去噪目标确保基础视觉质量;(2)角度约束的特征空间损失,通过对比学习机制显式拉开生成特征与负样本的角度距离,与ArcFace的超球面几何结构对齐,增强特征匹配的判别性;(3)多样性约束,通过最大化批内样本特征的角度距离防止模式崩塌。为自动平衡各损失项的相对权重,本文引入任务不确定性加权框架,将像素重建不确定性参数$\sigma_p$和特征匹配不确定性参数$\sigma_f$作为可学习参数与网络参数同步优化,避免手动调参的繁琐并提升训练稳定性。训练采用两阶段策略:预热阶段专注于建立基础生成能力,主训练阶段启用完整混合损失实现精细优化。推理阶段通过模板条件引导机制,在EDM确定性采样过程中融入目标特征模板的梯度信息,从纯噪声逐步生成与目标模板匹配的高质量人脸图像。该方法在第~\ref{chap:TIA}~章中详细阐述,并在第~\ref{chap:Results}~章中进行全面验证。

\subsubsection{基于扩散换脸与低秩适配微调的模型反演方法}

针对分类模型的模型反演攻击方法,本文选择扩散换脸模型作为生成先验,利用其显式的身份-属性解耦机制和标准化的身份控制接口实现高保真身份重建。换脸模型标准推理流程需要真实目标图像提供身份嵌入,而模型反演攻击场景仅有类别标签,本文针对这一核心矛盾设计了标签条件嵌入层,采用多层感知机将类别标签的独热编码映射为身份嵌入向量并进行归一化,无缝替代原有的ArcFace身份编码器。为适配新的嵌入分布同时保留换脸模型的生成先验,本文采用低秩适配技术进行参数高效微调,仅需训练原模型1\%-5\%参数量。本文构建了多目标损失框架:扩散先验损失保持生成质量,top-k max-margin损失驱动分类器攻击,p-reg特征正则化损失稳定优化轨迹,对比学习形式的身份一致性损失确保类别映射稳定性,LPIPS感知质量损失保证视觉真实度,正则化损失防止过拟合。通过任务不确定性加权机制将各损失项的不确定性参数$\{\sigma_i\}$作为可学习参数与网络参数联合优化,实现各目标的自动平衡。训练采用渐进式三阶段策略:图像条件预热阶段利用真实图像嵌入预训练LoRA建立换脸映射能力,混合条件过渡阶段通过余弦退火调度逐步从图像嵌入切换至标签嵌入实现平滑模态迁移,纯标签条件适配阶段激活完整损失函数优化分类器攻击效果。该方法在第~\ref{chap:MIA}~章中详细阐述,并在第~\ref{chap:Results}~章通过多个实验场景展示其在攻击准确率与生成保真度上的显著优势。

\subsection{章节安排}

本文共分为五章,各章节内容安排如下:

第~\ref{chap:introduction}~章为绪论。阐述人脸识别技术的应用现状与面临的隐私威胁,综述国内外研究进展,形式化定义模板逆向与模型反演两类攻击任务,说明本文的研究内容与创新贡献。

第~\ref{chap:theory}~章为理论基础。介绍人脸识别模型原理、扩散模型的技术要点、参数高效微调技术中低秩适配的应用方法,以及本研究采用的评估指标体系。

第~\ref{chap:TIA}~章为面向人脸特征提取模型的逆向重建方法。针对基于模板匹配的检索型人脸识别系统,提出基于扩散模型的模板逆向重建攻击框架,设计混合损失函数与任务不确定性加权机制,通过两阶段训练与模板条件引导实现高质量人脸重建。

第~\ref{chap:MIA}~章为基于换脸先验的模型反演攻击方法。针对基于分类的端到端人脸识别系统,提出融合扩散换脸先验、标签条件嵌入与LoRA参数高效微调的模型反演攻击框架,通过多目标优化与渐进式三阶段训练实现分类器攻击。

第~\ref{chap:Results}~章为实验结果与性能分析。通过在多个数据集与目标识别模型上的系统实验,验证本文所提出的基于角度约束扩散模型的模板逆向方法和基于渐进式训练的模型反演方法的有效性与先进性,分析关键模块的贡献与超参数的影响,展示相比现有方法的显著优势。

% Local Variables:
% TeX-master: "../main"
% TeX-engine: xetex
% End:
