% !Mode:: "TeX:UTF-8"
\begin{conclusions}

随着人脸识别系统在身份认证、公共安全等领域的广泛部署,其特征模板与模型输出的隐私泄露风险日益凸显。本研究针对模板逆向攻击与模型反演攻击两类威胁,提出两种高效逆向重建方法,从攻击视角系统研究了人脸识别系统的隐私脆弱性。

针对基于模板匹配的检索型人脸识别系统,本研究提出基于角度约束对比学习的模板逆向重建方法。该方法利用单位超球面几何特性设计角度约束对比学习损失,使特征优化方向与识别器决策边界精确对齐,相比传统欧氏距离损失具有更好的梯度稳定性。引入任务不确定性加权框架通过可学习参数自动平衡像素重建与特征匹配,避免手动调参。采用类内多样性约束防止模式崩塌,结合模板条件梯度引导机制提升特征匹配精度。实验表明,该方法在误识率为$10^{-2}$场景下在MOBIO数据集上攻击成功率达97.38\%,在误识率为$10^{-3}$的高安全场景下在MOBIO数据集上攻击成功率达87.87\%,在LFW数据集上FID降至18.27,显著优于现有方法。

针对基于分类的端到端人脸识别系统,本研究提出基于换脸先验迁移的多目标自适应模型反演方法。该方法选择扩散换脸模型作为生成先验,利用其显式的身份属性解耦机制实现高保真重建。针对换脸模型需要真实目标图像而攻击场景仅有类别标签的矛盾,设计标签条件嵌入层将类别标签映射为身份嵌入向量。采用低秩适配技术进行参数高效微调,仅需训练原模型1\%至5\%参数量即可适配新的嵌入分布。构建多目标损失框架涵盖扩散先验保真度、分类引导有效性等五个维度,通过任务不确定性加权机制实现自动平衡。采用渐进式三阶段训练策略,通过图像条件预热、混合条件过渡与纯标签条件适配实现从图像到标签的平滑模态迁移。该方法在ArcFace目标分类器上达到94.87\%目标准确率和83.15\%评估准确率,在多个分类器架构上表现稳定,验证了良好的跨架构泛化能力。

本研究工作揭示了人脸识别系统在特征表示与模型输出层面存在的隐私脆弱性。实验结果表明,即使在高安全阈值下,模板逆向攻击仍可实现超过70\%的攻击成功率,模型反演攻击达到超过94\%的目标准确率,证明当前系统在面对逆向重建攻击时缺乏足够的防护能力。研究成果可作为安全测试工具帮助开发者识别隐私风险,为设计更加安全可靠的生物特征识别系统提供参考。

\end{conclusions}
