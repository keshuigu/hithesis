% !Mode:: "TeX:UTF-8"
\begin{conclusions}

本文针对人脸识别系统的隐私泄露风险,系统研究了模板逆向攻击与模型反演攻击两类威胁,提出了基于扩散模型的高效逆向重建方法。

本文的主要贡献包括:第一,建立了模板逆向攻击与模型反演攻击的形式化框架,明确了攻击者的知识边界与能力假设。第二,提出了基于扩散生成模型的模板逆向方法,通过角度约束特征匹配对齐ArcFace超球面特征空间,引入任务不确定性加权框架平衡像素去噪与特征感知优化,采用多样性正则化防止特征崩塌。实验表明,该方法在多个标准数据集(MOBIO、LFW、AgeDB、IJB-C)上显著超越现有方法。第三,提出了基于换脸先验与参数高效微调的模型反演方法,通过标签条件嵌入层和渐进式三阶段训练实现图像到标签的平滑模态迁移。采用LoRA技术仅需微调约1\%参数即可实现精确匹配,在保持高攻击准确率的同时提升生成保真度。第四,建立了涵盖攻击有效性、生成质量与像素保真度的多维评估指标体系。

本文工作揭示了人脸识别系统在特征表示与模型输出层面的隐私脆弱性,证明了生成先验选择对攻击性能的决定性影响,展示了参数高效微调在隐私攻击场景中的应用潜力。所建立的方法可作为安全测试工具帮助开发者评估隐私风险,为人脸识别系统的安全防护提供重要参考。

本文研究主要聚焦于白盒攻击场景,在黑盒场景验证、低质量数据泛化、计算效率优化和理论可解释性等方面仍有待深入。未来研究可扩展到更复杂场景,提升跨架构泛化能力,构建完整的攻防对抗体系。

研究成果为理解和评估人脸识别系统的隐私风险提供了理论基础与技术支撑,期待推动生物特征隐私安全研究的发展,共同构建更加安全、可信的智能识别系统。

\end{conclusions}
