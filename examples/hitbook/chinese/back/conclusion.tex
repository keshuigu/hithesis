% !Mode:: "TeX:UTF-8"
\begin{conclusions}

随着人脸识别系统在身份认证等关键领域的广泛部署,特征模板与模型输出的隐私泄露风险日益凸显。本研究针对模板逆向攻击与模型反演攻击两类典型威胁,提出两种基于扩散模型的高效逆向重建方法,系统揭示了人脸识别系统的隐私脆弱性。

针对检索型人脸识别系统,本文提出基于角度约束对比学习的模板逆向重建方法。该方法以EDM扩散模型为生成骨干,针对人脸识别系统的单位超球面特征空间,创新性地设计了角度约束对比学习损失函数。通过引入负样本对比机制,该损失函数显式拉大生成特征与非目标类别的角度距离,使优化方向与识别器决策边界精确对齐,相比传统欧氏距离损失具有更强的判别能力和梯度稳定性。本文引入任务不确定性加权框架,将像素重建与特征匹配的不确定性参数作为可学习变量与网络参数联合优化,实现了多目标损失的自动平衡,避免了手动调参的繁琐。此外,通过类内多样性约束有效防止模式崩塌,在推理阶段采用模板条件梯度引导机制动态调整采样轨迹,显著提升特征匹配精度。实验表明,该方法在MOBIO数据集上于误识率$10^{-2}$和$10^{-3}$场景下分别达到97.38\%和87.87\%的攻击成功率,平均攻击成功率87.54\%;在LFW数据集上FID降至18.27,在多个识别器架构上均取得最优性能,显著优于现有方法。

针对分类型人脸识别系统,本文提出基于换脸先验迁移的多目标自适应模型反演方法。该方法创新性地将扩散换脸模型应用于模型反演攻击场景,充分利用其身份属性显式解耦的优势。针对换脸模型依赖图像条件而攻击场景仅有类别标签的核心矛盾,本文设计了标签条件嵌入层,采用多层感知机将离散类别标签映射为连续身份嵌入向量,建立从标签空间到嵌入空间的可学习桥梁。为适配新的嵌入分布同时保留预训练模型的生成先验,本文采用低秩适配技术进行参数高效微调,仅需训练原模型1\%至5\%的参数量即可实现高质量适配,有效避免过拟合并降低计算开销。本文构建了涵盖五个维度的多目标损失框架,通过任务不确定性加权机制实现扩散先验保真度、分类器攻击有效性、特征正则化、身份一致性与感知质量的协同优化。特别地,本文提出渐进式三阶段训练策略,通过图像条件预热、混合条件过渡与纯标签条件适配三个阶段,实现从图像到标签的平滑模态迁移,消融实验表明该策略使目标准确率从单阶段的70.85\%提升至94.87\%。在ArcFace分类器上,该方法达到94.87\%目标准确率和83.15\%评估准确率,FID为23.26;在IR152与Face.evoLVe等多个分类器架构上表现稳定,验证了良好的跨架构泛化能力。

本研究揭示了即使在高安全阈值下,当前人脸识别系统仍面临严重的隐私泄露风险。针对本文提出的攻击方法,现有防御措施主要包括特征空间防御(可撤销变换、特征加密)、模型输出防御(差分隐私、置信度限制)与对抗训练防御三类。这些防御策略在一定程度上可降低攻击有效性:特征混淆技术破坏超球面几何结构,削弱角度约束对比学习的判别能力;差分隐私机制干扰梯度信息,降低分类器引导的精确性。然而,现有防御方法普遍存在识别性能下降、计算开销增加等代价,且在强攻击者模型下的有效性仍待验证。

展望未来,生物特征隐私保护领域的研究可从以下方向深化:(1)在理论层面,建立攻防对抗的统一数学框架,量化隐私保护强度与系统性能损失之间的本质权衡关系,为防御机制设计提供理论指导;(2)在技术层面,探索轻量级防御机制,在保持识别精度的前提下有效抑制逆向重建,并研究多层次防御策略的协同效应;(3)在应用层面,针对不同安全等级场景设计自适应防护方案,平衡隐私保护需求与用户体验;(4)在评估层面,构建标准化的隐私风险评估体系,为系统安全性提供可量化的度量标准。本文的研究成果为理解当前防御措施的局限性提供了重要参考,也为设计更加安全可靠的生物特征识别系统奠定了基础。

\end{conclusions}
