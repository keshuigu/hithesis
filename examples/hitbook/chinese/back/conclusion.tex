% !Mode:: "TeX:UTF-8"
\begin{conclusions}

随着深度学习技术的快速发展,人脸识别系统已在身份认证、访问控制等关键领域广泛部署。然而,系统在实现高识别性能的同时,不可避免地在特征表示与模型输出中暴露了与原始图像高度相关的语义信息,为隐私攻击提供了可能。本文针对人脸识别系统面临的隐私泄露风险,系统研究了模板反演攻击与模型反演攻击两类典型威胁,提出了基于扩散模型的高效逆向重建方法,建立了完整的评估体系。

本文的主要贡献包括:第一,建立了模板反演攻击与模型反演攻击的形式化框架,明确了攻击者的知识边界与能力假设,为评估识别系统的最大隐私泄露风险提供了理论基准。第二,提出了基于明晰扩散模型的模板反演方法,通过角度约束特征匹配对齐ArcFace的超球面特征空间,引入任务不确定性加权框架自动平衡像素去噪与特征感知的协同优化。实验表明,该方法在LFW数据集上于误识率$10^{-3}$阈值下攻击成功率达到98.50\%,相比现有方法提升6.33个百分点。第三,提出了基于换脸先验与参数高效微调的模型反演方法,首次将换脸模型的身份解耦能力引入模型反演任务,通过标签条件嵌入层和渐进式三阶段训练策略实现从图像到标签的平滑模态迁移。采用低秩适配技术仅需微调约1\%的参数即可实现对目标分类器的精确匹配,在VGGFace2数据集上FID值降低约28\%。第四,建立了涵盖攻击成功率、生成质量与计算效率的多维度评估体系。

本文工作具有重要的理论与实践意义。理论层面,系统揭示了人脸识别系统在特征表示与模型输出层面的隐私脆弱性,证明了生成先验选择对攻击性能的决定性影响,展示了参数高效微调技术在隐私攻击场景中的应用潜力。实践层面,所建立的攻击方法可作为安全测试工具帮助系统开发者评估隐私风险,揭示的攻击机理为防御研究指明了方向。

本文研究仍存在一些局限性。首先,主要聚焦于白盒攻击场景,实际应用中更常见的黑盒场景尚未系统验证。其次,实验主要基于高质量公开数据集,方法在野外低质量数据上的泛化能力有待验证。第三,扩散模型的训练与推理仍需较大计算资源。第四,对深层机制的理论解释仍显不足。未来研究可扩展到更真实复杂的场景,提升计算效率与跨架构泛化能力,增强方法的可解释性,构建完整的攻防对抗体系。

人脸识别技术的广泛应用在提供便捷服务的同时也带来了严峻的隐私安全挑战。本文通过系统深入的研究,揭示了人脸识别系统的隐私脆弱性,提出了基于扩散模型的高效逆向重建方法,建立了全面的评估体系。研究成果为理解和评估人脸识别系统的隐私风险提供了理论基础与技术支撑,为防御机制的设计指明了方向,期待推动生物特征隐私安全研究的发展,共同构建更加安全、可信的智能识别系统。

\end{conclusions}
